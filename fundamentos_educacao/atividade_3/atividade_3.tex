\documentclass[a4paper, 12pt]{article}

\usepackage[top=2cm, bottom=2cm, left=2.5cm, right=2.5cm]{geometry}
\usepackage[utf8]{inputenc}
\usepackage{array}
\usepackage{graphicx}

\graphicspath{{img/}}

\begin{document}
\begin{flushleft}\includegraphics{logo}\\
\textbf{UNIVERSIDADE ESTADUAL DE PONTA GROSSA} \\
SISTEMA UNIVERSIDADE ABERTA DO BRASIL - UAB \\
\underline{Licenciatura em Matemática | Polo UAB em Jacarezinho}\end{flushleft} 
\textbf{ALUNO:} Ricardo Medeiros da Costa Junior   \textbf{RA:} 151774301 \\
\textbf{DISCIPLINA:} Fundamentos da Educação\\
\textbf{ATIVIDADE:} Atividade 3 - Tarefa: "A educação tem uma história"\\
\textbf{TUTOR(A):} Adilane de Assis Ferreira\\
\textbf{PERÍODO:} Terceiro Período \\ \\
Após a proclamação da república em 1889, instarou-se no Brasil o período chamado República Velha (1889-1830). Este período não ocasionou mudanças significativas na sociedade e na educação, pois foi conduzida pela tradicional elite agrária. Nesse período são realizadas reformas no ensino de inspiração positivista, que procuram romper com o humanismo tradicional e implantar um ensino enciclopédico e científico.\\
Em 1930, na era denominada era Vargas, começa a formação do moderno estado brasileiro. A urbanização e a industrialização criam novas necessidades educacionais. A escola nova fustiga a pedagogia tradicional que, até então, mantinha a hegemonia na educação brasileira. Apesar disso, a escola continuou a ser dual: primário/profissionalizante para os pobres e secundário/superior para os abastados.\\
Em 1961, é aprovada a primeira Lei de Diretrizes e Bases da Educação Nacional, de inspiração liberal com influência escolanovista. Os anos 60 são marcados por movimentos de cultura e de educação popular, com destaque para o trabalho de Paulo Freire. Com a implantação do regime militar esses movimentos são sufocados ou caem na clandestinidade.\\
Durante o regime militar, a educação brasileira sofre grande influência dos Estados Unidos, sendo implantadas leis que reformularam o ensino superior e os ensinos de primeiro e segundo graus, ambas de inspiração tecnicista e responsáveis por um aligeiramento do ensino.\\
Com a abertura democrática, que marca os anos 80, a educação toma um novo impulso, graças a atuação dos partidos de esquerda que apresentaram propostas de acesso e permanência na escola, de gestão democrática e de melhor qualificação do magistério. Em 1988, é promulgada a nova constituição e, em 1996, é aprovada a lei de Diretrizes e Bases da Educação Nacional.\\
Os anos 90 são marcados pela mudança no cenário internacional, pelas políticas neoliberais para a educação e pela globalização da economia. Políticas de qualidade total, típicas do setor industrial, são aplicadas na educação. As relações entre educação e política são minimizadas e os problemas da escola passam a ser considerados problemas de gestão, de avaliação e de qualificação dos docentes.\\
Sobre a minha consideração; apesar de não ter realizado uma pesquisa científica, nem ter me baseado em nenhuma fundamentação teórica, de acordo com dados apresentados no livro didático, pode-se afirmar que ouve um certo avanço na educação brasileira. No entanto, como disse o senador Cristovam Buarque, a educação no Brasil avançou em comparação com ela mesmo, porém em relação ao resto do mundo retrocedemos. Os dados exibidos no livro didático mostram um notável avanço quantitativo, ou seja, há mais alunos na escola, todavia os resultados qualitativos ainda são pífios.
\end{document}
