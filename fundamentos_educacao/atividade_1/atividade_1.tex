\documentclass[a4paper, 12pt]{article}

\usepackage[top=2cm, bottom=2cm, left=2.5cm, right=2.5cm]{geometry}
\usepackage[utf8]{inputenc}
\usepackage{array}
\usepackage{graphicx}

\graphicspath{{img/}}

\begin{document}
\begin{flushleft}\includegraphics{logo}\\
\textbf{UNIVERSIDADE ESTADUAL DE PONTA GROSSA} \\
SISTEMA UNIVERSIDADE ABERTA DO BRASIL - UAB \\
\underline{Licenciatura em Matemática | Polo UAB em Jacarezinho}\end{flushleft} 
\textbf{ALUNO:} Ricardo Medeiros da Costa Junior   \textbf{RA:} 151774301 \\
\textbf{DISCIPLINA:} Fundamentos da Educação \\
\textbf{ATIVIDADE:} Atividade 1 - Tarefa: "Filosofia, Ciências Pedagógicas e Formação do Educador" \\
\textbf{TUTOR(A):} Adilane de Assis Ferreira\\
\textbf{PERÍODO:} Terceiro Período \\ \\
    \begin{tabular}{|m{2.5cm}|m{3cm}|m{3cm}|m{3cm}|m{3cm}|}
      \hline
       & Psicologia da Educação & Sociologia da Educação & História da Educação & Filosofia da Educação\\
      \hline
      Objeto de Estudo & Segundo Coll (1992, p.51) a Psicologia da Educação se diferencia das outras áreas ou domínios da Psicologia, centrando seu interesse no estudo dos processos psicológicos que se relacionam com a participação das pessoas em atividades educativas & A sociologia da educação explicita a relação da escola com a estrutura social, ampliando a compreensão de questões como: relações professor-aluno, significado social e político do currículo, função social da escola e da educação escolar, sua organização, seus sujeitos e suas práticas & A educação, sendo um fenômeno universal, que toma formas diversas no tempo e no espaço é objeto dos estudos históricos. É a tarefa da história da educação descrever e analisar o processo educativo, desde as suas formas primitivas, realizadas por meio da convivência, da experiência, da imitação, até a realidade educacional extremamente complexa das sociedades contemporâneas & Segundo Saviani(1986), é uma reflexão sobre os problemas que surgem no ato de educar, pois a educação, enquanto prática social, deve responder aos anseios humanos de aperfeiçoamento do homem e da sociedade. \\
      \hline
      Contribuições à formação do educador & Permite ao educador o conhecimento e a compreensão do educando, do seu desenvolvimento e do processo de ensino/aprendizagem, situando-o num contexto sócio-histórico determinado & A sociologia nos leva a relfetir sobre o educando enquanto ser social, que sofre influência e influencia os seus semelhantes, que é membro de diversos grupos, partícipe de uma comunidade & Contribui para que o professor compreenda melhor a realidade presente por meio do estudo da educação do passado, bem como favorece a aceitação da diversidade humana e social & Compete à Filosofia da Educação realizar um trabalho integrador no processo de formação do magistério, fomentando a interdisciplinaridade e articulando os componentes das demais ciências pedagógicas, mediante a reflexão rigorosa, sistemática, metódica e crítica sobre as várias dimensões da existência dos sujeitos/educandos
      \hline
    \end{tabular}
\end{document}
