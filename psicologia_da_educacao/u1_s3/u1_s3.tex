\documentclass[a4paper, 12pt]{article}

\usepackage[top=2cm, bottom=2cm, left=2.5cm, right=2.5cm]{geometry}
\usepackage[utf8]{inputenc}
\usepackage{array}

\begin{document}
\begin{enumerate}
  
  % Questão 1 
\item Analise se a afirmação abaixo está certa ou errada. Em seguida, traga argumentos que justifiquem sua resposta: \newline
  ``A Psicologia da Educação é um campo homogêneo de conhecimentos, caracterizado pela unidade teórica``. \\ \newline
  Falsa. É exatamento ao contrário. É uma área caracterizada pelos múltiplos enfoques teóricos.
  
% Questão 2
\item Tomando por base o critério epistemológico e das relações homem-sociedade, responda qual a ideia principal que caracteriza:
 \begin{enumerate}
 \item as concepções subjetivistas: Enfatizam os fatores internos do sujeito, conferem ao próprio sujeito um papel preponderante, pois a realidade está submetida a ele.
 \item as concepções objetivistas: Enfatizam fatores externos ao sujeito, consideram o primado do objeto sobre o sujeito.
 \item as concepções interacionistas: Partem do pressuposto do que os fatores interno ao sujeito e a realidade do objeto, se influenciam mutuamente.
 \end{enumerate} 
 \end{enumerate}

\end{document}
