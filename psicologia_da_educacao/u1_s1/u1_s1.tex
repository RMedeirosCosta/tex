\documentclass[a4paper, 12pt]{article}

\usepackage[top=2cm, bottom=2cm, left=2.5cm, right=2.5cm]{geometry}
\usepackage[utf8]{inputenc}

\begin{document}
 \begin{enumerate}
  
  % Questão 1 
   \item Responda:
     \begin{enumerate}
      \item Podemos dizer que a psicologia remonta aos antigos filósofos? Por quê?
      \item Quando, como e onde a Psicologia se constituiu desde os seus primórdios, que estão na base do conhecimento da psicologia?
      \item Que tipo de perguntas os homens fizeram desde os seus primórdios, que estão na base do conhecimento de psicologia?
      \item Wundt é considerado o fundador da Psicologia Científica. Faça uma síntese sobre ele e suas idéias.
     \end{enumerate}

   % Questão 2
   \item No quadro abaixo, complete com as idéias centrais dos filósofos que influenciaram o nascimento da Psicologia: \newline \newline
     \centering
     \begin{tabular}{|c|c|c|}
     \hline
     \textbf{DESCARTES} & \textbf{LOCKE} & \textbf{KANT} \\ \hline
     texto1 & texto2 & texto3 \\ \hline
     \end{tabular} \newline
     
   % Questão 3
   \item Faça um relato do caso de Amala e Kamala com suas palavras, explicando a importância do convívio social para os seres humanos.
 \end{enumerate}
 
\end{document}
