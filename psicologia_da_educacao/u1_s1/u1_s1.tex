\documentclass[a4paper, 12pt]{article}

\usepackage[top=2cm, bottom=2cm, left=2.5cm, right=2.5cm]{geometry}
\usepackage[utf8]{inputenc}

\begin{document}
\begin{enumerate}
  
  % Questão 1 
\item Responda: 
  \begin{enumerate}
  \item Podemos dizer que a psicologia remonta aos antigos filósofos? Por quê? \newline \newline
    Sim, porque os homens sempre se mostraram fascinados por compreender a conduta e a natureza humana. Muitas perguntas quais fazemos hoje já eram feitas pelos antigos filósofos. \newline
  \item Quando, como e onde a Psicologia se constituiu como um estudo de caráter científico? \newline \newline
  Foi na Alemanha, no século XIX. Wilhem Wundt criou seu laboratório de Psicologia Experimental na Universidade de Leipzig. \newline 
  \item Que tipo de perguntas os homens fizeram desde os seus primórdios, que estão na base do conhecimento de psicologia? \newline \newline
  Eles se perguntavam sobre a memória, a aprendizagem, a motivação, a percepção, a atividade onírica e o comportamento anormal. \newline
  \item Wundt é considerado o fundador da Psicologia Científica. Faça uma síntese sobre ele e suas idéias. \newline \newline
    Wundt seguiu mais os empiristas, ao invés dos racionalistas. Conseguiu chegar as percepções complexas por meio do estudo das sensações e imagens. Acreditava estudar os princípios de associação, nos quais elementos se combinavam para formar experiências complexas. A psicologia experimental criada por Wundt foi o sucedâneo mais próximo e com mais profundidade da psicologia empirista.
    \newline
  \end{enumerate}

  % Questão 2
\item No quadro abaixo, complete com as idéias centrais dos filósofos que influenciaram o nascimento da Psicologia: \newline \newline
  \centering
  \begin{tabular}{|c|c|c|}
    \hline
    \textbf{DESCARTES} & \textbf{LOCKE} & \textbf{KANT} \\ \hline
    Descartes fazia \\
    uma interpretação \\
    física e macânica \\
    do corpo (cujo  \\
    funcionamento ele
    entendia como   \\
    uma máquina). Esta\\
    interpretação  \\
    foi a base \\
    precursora dos \\
    estudos da \\
    Fisiologia, \\
    que abriram portas \\
    da psicologia \\
    comportamental do   \\
    estímulo e da resposta. & texto2 &\\
    texto3 \\ \hline
  \end{tabular} \newline
  
  % Questão 3
\item Faça um relato do caso de Amala e Kamala com suas palavras, explicando a importância do convívio social para os seres humanos.
\end{enumerate}

\end{document}
