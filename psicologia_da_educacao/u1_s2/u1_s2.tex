\documentclass[a4paper, 12pt]{article}

\usepackage[top=2cm, bottom=2cm, left=0.5cm, right=0.5cm]{geometry}
\usepackage[utf8]{inputenc}
\usepackage{array}

\begin{document}
\begin{enumerate}
  
  % Questão 1 
\item Escreva um texto de 10 a 15 linhas tratando da importância da Psicologia na Educação. \newline \newline
  Como definida no livro didático, a Psicologia da Educação faz ligação entre a Psicologia Científica e as necessidades pedagógicas do professor. Por intermédio das coordenadas teóricas, conceituais e metodológicas da Psicologia Científica, a Psicologia da Educação auxilia o professor a conhecer como o ser humano forma seus conceitos sobre o mundo, como a alfabetização se relaciona com a escrita e qual o papel do erro no processo cognitivo, por exemplo. A ajuda da psicologia acerca dessas reflexões, demonstram a importância da psicologia na educação.
  
% Questão 2
\item Complete o quadro abaixo, com os conceitos de: \newline \newline
  \centering
  \begin{tabular}{| m{3cm} | m{4cm} | m{4cm} | m{3cm} | m{3cm} |}
      \hline
    PSICOLOGIA DA EDUCAÇÃO & APRENDIZAGEM & DESENVOLVIMENTO & PSICOLOGIA DO DESENVOLVIMENTO & PSICOLOGIA DA APRENDIZAGEM \\ \hline
    Área do conhecimento que busca relações entre práticas educativas e psicologia & É o processo ativo de apropriação, pelo sujeito, do mundo em que vive & São as mudanças evolutivas das pessoas & Sua função é estudar como nascem e se desenvolvem as funções psicológicas que distinguem o homem de outras espécies & Estuda o complexo processo pelo qual as formas de pensar e os conhecimentos existentes numa sociedade são apropriadas pela criança \\ \hline
  \end{tabular} \newline

 \end{enumerate}

\end{document}
