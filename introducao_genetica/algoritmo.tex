% http://latexbr.blogspot.com.br
\documentclass[a4paper]{article}
\usepackage[utf8]{inputenc}
\usepackage[brazil]{babel}
\usepackage{amssymb,amsfonts,indentfirst}
\usepackage[centertags]{amsmath}
\usepackage{algpseudocode,algorithm}	% pseudo-codigo

% Declaracoes
%\algrenewcommand\algorithmicend{\textbf{fim}}
%\algrenewcommand\algorithmicdo{\textbf{faça}}
%\algrenewcommand\algorithmicwhile{\textbf{enquanto}}
%\algrenewcommand\algorithmicfor{\textbf{para}}
%\algrenewcommand\algorithmicif{\textbf{se}}
%\algrenewcommand\algorithmicthen{\textbf{então}}
%\algrenewcommand\algorithmicelse{\textbf{senão}}
%\algrenewcommand\algorithmicreturn{\textbf{devolve}}
%\algrenewcommand\algorithmicfunction{\textbf{função}}

%\algrenewtext{EndWhile}{\algorithmicend\ \algorithmicwhile}
%\algrenewtext{EndFor}{\algorithmicend\ \algorithmicfor}
%\algrenewtext{EndIf}{\algorithmicend\ \algorithmicif}
%\algrenewtext{EndFunction}{\algorithmicend\ \algorithmicfunction}

%\algnewcommand\algorithmicto{\textbf{até}}
%\algrenewtext{For}[3]%
%{\algorithmicfor\ #1 $\gets$ #2 \algorithmicto\ #3 \algorithmicdo}

% Novos comandos
\newcommand{\mei}{\leqslant} 		% menor ou igual

\title{Algoritmo}
\author{}
\date{\the\year}
\makeindex

\begin{document}
\maketitle
\thispagestyle{empty}

\begin{algorithm}
\caption{Identificação da ORF}
\begin{algorithmic}[1]
\Function{GetSequence}
$OrfSlideWindow \gets 1 
$CurrentPosition \gets 0 
\If {$OrfSlideWindow != 3} 
   \If {LocateStartCodon} 
     \State $initial \gets $CurrentPosition 
      \If {LocateStopCodon} 
         \State $final \gets $CurrentPosition 
          \If {LocatePromoterRegion \and $RegionIndex < $initial} 
              \State $PromoterRegion \gets $CurrentPosition 
              \State $CurrentPosition \gets $CurrentPosition+1 
              \If {LocateTataBox \and $TataBoxIndex < $initial} 
                 \State \print ORF: GetOrf[$initial, $final] 
              \Else 
                 \State $OrfSlideWindow \gets $OrdSlideWindow + 1 
                 
\Else 
    \State \print ORF não encontrada 
\EndFunction
\end{algorithmic}
\end{algorithm}

\begin{algorithm}
\caption{Identificação para identificação do Operon}
\begin{algorithmic}[1]
\Function{ReceiveORF}
$OrfSlideWindow \gets 1
$GroupExons \gets 0
$OP \gets 1
$J \gets 1
\If {$OrfSlideWindow != 3} 
   \If {LocateStartCodon[$CurrentPosition++]} 
     \State $initial[op] \gets $CurrentPosition 
      \If {LocateStopCodon[CurrentPosition+1]} 
         \State $final \gets $Position 
          \If {$OpStopCodon}
              \State $Final[op] \gets $CurrentPosition
              \State $GetSequence[op] \gets ($initial[op], $final[op])
              \State $GroupExons \gets $GroupExons + $J
              \State $OP++
              \State $CurrentPosition++
              \State $OpSlideWindow \gets $OpSlideWindow + 1
          \Else
              \State $OpSlideWindow \gets $OpSlideWindow + 1
              \Else
                 \If {$GroupExons > 1}
                     \State \Print GetSequence[op]
                 \Else
                 \State Print Não há operon
    \Else
       \State \Print não há operon        
                
\EndFunction
\end{algorithmic}
\end{algorithm}


\end{document}
