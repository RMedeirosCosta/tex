\documentclass[a4paper, 12pt]{article}

\usepackage[top=2cm, bottom=2cm, left=2.5cm, right=2.5cm]{geometry}
\usepackage[utf8]{inputenc}
\usepackage{array}
\usepackage{graphicx}

\graphicspath{{img/}}

\begin{document}
\begin{flushleft}\includegraphics{logo}\\
\textbf{UNIVERSIDADE ESTADUAL DE PONTA GROSSA} \\
SISTEMA UNIVERSIDADE ABERTA DO BRASIL - UAB \\
\underline{Licenciatura em Matemática | Polo UAB em Jacarezinho}\end{flushleft} 
\textbf{ALUNO:} Ricardo Medeiros da Costa Junior   \textbf{RA:} 151774301 \\
\textbf{DISCIPLINA:} Introdução ao Cálculo Diferencial e Integral \\
\textbf{ATIVIDADE:} Lista potenciação, radiciação e logaritmação \\ \\
\begin{enumerate}
\item Efetue as operações indicadas:
  \begin{enumerate}
  \item $ 2^{5} \cdot 2^{-3} \cdot 2^4 = $ \\
    $ 2^6 = 2 \cdot 2 \cdot 2 \cdot 2 \cdot 2 \cdot 2 = 64 $ \\
  \item $ \left ( 2^2 \cdot 2^{-3} \cdot 3^2 \cdot 3^{-1} \right )^4 = $ \\
    $ \left ( 2^{-1} \cdot 3 \right )^{⁴} = $ \\
    $ 2^{-4} \cdot 3^{4} = $ \\
    $ \frac{1}{16} \cdot 81 = \frac{81}{16} $ \\
  \item $ -5 + \left [ \left (-3 \right ) \cdot \left (-2 \right )^{-3} \right ] \div \left (-2 \right )^{-1} = $ \\
    $ -5 + \left [ \left (-3 \right ) \cdot \left (-\frac{1}{8} \right ) \right ] \div \left (-\frac{1}{2} \right ) = $ \\
    $ -5 + \frac{3}{8} \div \left (-\frac{1}{2} \right ) = $ \\
    $ -5 - \frac{3}{4} = \frac{-20-3}{4} = \frac{-23}{4} $ \\
  \item $ -2 -3² + \left [ \left ( -3 \right )^{-2} \cdot \left ( -2 \right )^{-1} \right ] \cdot \left ( -3 \right )^2 = $ \\
    $ -2 + 6 + \left [ \frac{1}{9} \cdot \left (- \frac{1}{2} \right ) \right ] \cdot 9 = $ \\
    $ -2 + 6 - \frac{1}{18} \cdot 9 = $ \\
    $ -2 + 6 - \frac{1}{2} = $ \\
    $ 4 - \frac{1}{2} = $ \\
    $ \frac{8-1}{2} = \frac{7}{2} $ \\            
  \end{enumerate}
\item Calcule:
  \begin{enumerate}
  \item $ \sqrt[5]{1} + \sqrt[6]{0} + \sqrt[4]{81} + \sqrt[3]{-125} - \sqrt[3]{64} $ \\
   $ 1 + 3 - 5 - 4 = -5 $ \\
  \item $ \sqrt[3]{189} + \sqrt[3]{56} = $ \\
    $ \sqrt[3]{27} \cdot \sqrt[3]{7} + \sqrt[3]{8} \cdot \sqrt[3]{7} = $ \\
    $ 3\sqrt[3]{7} + 2\sqrt[3]{7} = $ \\
    $ \sqrt[3]{7} \left ( 3 + 2 \right ) = $ \\
    $ 5\sqrt[3]{7} \approx 9,56 $ \\
  \item $ 3 \sqrt{12} - 2 \sqrt{27} + \sqrt{2} - \sqrt{75} + \sqrt{48} = $ \\
    $ 3 \sqrt{3} \cdot \sqrt{4} - 2 \sqrt{9} \cdot \sqrt{3} + \sqrt{2} - \sqrt{15} \cdot \sqrt{5} + \sqrt{4} \cdot \sqrt{4} \cdot \sqrt{3} = $ \\
    $ 6 \sqrt{3} - 6 \sqrt{3} + \sqrt{2} - \sqrt{3} \cdot \sqrt{5} \cdot \sqrt{5} + 4 \sqrt{3} = $ \\
    $ \sqrt{2} - 5\sqrt{3} + 4\sqrt{3} = $ \\
    $ \sqrt{2} - \left [ \sqrt{3} \left ( 5 + 4 \right ) \right ] = $ \\
    $ \sqrt{2} - 9\sqrt{3} \approx -14,17 $
  \end{enumerate}
\item Simplifique:
  \begin{enumerate}
  \item $ \sqrt[3]{2^8} = $ \\
    $ \sqrt[3]{2^3} \cdot \sqrt[3]{2^3} \cdot \sqrt[3]{2^2} = $ \\
    $ 4 \sqrt[3]{2^²} \  \vee \approx 6,35 $ \\ 
  \item $ \sqrt[3]{54} = $ \\
    $ \sqrt[3]{27} \cdot \sqrt[3]{2} = $ \\
    $ 3 \sqrt[3]{2} \ \vee \approx 1,26 $ \\
  \item $ 81^{\frac{3}{4}} = $ \\
    $ \sqrt[4]{81³} = $ \\
    $ \sqrt[4]{\left (3^4 \right)^3} = $ \\
    $ \sqrt[4]{3^{12}} = $ \\
    $ 3^{\frac{12}{4}} = $ \\
    $ 3^3 = 27 $ \\
  \item $ \left ( 0,0001 \right )^{\frac{-1}{4}} = $ \\
    $ \sqrt[4]{\left ( 1 \cdot 10^{-4} \right )^{-1}} = $ \\
    $ \sqrt[4]{\left ( 1^{-1} \cdot 10^4 \right ) } = $ \\
    $ \sqrt[4]{1^{-1}} \cdot \sqrt[4]{10^4} = $ \\
    $ 1 \cdot 10 = 10 $ 
  \end{enumerate}
\item Calcule:
  \begin{enumerate}
  \item $ \log_7 49 = 2 $ \\
  \item $ \log_3 \left ( \frac{1}{27} \right ) = $ \\
    $ \log_3 1 - \log_3 27 = -3 $ \\
  \item $ \log_{10} 0,001 = $ \\
    $ \log_{10} \left ( 1 \cdot 10^{-4} \right ) =  $ \\
    $  \log_{10} 1 + \log_{10} 10^{-4} = -4 $ \\
  \item $ \log_{\frac{1}{2}} \sqrt[3]{32} = $ \\
    $ \frac{1}{3} \log_{\frac{1}{2}} 32 = $ \\
    $ \frac{1}{3} \left ( \log_{\frac{1}{2}} 4 + \log_{\frac{1}{2}} 8 \right ) = $ \\
    $ \frac{1}{3} \left ( -2 -3 \right ) = \frac{1}{3} \left ( -5 \right ) = $ \\
    $ -\frac{5}{3} $ \\
  \item $ \log_8 \sqrt[5]{16} = $ \\
    $ \frac{1}{5} \log_{8} 16 = $ \\
    $ \frac{1}{5} \left ( \log_8 2 +  \log_8 2  +  \log_8 2 +  \log_8 2  \right ) = $ \\
    $ \frac{1}{5} \cdot \left ( \frac{4}{3} \right ) = \frac{4}{15} $ \\
  \item $ \log_{16} 0,125 = $ \\
    $ \log_{16} \left ( 5^3 \cdot 10^{-3} \right ) = $ \\
    $ \log_{16} \left ( 5^3 \cdot \left ( 2 \cdot 5 \right )^{-3} \right ) = $ \\
    $ \log_{16} \left ( 5^3 \cdot 2^{-3} \cdot 5^{-3} \right ) $ \\ 
    $ \log_{16} \left ( 2^{-3} \right ) = -3 \log_{16} 2 = $ \\
    $ -3 \log_{2^4} 2 = -3 \cdot \frac{1}{4} = -\frac{3}{4} $
  \end{enumerate}
\end{enumerate}
\end{document}
