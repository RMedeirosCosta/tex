\documentclass[a4paper, 12pt]{article}

\usepackage[top=2cm, bottom=2cm, left=2.5cm, right=2.5cm]{geometry}
\usepackage[utf8]{inputenc}
\usepackage{array}
\usepackage{graphicx}

\graphicspath{{img/}}

\begin{document}
\begin{flushleft}\includegraphics{logo}\\
\textbf{UNIVERSIDADE ESTADUAL DE PONTA GROSSA} \\
SISTEMA UNIVERSIDADE ABERTA DO BRASIL - UAB \\
\underline{Licenciatura em Matemática | Polo UAB em Jacarezinho}\end{flushleft} 
\textbf{ALUNO:} Ricardo Medeiros da Costa Junior   \textbf{RA:} 151774301 \\
\textbf{DISCIPLINA:} 509551 - Didática \\
\textbf{ATIVIDADE:} Atividade 1 - Conceito Didática (Valor: 10,0) \\ 
\textbf{TUTOR(A):} Giane Correia Silva \\
\textbf{PERÍODO:} Quarto \\\\
\begin{enumerate}
\item A partir das leituras realizadas e das reflexões apresentadas no vídeo, elabore com suas palavras uma definição para o termo “Didática”, expressando sua compreensão.\\\\
  Pelo que pude analisar no conteúdo do livro didático, do vídeo ``Didática Geral - A identificação da didática'' e das definições dos autores da tabela acima (algumas, inclusive, aparecem no livro didático); didática pode ser definida como uma ``ciência'' que tem por objetivo sistematizar o processo de ensino-aprendizagem. Partindo desta perspectiva, a didática que conhecemos hoje quebra o paradigma que professor nasce com o ``dom de ensinar''. De fato, ele aprende a ensinar, assim como aprende os conteúdos específicos que irá lecionar. A didática fornece um emaranhado de métodos que permite sistematizar o processo de ensino-aprendizagem, seja a organização dos conteúdos, a forma como serão exposto esses conteúdos, avaliação dentre outros. Para finalizar, a didática é o elo que conecta todos os professores de diferentes disciplinas, pois todos os professores deveriam saber didática antes de poder ensinar.
\item Apresente seu posicionamento sobre o papel da disciplina de Didática em seu processo de formação para a docência. \\\\
  A disciplina de Didática tem papel imprescindível, pois é por meio dela que posso realizar com eficiência a profissão de professor. De pouco vale ter completo domínio do conteúdo que irei ensinar se não consigo mediar esse conhecimento para os alunos que estarei lecionando. Portanto, ao meu ver, a disciplina de didática é tão importante quanto as disciplinas fundamentais na matemática, pois ela proporcionará realizar meu trabalho com eficiência aumentanto a probabilidade dos meus alunos conseguirem assimilar o conhecimento.
\end{enumerate}  
\end{document}
