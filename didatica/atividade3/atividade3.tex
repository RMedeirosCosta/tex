\documentclass[a4paper, 12pt]{article}

\usepackage[top=2cm, bottom=2cm, left=2.5cm, right=2.5cm]{geometry}
\usepackage[utf8]{inputenc}
\usepackage{array}
\usepackage{graphicx}

\graphicspath{{img/}}

\begin{document}
\begin{flushleft}\includegraphics{logo}\\
\textbf{UNIVERSIDADE ESTADUAL DE PONTA GROSSA} \\
SISTEMA UNIVERSIDADE ABERTA DO BRASIL - UAB \\
\underline{Licenciatura em Matemática | Polo UAB em Jacarezinho}\end{flushleft} 
\textbf{ALUNO:} Ricardo Medeiros da Costa Junior   \textbf{RA:} 151774301 \\
\textbf{DISCIPLINA:} 509551 - Didática \\
\textbf{ATIVIDADE:} Atividade 3 - Objetivos (Valor: 10,0) \\ 
\textbf{TUTOR(A):} Giane Correia Silva \\
\textbf{PERÍODO:} Quarto \\\\
\begin{enumerate}
\item 1 - Apresente a diferença entre os objetivos gerais e específicos e formule dois objetivos que ilustrem cada um dos níveis com o tema: Unidades de medida ou Geometria Plana (6ºano).\\\\
  Os objetivos gerais explicam as razões de uma decisão, bem como a importância dos argumentos de aprendizagem para um período de tempo mais abrangente. Desse modo, os objetivos gerais possuem uma dimensão ampla. Os objetivos específicos indicam o que os alunos poderão aprender com a aula proposta, assumindo, portanto, uma dimensão mais restrita, ou seja, uma temporalidade mais imediata - a curto prazo.
  \begin{description}
  \item[Unidades de medida]
    \begin{description}
    \item[Objetivos Gerais] Refletir sobre critérios que possibilitem entender cálculos e transformações sobre medidas de comprimento.\\
      Aprofundar o estudo de equivalências entre unidades de medidas.
    \item[Objetivos Específicos] Aprender equivalências entre unidades de medidas de comprimentos.\\
      Compreender os símbolos utilizados em cada unidade
    \end{description}
    \begin{description}
    \item[Objetivos Gerais] Encontrar a soma dos ângulos internos de um polígono regular decompondo-o em triângulos.\\
      Possibilitar a compreensão de um polígono regular
    \item[Objetivos Específicos] Identificar e nomear os ângulos, vértices e lados, suas unidades e instrumentos de medida.\\
      Desenhar figuras geométricas planas.      
    \end{description}
  \end{description}
  
\item Considerando a categorização proposta por Zabala, na qual ele discute a articulação entre os conteúdos e os objetivos, proponha objetivos para o conteúdo "operações com frações" (6ºano), que contemple os seguintes enfoques:
\begin{description}
\item[Assunto: Operações com fracões]
\item[Objetivo Conceitual:] Reconhecer a equivalência entre escritas fracionárias.
\item[Objetivo Procedimental:] Estabelecer relações entre divisão e frações por meio de exercícios
\item[Objetivo Atitudinal:] Reconhecer a necessidade de utilização de outros números em situações que os números naturais não são suficientes para exprimir o resultado de uma divisão.  
\end{description}
\end{enumerate}  
\end{document}
