\documentclass[a4paper, 12pt]{article}

\usepackage[top=2cm, bottom=2cm, left=2.5cm, right=2.5cm]{geometry}
\usepackage[utf8]{inputenc}
\usepackage{array}
\usepackage{graphicx}
\usepackage{hyperref}

\graphicspath{{img/}}

\begin{document}
\begin{flushleft}\includegraphics{logo}\\
\textbf{UNIVERSIDADE ESTADUAL DE PONTA GROSSA} \\
SISTEMA UNIVERSIDADE ABERTA DO BRASIL - UAB \\
\underline{Licenciatura em Matemática | Polo UAB em Jacarezinho}\end{flushleft} \\
\textbf{ALUNO:} Ricardo Medeiros da Costa Junior   \textbf{RA:} 151774301 \\
\textbf{DISCIPLINA:} Instrumentação para o Ensino da Matemática III \\
\textbf{ATIVIDADE:} Tarefa atividade 1\end{flushleft} \\

\begin{enumerate}
\item Assista aos vídeos 1, 2, 3, 4  postados na Unidade I e faça um texto de até duas páginas que aborde os temas neles tratados sobre “Tecnologia”, conforme os itens abaixo:(5 pontos)
  \begin{enumerate}
  \item Vídeo 1 – A história da tecnologia (1 ponto)
  \item Vídeo 2 – O uso da tecnologia em sala de aula (1 ponto)
  \item Vídeo 3 – A importância da tecnologia na educação (1 ponto)
  \item Vídeo 4 – Tratamento da informação a tecnologia no ensino da Matemática (1 ponto)
  \item Conclusão pessoal sobre a relação “Tecnologia e Educação” no processo ensino aprendizagem. (1 ponto)    
  \end{enumerate}
  \item Faça a leitura atenta do texto I  - Ciência e  Tecnologia e   do texto II -  Educação e Tecnologias: o novo ritmo da informação,   e a partir deles organize um glossário[1] explicando o conceito de cada um do termos abaixo: (5 pontos) (0,25 cada) \\ \\
    \begin{tabular}{|m{0.5cm}|m{5cm}|m{8cm}|}
      \hline
      1 & Ciberespaço & Teste \\ \\
      \hline
      2 & Ciência & \\ \\
      \hline
      3 & Ciência Aplicada & \\ \\
      \hline
      4 & Ciência Natural & \\ \\
      \hline
      5 & Ciência Pura & \\ \\
      \hline
      6 & Ciência Social & \\ \\
      \hline
      7 & Conhecimento Científico & \\ \\
      \hline
      8 & Ferramentas tecnológicas & \\ \\
      \hline
      9 & Informática & Teste & \\ \\
      \hline
      10 & Interatividade & \\ \\
      \hline
      11 & Linguagem digital & \\ \\
      \hline
      12 & Linguagem escrita & \\ \\
      \hline
      13 & Linguagem oral & \\ \\
      \hline
      14 & Método Científico & \\ \\
      \hline
      15 & Mídia & \\ \\
      \hline
      16 & NTICs & \\ \\
      \hline
      17 & Redes & \\ \\
      \hline
      18 & Técnica & \\ \\
      \hline
      19 & Tecnologia & \\ \\
      \hline
      20 & TICs & \\ \\
      \hline
    \end{tabular}
  \item Após a leitura  atenta do Texto III - Diretrizes para o uso de Tecnologias Educacionais   faça o que se pede nos itens abaixo: (20 pontos)
    \begin{enumerate}
    \item Faça um texto descritivo  resumindo as diretrizes para o uso das tecnologias educacionais nas escolas públicas paranaenses. (5 pontos)
    \item Acesse o portal dia-a-dia da educação  em  "Educadores"
      \url{http://www.educadores.diaadia.pr.gov.br/modules/conteudo/conteudo.php?conteudo=3}
      e liste os recursos didáticos direcionados para o ensino de matemática nele disponibilizados. (2 pontos)
    \item Assista o Vídeo 7 – Um triângulo fractal especial, na Unidade 1 em Videoteca. (13 pontos)
      \begin{itemize}
      \item Qual o tema tratado no vídeo? (1 ponto)
      \item O que é o triângulo de Sierpinski? (1 ponto)
      \item A qual bloco de conteúdos dos PCN esse tema de matemática se insere? (1 ponto)
      \item Quais conceitos o professor de matemática pode explorar na suas aulas? (2 pontos)
      \item Qual é a função de interação para o cálculo do perímetro do triângulo? (1 ponto)
      \item Resolva a seguinte situação-problema: (7 pontos) \\ \\
        Uma praça de formato triangular tem todos os lados medindo 30 metros de comprimento, cada um. Imagine que você precisa criar canteiros também de forma triangular repetindo o padrão da praça. Monte uma tabela considerando 5 (cinco) iterações e defina o tamanho do lado e a medida do perímetro em cada caso. \\ \\
      \begin{tabular}{|m{3cm} | m{5cm} | m{4cm}|}
      \hline
      Interação & Comprimento de cada lado & Perímetro \\ \\
      \hline
      1 & & \\ \\
      \hline
      2 & & \\ \\
      \hline
      3 & & \\ \\
      \hline
      4 & & \\ \\
      \hline
      5 & & \\ \\
      \hline
      \end{itemize}      
      \end{enumerate}
\end{enumerate}
\end{document}
