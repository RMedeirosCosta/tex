\documentclass[a4paper, 12pt]{article}

\usepackage[top=2cm, bottom=2cm, left=2.5cm, right=2.5cm]{geometry}
\usepackage[utf8]{inputenc}
\usepackage{array}
\usepackage{graphicx}
\usepackage{hyperref}

\graphicspath{{img/}}

\begin{document}
\begin{flushleft}\includegraphics{logo}\\
\textbf{UNIVERSIDADE ESTADUAL DE PONTA GROSSA} \\
SISTEMA UNIVERSIDADE ABERTA DO BRASIL - UAB \\
\underline{Licenciatura em Matemática | Polo UAB em Jacarezinho}\end{flushleft} \\
\textbf{ALUNO:} Ricardo Medeiros da Costa Junior   \textbf{RA:} 151774301 \\
\textbf{DISCIPLINA:} Instrumentação para o Ensino da Matemática III \\
\textbf{ATIVIDADE:} Tarefa atividade 1\end{flushleft} \\

\begin{enumerate}
\item Assista aos vídeos 1, 2, 3, 4  postados na Unidade I e faça um texto de até duas páginas que aborde os temas neles tratados sobre “Tecnologia”, conforme os itens abaixo:(5 pontos)
  \begin{enumerate}
  \item Vídeo 1 – A história da tecnologia (1 ponto) \\    
    Neste vídeo é demonstrado o uso da tecnologia na educação desde a era pré-histórica. Isso é feito em ordem cronológica, cujo é exibido o uso da tecnologia, o período em questão e um desenho animado no qual os personagens mostram como é feito o uso de determinada tecnologia. \\
    Como citado anteriormente, no início é exibido povos primitivos usando pinturas rupestres pra comunicação, por volta de 30.000 a.C. Após isso, é exposto os gregos, com a Academia de Pitágoras, por volta de 510 a.C, no qual a tecnologia destacada é a comunicação oral. Em seguida vemos um personagem chinês criando o papel, na China, por volta de 105 d.C. Na sequência é demonstrado a transcrição em manuscrito em 382 d.C. mas não é informado onde isso aconteceu. Após uma pesquisa pela internet, foi constatado que houve um Concílio de Roma em 382 d.C., e pela vestimenta do personagem semelhante ao clero da antiguidade e a arquitetura do ambiente no desenho, constata-se que essa etapa do vídeo, aconteceu em Roma. A próxima etapa do vídeo é quase na idade moderna em 1450, com a prensa móvel de Gutenberg, que provavelmente foi utilizada para fabricação de livros e outros recursos impressos, facilitando a disseminação do conhecimento. A seguir, de 1600 a 1800, tecnologias são utilizadas na educação pública, como o quadro negro e outros materiais escolares. Após isso, no século XX começa a Era Visual de 1910 a 1940, no qual foram utilizadas tecnologias de projeção de imagens. Em seguida, a Era da Informação, de 1960 a 1980, com a televisão, fitas VHS e recursos de áudio. De 1990 a 1991 caracteriza-se pela Era da Computação, no qual os computadores domésticos começam a se tornar acessíveis nos ambientes acadêmicos. Em 1995 começa Era Digital, com o poderoso recurso da internet, cujo é possível se comunicar com várias partes do mundo por meio do computador. E por fim, nos anos 2000 inicia-se a Era da Interatividade, no qual é possível utilizarmos tecnologias como smartphones, laptops e outros dispositivos para aprendizagem.  \\
  \item Vídeo 2 – O uso da tecnologia em sala de aula (1 ponto) \\
    O vídeo começa enfatizando a banalização da tecnologia, que esta por toda parte. Em seguida é ressaltado que os professores devem se preparar para fazer o melhor uso das tecnologias disponíveis, para deixar de lado velhos métodos que podem ser aperfeiçoados pelo uso da tecnologia.
    Por fim, são elencados algumas vantagens do uso da tecnologia, como: aproximação de várias fontes de conhecimento e a aprendizagem de uma forma prazerosa. \\
  \item Vídeo 3 – A importância da tecnologia na educação (1 ponto) \\ 
    O vídeo começa com o autor levantando algumas questões, tais como: ``Como será a tecnologia na educação?'', ``Seremos substituídos por máquinas?'', ``As escolas estarão preparadas?''. Questões que provavelmente aparecem no meio acadêmico. Em seguida o autor afirma que os professores que não estiverem preparados para utilizar as tecnologias acabarão dependendo dos próprios alunos. Após isso, são demonstradas algumas vantagens em se utilizar as tecnologias na educação. Fazendo uso dessas tecnologias o educador amplia seus horizontes, oferece uma experiência de aprendizagem mais eficaz para seus alunos e depois responde a questão inicial, que os professores não irão perder seus empregos para os recursos tecnológicos, mas sim que estes recursos servem para auxiliar o trabalho do professor.
  \item Vídeo 4 – Tratamento da informação: a tecnologia no ensino da Matemática (1 ponto) \\ 
    Este vídeo pode ser separado em duas partes, ou dois ``atos''. No começo é mostrado um caso de sucesso, no qual uma professora de uma escola em São Paulo utiliza jogos eletrônicos para ensinar cálculos numéricos para os alunos. Depois é exibida uma entrevista com um especialista na área de tecnologia na educação, no qual ele dá uma pequena introdução sobre a história dos computadores pessoais e logo após enumera algumas dificuldades para utilização de tecnologias no ensino. Como a falta de preparo dos professores e como podem ser empregadas as tecnologias no ensino, que também é uma questão importante.
    Na sequência a equipe de reportagem demonstra mais casos onde a tecnologia é utilizada na educação com eficácia. Durante uma das entrevistas, uma educadora diz que existem profissionais da educação que não são a favor do uso de determinadas tecnologias, mas ao contrário do que foi exposto anteriormente, alguns educadores questionam se ao invés de auxiliar a aprendizagem, essas tecnologias ``mascaram'' a verdadeira aprendizagem.  Como ocorre com as calculadoras, no qual muitos professores questionam se elas auxiliam ou atrapalham o aprendizado dos alunos.
  \item Conclusão pessoal sobre a relação “Tecnologia e Educação” no processo ensino aprendizagem. (1 ponto) \\
    Ao meu ver as tecnologias na educação obviamente auxiliam no processo de aprendizagem. Principalmente nesta modalidade a distância, em que a tecnologia é indispensável. Se não utilizasse a tecnologia, eu não teria conseguido visualizar os documentos propostos para leitura e nem teria assistido os vídeos. Não teria feito essa atividade, em que utilizei meu computador e não conseguiria enviá-la, através da internet e do ambiente web. \\
    No entanto, deve-se tomar cuidado com o abuso das tecnologias. Quando é utilizado determinada tecnologia para automatizar uma tarefa, é imprescindível que a realização dessa tarefa seja um conhecimento consolidado. Por exemplo, é possível utilizar uma calculadora científica ou determinados softwares para construção de um gráfico a partir de uma função, mas deve-ser utilizar esses recursos para verificação de resultados ou se o aluno irá utilizado esse gráfico para determinado fim, evitando a tarefa onerosa de construção do gráfico.
  \end{enumerate}
  \item Faça a leitura atenta do texto I  - Ciência e  Tecnologia e   do texto II -  Educação e Tecnologias: o novo ritmo da informação,   e a partir deles organize um glossário[1] explicando o conceito de cada um do termos abaixo: (5 pontos) (0,25 cada) \\
    \begin{tabular}{|m{0.5cm}|m{5cm}|m{8cm}|}
      \hline
      1 & Ciberespaço & É um espaço virtual que é provido por meio de tecnologia  \\ \\
      \hline
      2 & Ciência & Ciência é o conhecimento obtido e testado através do método científico \\ \\
      \hline
      3 & Ciência Aplicada & É a aplicação de teorias às necessidades humanas \\ \\
      \hline
      4 & Ciência Natural & É o campo que usa o método científico para estudar a natureza \\ \\
      \hline
      5 & Ciência Pura & Está relacionado a utilização do método científico para elaboração de teorias \\ \\
      \hline
      6 & Ciência Social & É o campo que usa o método científico para estudo do comportamento humando e da sociedade \\ \\
      \hline
      7 & Conhecimento Científico & São os conhecimentos obtidos por meio do método científico \\ \\
      \hline
      8 & Ferramentas tecnológicas & São ferramentas produzidas por meio do conhecimento científico. \\ \\
      \hline
      9 & Informática & São o conjunto das ciências relacionadas ao armazenamento, transmissão e processamento de informações em meios digitais \\ \\
      \hline
      10 & Interatividade & É a forma de comunicação e relacionamento provido pelas tecnologias de comunicação. \\ \\
      \hline
      11 & Linguagem digital & Linguagem digital é a forma de linguagem escrita e oral por meio de recursos tecnológicos. Podem estar como hypertextos, mídias e em outros formatos  \\ \\
      \hline
      12 & Linguagem escrita & É a linguagem por meio impresso, tais como livros, revistas, jornais e outros materiais. \\ \\
      \hline
      13 & Linguagem oral & É a comunicação por meio da fala. \\ \\
      \hline
      14 & Método Científico & São um conjunto de regras de como deve ser o procedimento para produzir conhecimento científico.  \\ \\
      \hline
      15 & Mídia & São dispositivos tecnológicos utilizados para transmitir informação\\ \\
      \hline
      16 & NTICs & É o acrônimo para Novas Tecnologicas de Informação e Comunicação\\ \\
      \hline
      17 & Redes & É um conjunto de computadores interligados por um sistema de comunicação. \\ \\
      \hline
      18 & Técnica & São os procedimentos relacionados a uma ciência\\ \\
      \hline
      19 & Tecnologia & É um produto da ciência e da engenharia que envolve um conjunto de instrumentos, métodos e técnicas que visam a solução de problemas. \\ \\
      \hline
      20 & TICs & É o acrônimo para Tecnologias da Informação e Comunicação \\ \\
      \hline
    \end{tabular}
  \item Após a leitura  atenta do Texto III - Diretrizes para o uso de Tecnologias Educacionais faça o que se pede nos itens abaixo: (20 pontos)
    \begin{enumerate}
    \item Faça um texto descritivo  resumindo as diretrizes para o uso das tecnologias educacionais nas escolas públicas paranaenses. (5 pontos) \\
      De acordo com as diretrizes o uso de Tecnologias de Informação e Comunicação nas escolas públicas do Estaduais da Educação básica do Paraná estão separadas em quatro tecnologias. São elas: Mídia impressa na escola, Televisão Paulo Freire, Ambientes Virtuais na Web e Pesquisa escolar e Internet. Cada uma delas deve ser utilizadas por meio de um mediador. Seja a mediação do professor, do assistente técnico para as tecnologias da educação ou pela mediação do professor-tutor. \\
      A mediação é um conceito amplo, a mediação tratada nas diretrizes é a mediação didático-pedagógica. Essa mediação pedagógica deve se dar por meio de planejamento, estratégias e métodos do processo de ensino, a  partir da ação intencional do educador considerando o desenvolvimento do aluno e os princípios da educação. \\
      A mídia impressa, também chamada de mídia offline, é o recurso mais utilizado na educação e está em formato de livros, revistas, enciclopédias, obras literárias e didático-pedagógicas, histórias em quadrinhos, propagandas, encartes, mapas, entre outros. A mídia impressa possui uma vantagem em relação a fala. Ela fica escrita, registrada e pode ser tomada sempre que necessário. As escolas públicas do Paraná recebem diversos materiais impressos: livro didático (PNLD), livro didático público (LDP), diretrizes curriculares do estado (DCE), cadernos temáticos e pedagógicos, Grade da Programação da TV Paulo Freire, tutoriais e manuais diversos. \\
      A TV Paulo Freire teve sua primeira transmissão no dia 27 de junho de 2006. A programação está consonância com as políticas e ações da Seed-PR. O público alvo da TV Paulo Freire são os professores, alunos, funcionários e a comunidade em geral. A programação exibida de manhã repete-se à tarde e à noite. Determinados programas são exibidos em horários programados devido ao seu público alvo. Por exemplo, programas voltado aos professores são exibidos durante o intervalo do turno. O calendário escolar é levado em consideração assim como a representação de programas em dias e horários alternativos possibilita a ampliação do acesso pela comunidade escolar. Há uma grande mescla na grade de programação, tais como programas curtos e longos, críticos e mais leves. \\
      Os Ambientes Virtuais são um conjunto de ferramentas web que facilitam a aprendizagem por meio da internet. São utilizados para diversas finalidades, como formação continuada de docentes; atualização, pesquisa e informação por parte de alunos e educadores. O portal Dia-a-dia Educação possui consideráveis recursos como páginas disciplinares, páginas de TV Multimídia, Simuladores e Animações, Catálogo de Sítios, literatura online, filmes, mapas, bibliotecas online e outros materiais.
      A Pesquisa Escolar e a Internet devem ser realizadas de maneira adequada. É essencial a orientação do professor na pré-pesquisa pois como mediador, deve-se demonstrar ao aluno a função da pesquisa escolar como possibilitadora de aprendizagens e não apenas para realização de cópias. O objetivo é a construção de conhecimento.
      
    \item Acesse o portal dia-a-dia da educação  em  "Educadores"
      \url{http://www.educadores.diaadia.pr.gov.br/modules/conteudo/conteudo.php?conteudo=3}
      e liste os recursos didáticos direcionados para o ensino de matemática nele disponibilizados. (2 pontos) \\
      Existem diversos recursos didáticos direcionados para o ensino da matemática. Desde recursos para a preparação para o ENEM na disciplina de matemática - como provas anteriores, matriz de referência, cálculo de nota e etc - até jogos para serem usados na sala de aula, problemas matemáticos, hora atividade interativa, modelos de apresentação em slide, sugestões de aulas da Seed e do MEC entre outros materiais.
    \item Assista o Vídeo 7 – Um triângulo fractal especial, na Unidade 1 em Videoteca. (13 pontos) \\ \\
      \begin{itemize}
      \item Qual o tema tratado no vídeo? (1 ponto) \\ \\
        O triângulo de Sierpinski. \\
      \item O que é o triângulo de Sierpinski? (1 ponto) \\ \\
        É uma figura geométrica que é obtida por um processo interativo. No qual os triângulos são retirados infinitamente. \\
      \item A qual bloco de conteúdos dos PCN esse tema de matemática se insere? (1 ponto) \\
        Grandezas e Medidas.
      \item Quais conceitos o professor de matemática pode explorar na suas aulas? (2 pontos) \\
        O professor pode explorar o conceito algébrico ao generalizar a fórmula do triângulo de Sierpinski, o conceito geométrico e os conceitos de operações de divisão e potenciação. \\
      \item Qual é a função de interação para o cálculo do perímetro do triângulo? (1 ponto)
        $$3^{x+1}\cdot\left(\frac{l}{2}\right)^{x}$$ no qual x é a quantidade interações e l é o tamanho do lado do triângulo.
      \item Resolva a seguinte situação-problema: (7 pontos) \\ \\
        Uma praça de formato triangular tem todos os lados medindo 30 metros de comprimento, cada um. Imagine que você precisa criar canteiros também de forma triangular repetindo o padrão da praça. Monte uma tabela considerando 5 (cinco) iterações e defina o tamanho do lado e a medida do perímetro em cada caso. \\ \\
      \begin{tabular}{|m{3cm} | m{5cm} | m{4cm}|}
      \hline
      Interação & Comprimento de cada lado & Perímetro \\ \\
      \hline
      1 & 15 & 135 \\ \\
      \hline
      2 & 225 & 91125 \\ \\
      \hline
      3 & 3375 & 4100625 \\ \\
      \hline
      4 & 50625 & 184528125 \\ \\
      \hline
      5 & 759375 & 8303765625  \\ \\
      \hline
      \end{tabular}
      \end{itemize}      
      \end{enumerate}
\end{enumerate}
\end{document}
