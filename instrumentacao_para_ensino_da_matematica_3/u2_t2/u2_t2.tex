\documentclass[a4paper, 12pt]{article}

\usepackage[top=2cm, bottom=2cm, left=2.5cm, right=2.5cm]{geometry}
\usepackage[utf8]{inputenc}
\usepackage{array}
\usepackage{verbatim}
\usepackage{graphicx}
\usepackage{hyperref}

\graphicspath{{img/}}

\begin{document}
\begin{flushleft}\includegraphics{logo}\\
\textbf{UNIVERSIDADE ESTADUAL DE PONTA GROSSA} \\
SISTEMA UNIVERSIDADE ABERTA DO BRASIL - UAB \\
\underline{Licenciatura em Matemática | Polo UAB em Jacarezinho}\end{flushleft} \\
\textbf{ALUNO:} Ricardo Medeiros da Costa Junior   \textbf{RA:} 151774301 \\
\textbf{DISCIPLINA:} Instrumentação para o Ensino da Matemática III \\
\textbf{ATIVIDADE:} Tarefa atividade 2\end{flushleft} \\
\begin{enumerate}
\item Para compreender o conceito de aprendizagem significativa assista os vídeos:
  \begin{itemize}
  \item Professor Madruga - Aprendizagem Significativa
  \item Aprendizagem Significativa: do que estamos falando?
  \item Aprendizagem Significativa - a negociação dos sentidos
  \item A estrutura de uma aula significativa    
  \end{itemize}

  Em seguida responda as seguintes questões: 
  \begin{enumerate}
  \item O que diferencia aprendizagem significativa de outros tipos de aprendizagens? \\
    Segundo Ausubel a aprendizagem é significativa quando ela se relaciona com algo que aprendemos anteriormente. Em contrapartida, aprendizagem mecânica é a aprendizagem que nosso cérebro não consegue relacionar com nenhum conhecimento e por sua vez, tende a ser esquecida com maior facilidade.
    
  \item Como podemos definir sentido e significado? \\
    O sentido é subjetivo, está relacionado com as experiências e conhecimento de cada indivíduo, é definido por cada pessoa. O significado é uma construção social. É comum para todas as pessoas, é cientificamente embasado.
  \item Como transformar sentido em significado? \\
    Primeiramente, o diálogo para saber qual o sentido o aluno deu para determinado conteúdo, para posteriormente adequá-lo ao significado científico. O professor tem o papel de assessorar o aluno nessa trajetória. Para finalizar, quando o professor estiver montando seu plano de aula, o mesmo deve ficar atento em possíveis conteúdos que os alunos terão dificuldades e assim preparar atividades práticas para facilitar o processo de aprendizagem significativa por parte do aluno.
  \item O que é uma aula significativa e quais são as suas partes? \\
    Aula Significativa é uma aula estruturada de forma a facilitar o aluno a construir significado sobre o que ele está aprendendo. 1 - Construção de sentido; 2 - Apresentação do conteúdo; 3 - Verificação da aprendizagem.
  \item Como o professor pode ajudar o aluno a construir sentido e significado? \\
    Na construção de sentido, devemos contextualizar o conceito o mais próximo possível da realidade do aluno, o professor deve estar aberto ao diálogo, para saber qual o sentido que o aluno está formando sobre o conteúdo. A construção de significado é realizada na segunda parte da aula significativa. A construção deve ser feita a levar o aluno a construir o significado junto com o professor. O professor pode oferecer situações contextuais inclusivas para que o cérebro realize a assimilação do conceito.
  \end{enumerate}

\item Nas afirmativas abaixo complete os espaços em branco com as expressões: \\\\
  (AS) para \underline{Aprendizagem Significativa}\\
  (AM) para \underline{Aprendizagem Mecânica}\\\\
  (AS) A aprendizagem ocorre por descoberta.\\
  (AM) O novo conhecimento não considera os conhecimentos prévios do aluno\\
  (AM) O novo conhecimento não é integrado a conceitos existentes na estrutura significativa do aluno.\\
  (AM) Práticas, exercícios e réplicas reflexivas não contribuem para aprendizagem.\\
  (AM) A aprendizagem ocorre somente por recepção.
\end{enumerate}
\end{document}
