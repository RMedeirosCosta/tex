\documentclass[a4paper, 12pt]{article}
\usepackage[top=2cm, bottom=2cm, left=2.5cm, right=2.5cm]{geometry}
\usepackage[utf8]{inputenc}
\usepackage{hyperref}

\begin{document}
\textbf{UNIVERSIDADE TECNOLÓGICA FEDERAL DO PARANÁ}\\
\centerline{\underline{Biologia Computacional e Sistêmica | PPGBIOINFO}}\\\\
\textbf{ALUNO:} Ricardo Medeiros da Costa Junior   \textbf{RA:} a1598996 \\
\textbf{DISCIPLINA:} Programação para Bioinformática 1 \\
\textbf{ATIVIDADE:} Tarefa 1 \\
\begin{enumerate} 
\item Qual a relação de uma receita culinária com um algoritmo? \\ \\
Pode-se considerar uma receita culinária um algoritmo, apesar que alguns algoritmos podem ser mais complexos com iterações, recursividade e condições.
Os ingredientes são os parâmetros de entrada, com seus tipos (ovo, sal, manteiga) e os valores (100g, 2ml).
Os passos para realizar a receita são equivalentes as senteças de um algoritmo. As variáveis de um algoritmo podem ser considerado algumas substâncias usadas em algumas receitas. Para se fazer uma macarronada é necessário fazer o molho, no entanto não comemos o molho separadamente (pelo menos não é comum), usamos ele para obter a macarronada. O mesmo acontece com as variáveis do algoritmo, que são utilizadas para armazenar um estado ou valor que será utilizado posteriormente. Assim como na receita, se algum ingrediente estiver errado ou algum passo da receita não estiver correta o resultado não será esperado, o algoritmo falhará. \\
\item Definição de algoritmo por Donald Knuth \\\\
  Knuth propõe uma lista de cinco propriedades que são requerimentos para um algoritmo:
  \begin{description}
  \item[Finiteness:] Um algoritmo sempre deve terminar após um sequência de passos finitos.
  \item[Definiteness:] Cada passo de um algoritmo deve ser precisamente definido. Para ser cumpridas as ações, cada passo tem que rigorosamente especificado para cada caso.
   \item[Input:] Quantidades que são informadas antes do algoritmo começar. Essas entradas são obtidas de um conjunto de objetos específicos
  \item[Output:] Quantidades que tem uma relação específica com as entradas
  \item[Effectiveness:] Todas as operações executadas no algorimo devem ser suficientemente básicas que elas pode ser feitas corretamente em um tempo finito por um homem utilizando papel e caneta.
  \end{description}
  
\end{enumerate}
\end{document}


