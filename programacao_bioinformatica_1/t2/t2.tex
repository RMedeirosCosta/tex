\documentclass[a4paper, 12pt]{article}
\usepackage[top=2cm, bottom=2cm, left=2.5cm, right=2.5cm]{geometry}
\usepackage[utf8]{inputenc}
\usepackage{hyperref}

\begin{document}
\textbf{UNIVERSIDADE TECNOLÓGICA FEDERAL DO PARANÁ}\\
\centerline{\underline{Biologia Computacional e Sistêmica | PPGBIOINFO}}\\\\
\textbf{ALUNO:} Ricardo Medeiros da Costa Junior   \textbf{RA:} a1598996 \\
\textbf{DISCIPLINA:} Programação para Bioinformática 1 \\
\textbf{ATIVIDADE:} Tarefa 2 \\
\begin{enumerate} 
\item Procure e estude a estrutura condicional SE \\\\
  A estrutura condicional possibilita a execução de determinadas sentenças se uma condição for ou não satisfeita. A estrutura condicional pode ser \emph{simples} ou \emph{composta}. Se for simples, executa um comando ou bloco de comandos caso uma condição seja verdadeira, caso não seja, a estrutura é finalizada.\\
  A estrutura condicional segue o mesmo princípio, no entanto, pode ser executado outro comando ou bloco de comandos se a condição não for satisfeita. 
\item Traga dois exemplos de algoritmo
  \begin{itemize}
    \item Primeiro \\\\
    Verifica par \\
Data 11/03/2016 \\
Objetivo: Verifica se um número inteiro é par \\
OBS:\\
Entrada: numero (inteiro)\\
Saida: verdadeiro ou falso\\

Leia(numero)\\

SE numero MOD 2 == 0 ENTAO\\
ESCREVA(``É par'')\\
SENAO\\
ESCREVA(``NAO É PAR'')\\

\item Segundo\\\\
      Verifica positivo \\
Data 11/03/2016 \\
Objetivo: Verifica se um número inteiro é positivo \\
OBS:\\
Entrada: numero (inteiro)\\
Saida: verdadeiro ou falso\\

Leia(numero)\\

SE numero $>$ 0 ENTAO\\
ESCREVA(``É positivo'')\\
SE numero $<$ 0 ENTAO\\
ESCREVA(``É negativo'')\\
SENAO\\
ESCREVA(``O número é 0'')\\

  
  \end{itemize}
\item Ler o capítulo 7 - O estilo na Programação do livro Ciência dos Computadores - Uma abordagem algorítmica
\end{enumerate}
\end{document}
