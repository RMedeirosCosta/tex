\documentclass[a4paper, 12pt]{article}
\usepackage[top=2cm, bottom=2cm, left=2.5cm, right=2.5cm]{geometry}
\usepackage[utf8]{inputenc}
\usepackage{array}
\usepackage{graphicx}
\usepackage{amsmath}
\usepackage{cancel}

\graphicspath{{img/}}

\begin{document}
\begin{flushleft}\includegraphics{logo}\\
\textbf{UNIVERSIDADE ESTADUAL DE PONTA GROSSA} \\
SISTEMA UNIVERSIDADE ABERTA DO BRASIL - UAB \\
\underline{Licenciatura em Matemática | Polo UAB em Jacarezinho}\end{flushleft} 
\textbf{ALUNO:} Ricardo Medeiros da Costa Junior   \textbf{RA:} 151774301 \\
\textbf{DISCIPLINA:} Matemática Financeira \\
\textbf{ATIVIDADE:} Atividade 4 - Tarefa: Juro Composto (Valor: 5,0) \\ 
\textbf{TUTOR(A):} Edilane Cristine Budasz \\
\textbf{PERÍODO:} Quarto \\

\textbf{Resolva os problemas 3 ao 7 da página 52 do livro didático de forma detalhada utilizando as fórmulas e com os passos da HP-12C.}

\begin{itemize}

\item Uma empresa pretende comprar um equipamento de R\$ 100.000,00 daqui
a 4 anos com o montante de uma aplicação financeira. Quanto essa empresa
deve aplicar hoje em uma financeira que paga juros de 2 \% a.m?
$$ FV = R\$100.000,00 $$
$$ n = 4 \cdot 12 \Rightarrow n = 48 $$
$$ i = 2 a.m. \Rightarrow i = 0,02 $$
$$ PV = ? $$
\\
$$ FV = PV(1+i)^n $$
$$ 100000 = PV(1+0,02)^{48} \Rightarrow $$
$$ 100000 = PV(1,02)^{48}  \Rightarrow $$
$$ 100000 = PV2,587  \Rightarrow $$
$$ PV = \frac{100000}{2,587}  \Rightarrow $$
$$ \boxed{PV = R\$38.653,760} $$
\\
Pela calculadora HP-12C\\\\
\emph{f REG}\\
\emph{100000 CHS FV}\\
\emph{2 i}\\
\emph{4 ENTER 12 $ \times$ n}\\
\emph{PV R\$ 38.653,76}

\item Um  capital  de  R\$  51.879,31  aplicado  por  6  meses  resultou  em  R\$ 120.000,00. Qual a taxa mensal da transação?
  
$$ PV = R\$51.879,31 $$
$$ n = 6 $$
$$ i = ? $$
$$ FV = R\$120.000,00 $$
\\
$$ FV = PV(1+i)^6 $$
$$ 120000 = 51879,31(1+i)^6 \Rightarrow $$
$$ \frac{120000}{51879,31} = (1+i)^6 \Rightarrow $$
$$ \sqrt[6]{2,313} = \sqrt[6]{(1+i)^6} \Rightarrow $$
$$ 1,15 = 1+i \Rightarrow $$
$$ 0,15 = i \Rightarrow $$
$$ \boxed{i = 15\% a.m.} $$
\\

Pela calculadora HP-12C\\\\
\emph{f REG}\\
\emph{51879,31 CHS PV}\\
\emph{6 n}\\
\emph{120000 FV}\\
\emph{i 15\%}
  
\item Em quantos meses triplica um capital que cresce à taxa de 3\% a.m?

$$ FV = 3PV $$
$$ n = ? $$
$$ i = 3 \Rightarrow i = 0,03 $$
$$ PV = PV $$
\\
$$ 2^{x} \geq 14 \Rightarrow $$
$$ \log(2^{x}) \geq \log(14) \Rightarrow $$
$$ x\log(2) \geq \log(14) \Rightarrow $$
$$ x \geq \frac{\log(14)}{\log(2)} \Rightarrow $$
$$ \boxed{x \geq 3,8073...} $$

\\
Como os nucleotídeos pertecem ao conjunto dos naturais (não existe meio nucleotídeo, por exemplo), cada códon deve possuir no mínimo 4 nt para satisfazer a condição. $S=\{x \in \mathbb{N} | x \geq 4 \}$ \\\\

Pela calculadora HP-12C\\\\
\emph{f REG}\\
\emph{1 CHS PV}\\
\emph{3 FV}\\
\emph{3 i}\\
\emph{n 38}  

\item  A rentabilidade efetiva de um investimento é de 10\% a.a. Se os juros 
ganhos forem de R\$ 27.473,00, sobre um capital investido de R\$ 83.000,00, 
quanto tempo o capital ficará aplicado?

$$ FV = R\$83.000,00 +  R\$27.473,00 \Rightarrow FV = R\$110.473,00  $$
$$ n = ? $$
$$ i = 10 \Rightarrow i = 0,10 $$
$$ J = R\$27.473,00 $$
$$ PV = R\$83.000,00 $$
\\
$$ FV = PV(1+i)^n $$
$$ 110473 = 83000(1+0,1)^{n} \Rightarrow $$
$$ \frac{110473}{83000} = 1,1^N \Rightarrow $$
$$ 1,331 = 1,1^{n} \Rightarrow $$
$$ \log(1,331) = \log(1,1)^{n} \Rightarrow $$
$$ \log(1,331) = n \log(1,1) \Rightarrow $$
$$ n  = \frac{\log(1,331)}{\log(1,1)} \Rightarrow $$
$$ n  = \frac{0,124}{4,139} \Rightarrow $$
$$ \boxed{n  \approx 3} $$
\\

3 anos. \\

Pela calculadora HP-12C\\\\
\emph{f REG}\\
\emph{83000 CHS PV}\\
\emph{CHS ENTER 27473 $+$ FV}\\
\emph{10 i}\\
\emph{n 3}  

\item Em quanto tempo os juros gerados por um capital igualam-se ao próprio 
capital, aplicando-se uma taxa efetiva de 3\% a.m?
  
$$ PV = PV  $$
$$ n = ? $$
$$ i = 3 \% \Rightarrow i = 0,03 $$
$$ FV = PV $$
\\
$$ FV = PV(1+i)^n $$
$$ PV = PV(1+0,03)^n - PV \Rightarrow $$
$$ 2 \bcancel{PV} = \bcancel{PV}(1,03)^{n} \Rightarrow $$
$$ 2 = 1,03^N \Rightarrow $$
$$ \ln(2) = \ln(1,03^N) \Rightarrow $$
$$ \ln(2) = n \ln(1,03) \Rightarrow $$
$$ n = \frac{\ln(2)}{\ln(1,03)} \Rightarrow $$
$$ \boxed{n = 23,449} $$

\\

Como a taxa mensal é a quantidade de meses, então o resultado tem que ser um número natural, portanto será arredondado para 24 meses. \\

Pela calculadora HP-12C\\\\
\emph{f REG}\\
\emph{3 i}\\
\emph{1 CHS PV CHS ENTER 2 $ \times $ }\\
\emph{n 24}

\end{itemize}

\end{document}

