\documentclass[a4paper, 12pt]{article}

\usepackage[top=2cm, bottom=2cm, left=2.5cm, right=2.5cm]{geometry}
\usepackage[utf8]{inputenc}
\usepackage{array}
\usepackage{graphicx}
\usepackage{amsmath}
\usepackage{cancel}

\graphicspath{{img/}}

\begin{document}
\begin{flushleft}\includegraphics{logo}\\
\textbf{UNIVERSIDADE ESTADUAL DE PONTA GROSSA} \\
SISTEMA UNIVERSIDADE ABERTA DO BRASIL - UAB \\
\underline{Licenciatura em Matemática | Polo UAB em Jacarezinho}\end{flushleft} 
\textbf{ALUNO:} Ricardo Medeiros da Costa Junior   \textbf{RA:} 151774301 \\
\textbf{DISCIPLINA:} 101531 - Matemática Financeira \\
\textbf{ATIVIDADE:} Atividade 10 - Tarefa: SAC(Valor: 5,0) \\ 
\textbf{TUTOR(A):} Edilane Cristine Budasz \\
\textbf{PERÍODO:} Quarto \\\\
Prezado(a) Acadêmico(a)\\

Nesta tarefa você deverá elaborar um problema de financiamento ou empréstimo com 10 parcelas.\\

O sistema de amortizaçao deve ser o sistema SAC.\\

Você deverá calcular todos os dados da tabela e explicar detalhadamente os cálculos realizados.\\\\\\

Um empréstimo de R\$ 1.000.000,00 deverá ser pago, dentro de um prazo de 10 meses, em 10 prestações mensais, à taxa de 5\% ao mês. Construa uma planilha deste empréstimo.\\\\
\textbf{Resolução}
O valor do financiamento é de R\$1.000.000,00, à taxa de 5\% a.m., para ser pago em 10 parcelas. Para elaborar a planilha de pagamento, adotaremos os seguines cálculos:\\
\begin{itemize}
\item calcular a parcela de amortização:\\
  $$ A = \frac{100000\bcancel{0}}{1\bcancel{0}}=100000$$
\item calcular a parcela de juros: a taxa deverá ser aplicada sobre o saldo devedor no período anterior.\\
  $$ J = 1000000 \cdot 0,05 = 50000$$
\item o valor da primeira prestação é obtido pela soma da amortização e o juro.\\
  $$ PMT = 100000 + 50000 = 150000 $$
\item o saldo devedor do período: resultado da subtração do valor da amortização pelo saldo devedor do período anterior.\\
  $$ 1000000 - 100000 = 900000 $$\\\\
  \begin{tabular}{ | c | c | c | c | c |}
  \hline
  Período & Saldo Devedor (R\$) & Amorização (R\$) & Juros (R\$) & Prestação (R\$) \\ \hline
  0 & 1.000.000 & - & - & - \\ \hline
  1 &   900.000 & 100.000 & 50.000 & 150.000 \\ \hline
  \end{tabular} \\\\
  Para a terceira coluna da tabela devemos fazer:\\
  \begin{enumerate}
  \item Juros, 5\% sobre 900000:\\
    $$ 900.000 \cdot 0,05 = 45000 $$
  \item Amorização 100000
  \item Prestação: $ 100000 + 45000 = 145000 $
  \item Saldo devedor: $ 900000 - 100000 = 800000 $    
  \end{enumerate}
  Ao repetir esta sequência de cálculos em cada período obteremos a seguinte tabela:\\
  \begin{tabular}{ | c | c | c | c | c |}
  \hline
  Período & Saldo Devedor (R\$) & Amorização (R\$) & Juros (R\$) & Prestação (R\$) \\ \hline
  0 & 1.000.000 & - & - & - \\ \hline
  1 &   900.000 & 100.000 & 50.000 & 150.000 \\ \hline
  2 &   800.000 & 100.000 & 45.000 & 145.000 \\ \hline
  3 &   700.000 & 100.000 & 40.000 & 140.000 \\ \hline
  4 &   600.000 & 100.000 & 35.000 & 135.000 \\ \hline
  5 &   500.000 & 100.000 & 30.000 & 130.000 \\ \hline
  6 &   400.000 & 100.000 & 25.000 & 125.000 \\ \hline
  7 &   300.000 & 100.000 & 20.000 & 120.000 \\ \hline
  8 &   200.000 & 100.000 & 15.000 & 115.000 \\ \hline
  9 &   100.000 & 100.000 & 10.000 & 110.000 \\ \hline
  10 &   0 & 100.000 & 5.000 & 105.000 \\ \hline  
  \end{tabular} \\\\
  

  
\end{itemize}

\end{document}
