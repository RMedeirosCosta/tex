\documentclass[a4paper, 12pt]{article}
\usepackage[top=2cm, bottom=2cm, left=2.5cm, right=2.5cm]{geometry}
\usepackage[utf8]{inputenc}
\usepackage{array}
\usepackage{graphicx}
\usepackage{amsmath}
\usepackage{cancel}
\usepackage[normalem]{ulem}

\graphicspath{{img/}}

\begin{document}
\begin{flushleft}\includegraphics{logo}\\
\textbf{UNIVERSIDADE ESTADUAL DE PONTA GROSSA} \\
SISTEMA UNIVERSIDADE ABERTA DO BRASIL - UAB \\
\underline{Licenciatura em Matemática | Polo UAB em Jacarezinho}\end{flushleft} 
\textbf{ALUNO:} Ricardo Medeiros da Costa Junior   \textbf{RA:} 151774301 \\
\textbf{DISCIPLINA:} Matemática Financeira \\
\textbf{ATIVIDADE:} Atividade 8 - Tarefa: Rendas imediatas e diferidas (Valor 5,0) \\
\textbf{TUTOR(A):} Edilane Cristine Budasz \\
\textbf{PERÍODO:} Quarto \\

\textbf{Resolva os problemas a seguir de forma detalhada utilizando as fórmulas e com os passos da HP-12C.}
\\
\textbf{Digite em um editor de texto.} \\
\textbf{Não esqueça do cabeçalho padrão!} \\
\textbf{Depois anexe o arquivo nesta atividade.} \\
\textbf{Estes exercícios constam no livro didático.} \\
\textbf{Pag 85 - exercício 3, 5 e 7}\\
\textbf{Pag 91 - exercício 3 e 5}\\

\begin{itemize}

\item Determine  o  valor  futuro  de  12  depósitos  de  R\$  350,00,  realizados  no início de cada mês, sendo que a taxa de rendimento é de 1,4\% ao mês
$$ PMT = R\$350 $$
$$ FV = ? $$
$$ i = 1,4\% \Rightarrow i = 0,014 $$
$$ n = 12 $$
\\
$$ FV = PMT\frac{(1+i)^{n}-1}{i}\cdot(1+i) \Rightarrow $$
$$ FV = 350\frac{(1+0,014)^{12}-1}{0,014}\cdot(1+0,014) \Rightarrow $$
$$ FV = 350\frac{(1,014)^{12}-1}{0,014}\cdot(1,014) \Rightarrow $$
$$ FV = 350\frac{0,1815}{0,014}\cdot(1,014) \Rightarrow $$
$$ FV = 350\frac{0,1815}{0,014}\cdot(1,014) \Rightarrow $$
$$ \boxed{FV = R\$4602,52} $$

\\

Pela calculadora HP-12C\\\\
\emph{f REG}\\
\emph{g BEG}\\
\emph{350 CHS PMT}\\
\emph{1,4 i}\\
\emph{12 n}\\
\emph{FV 4602,52}

\item Quanto se deve depositar no início de cada mês, numa instituição financeira que paga 1,5\% ao mês, para constituir o montante de R\$ 30.000,00 no fim de 14 meses?\\
  
$$ FV = R\$30.000,00 $$
$$ PMT = ? $$  
$$ i = 1,5\% \Rightarrow i = 0,015 $$
$$ n = 14 $$

\\
$$ PMT = \frac{FV\cdot i}{(1+i)((1+i)^{n}-1)} \Rightarrow $$
$$ PMT = \frac{30000\cdot 0,015}{(1+0,015)((1+0,015)^{14}-1)} \Rightarrow $$
$$ PMT = \frac{450}{(1,015)((1,015)^{14}-1)} \Rightarrow $$
$$ PMT = \frac{450}{0,2352} \Rightarrow $$
$$ \boxed{PMT = R\$1913,00} $$
\\
Pela calculadora HP-12C\\\\
\emph{f REG}\\
\emph{g BEG}\\
\emph{30000 CHS FV}\\
\emph{14 n}\\
\emph{1,5 i}\\
\emph{PMT 1913,00}  

\item Determine a taxa que se aplica em uma instituição financeira, no início de cada trimestre, sabendo que a importância de R\$ 10.524,68 no final de 1 ano, resulta em um montante de R\$ 50.000,00? \\
  
$$ FV = R\$50.000,00 $$
$$ PMT = R\$10.524,68 $$  
$$ i = ? $$
$$ n = 1\%aa \Rightarrow n = 4 $$

\\
$$ FV = PMT\frac{(1+i)^{n}-1}{i}\cdot(1+i) \Rightarrow $$
$$ 50000 = 10524,68\frac{(1+i)^{4}-1}{i}\cdot(1+i) $$
\\

  \begin{tabular}{ | c | c | c |}
  \hline
    7,2 & 50.242,01 & \\ \hline
    i & 50.000,00 & 242,01 \\ \hline
    6,6 & 49.518,79 & 723,22 \\ \hline  
  \end{tabular} \\\\

  \begin{tabular}{ c c }
    0,6 & 723,22 \\ 
    i & 242,01 \\ 
  \end{tabular} 
  
$$ 145,206 = 723,22i \Rightarrow $$
$$ i = \frac{145,206}{723,22} $$
$$ \boxed{i = 0,20077708...}


\emph{f REG}\\
\emph{g BEG}\\
\emph{50000 CHS FV}\\
\emph{10524,68 PMT}\\
\emph{4 n} \\
\emph{i 7,00} \\
\textbf{Acredito que a calculadora tenha arredondado, pois no livro aparece 6,99\%}

\item Que dívida pode ser amortizada com 8 prestações mensais de R\$ 1.000,00, sendo de 7\% a.m. a taxa de juro e devendo ser paga a primeira prestação 3 meses depois de realizado o empréstimo?

$$ PV = ? $$
$$ PMT = R\$1.000,00 $$  
$$ i = 7\% \Rightarrow i = 0,07 $$
$$ n = 8 $$
$$ m = 2 $$  \\
  
$$ PV = PMT\frac{1-(1+i)^{-n}}{i}(1+i)^{-m} $$
$$ PV = 1000\frac{1-(1+0,07)^{-8}}{0,07}(1+0,07)^{-2} $$
$$ PV = 1000\frac{1-(1,07)^{-8}}{0,07}(1,07)^{-2} $$
$$ PV = 1000 \cdot 5,9712 \cdot 0,8734 $$
$$ PV = 1000 \cdot 5,215563 $$
$$ \boxed{PV = 5.215,56} $$  

%Pela calculadora HP-12C\\\\
%\emph{f REG}\\
%\emph{g END}\\
%\emph{15000 CHS PV}\\
%\emph{36 n}\\
%\emph{400 PMT}\\
%\emph{i 0,22}  

\item Qual o valor da prestação mensal referente a um financiamento de um imóvel de R\$ 120.000,00, a ser liquidado em 12 meses, à taxa de 3\% ao mês, sendo que a primeira prestação vence a 90 dias da data do contrato?

$$ PV = R\$120.000,00 $$
$$ PMT = ? $$  
$$ i = 3\% \Rightarrow i = 0,03 $$
$$ n = 12 $$
$$ m = 3 $$  

$$ PV = PMT\frac{1-(1+i)^{-n}}{i}(1+i)^{-m} $$
$$ 120000 = PMT\frac{1-(1+0,03)^{-12}}{0,03}(1+0,03)^{-3} $$
$$ 120000 = PMT\frac{1-(1,03)^{-12}}{0,03}(1,03)^{-3} $$
$$ 120000 = PMT \cdot 9,954 \cdot 0,915 $$
$$ 120000 = PMT \cdot 9,382603 $$
$$ PMT = \frac{120000}{9,382603} $$
$$ \boxed{PMT = 12789,627} $$        


%Pela calculadora HP-12C\\\\
%\emph{f REG}\\
%\emph{g END}\\
%\emph{5000 ENTER}\\
%\emph{5000 ENTER}\\
%\emph{10 \times 100 - 4500 CHS PV} \\
%\emph{10 n}\\
%\emph{3 i}\\
%\emph{PMT 527,54}\\

\end{itemize}

\end{document}

