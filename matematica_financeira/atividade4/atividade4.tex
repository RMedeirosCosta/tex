\documentclass[a4paper, 12pt]{article}
\usepackage[top=2cm, bottom=2cm, left=2.5cm, right=2.5cm]{geometry}
\usepackage[utf8]{inputenc}
\usepackage{array}
\usepackage{graphicx}
\usepackage{amsmath}
\usepackage{cancel}

\graphicspath{{img/}}

\begin{document}
\begin{flushleft}\includegraphics{logo}\\
\textbf{UNIVERSIDADE ESTADUAL DE PONTA GROSSA} \\
SISTEMA UNIVERSIDADE ABERTA DO BRASIL - UAB \\
\underline{Licenciatura em Matemática | Polo UAB em Jacarezinho}\end{flushleft} 
\textbf{ALUNO:} Ricardo Medeiros da Costa Junior   \textbf{RA:} 151774301 \\
\textbf{DISCIPLINA:} Matemática Financeira \\
\textbf{ATIVIDADE:} Atividade 4 - Tarefa: Juro Composto (Valor: 5,0) \\ 
\textbf{TUTOR(A):} Edilane Cristine Budasz \\
\textbf{PERÍODO:} Quarto \\

\textbf{Resolva os problemas 3 ao 7 da página 52 do livro didático de forma detalhada utilizando as fórmulas e com os passos da HP-12C.}

\begin{itemize}

\item Uma empresa pretende comprar um equipamento de R\$ 100.000,00 daqui
a 4 anos com o montante de uma aplicação financeira. Quanto essa empresa
deve aplicar hoje em uma financeira que paga juros de 2 \% a.m?

\item Um  capital  de  R\$  51.879,31  aplicado  por  6  meses  resultou  em  R\$ 120.000,00. Qual a taxa mensal da transação?

\item Em quantos meses triplica um capital que cresce à taxa de 3\% a.m?

\item  A rentabilidade efetiva de um investimento é de 10\% a.a. Se os juros 
ganhos forem de R\$ 27.473,00, sobre um capital investido de R\$ 83.000,00, 
quanto tempo o capital ficará aplicado?

\item Em quanto tempo os juros gerados por um capital igualam-se ao próprio 
capital, aplicando-se uma taxa efetiva de 3\% a.m?
  
\end{itemize}

\end{document}

