\documentclass[a4paper, 12pt]{article}
\usepackage[top=2cm, bottom=2cm, left=2.5cm, right=2.5cm]{geometry}
\usepackage[utf8]{inputenc}
\usepackage{array}
\usepackage{graphicx}
\usepackage{amsmath}
\usepackage{cancel}

\graphicspath{{img/}}

\begin{document}
\begin{flushleft}\includegraphics{logo}\\
\textbf{UNIVERSIDADE ESTADUAL DE PONTA GROSSA} \\
SISTEMA UNIVERSIDADE ABERTA DO BRASIL - UAB \\
\underline{Licenciatura em Matemática | Polo UAB em Jacarezinho}\end{flushleft} 
\textbf{ALUNO:} Ricardo Medeiros da Costa Junior   \textbf{RA:} 151774301 \\
\textbf{DISCIPLINA:} Matemática Financeira \\
\textbf{ATIVIDADE:} Atividade 5 - Tarefa: Taxas e descontos (Valor 5,0) \\
\textbf{TUTOR(A):} Edilane Cristine Budasz \\
\textbf{PERÍODO:} Quarto \\

\textbf{Agora que já estudou a seção 2 e 3 da unidade 2, resolva os problemas do livro didático das páginas 62 e 63 e anexe aqui as resoluções dos seguintes exercícios:  \\\\
  Problemas 4 e 7 da página 63. \\\\
  Problemas 3 da página 67.}

\begin{itemize}

\item Calcular a taxa de desconto racional composto de um título de R\$ 20.000,00, descontado  4  meses  antes  do  vencimento,  recebendo  líquido  o  valor  de  R\$16.290,13.

\item Um  título  de  R\$  20.000,00  foi  descontado  num  banco,  pelo  desconto comercial composto, à taxa de 5\%a.m., sendo creditada, na conta do cliente, a importância  de  R\$  16.290,13.  Quanto  tempo  antes  do  vencimento  foi descontado esse título?

\item Se o valor nominal de um título é de R\$ 28.800,00 e será descontado antecipadamente no prazo de 4 meses com uma taxa de 2,5\% ao mês, calcule o valor do desconto composto comercial.

\item Resgatei um título por R\$ 1.645,41, com 120 dias antes do vencimento. Qual o valor nominal do título, sendo a taxa de desconto racional de 60\% a.a. com capitalização mensal?

\item Paulo tem um compromisso representado por duas promissórias: uma de R\$ 200.000,00 e outra de R\$ 250.000,00, vencíveis em quatro e seis meses, respectivamente. Prevendo que não terá esses valores nas datas estipuladas, solicita ao banco credor a substituição dos dois títulos por um único a vencer em  10  meses.  Sabendo-se  que  o  banco  adota  em  juros  compostos  de  5\% a.m.,  com  desconto  racional,  calcule  o  valor  da  nota  promissória.
  
\end{itemize}

\end{document}

