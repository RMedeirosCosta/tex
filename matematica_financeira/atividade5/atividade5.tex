\documentclass[a4paper, 12pt]{article}
\usepackage[top=2cm, bottom=2cm, left=2.5cm, right=2.5cm]{geometry}
\usepackage[utf8]{inputenc}
\usepackage{array}
\usepackage{graphicx}
\usepackage{amsmath}
\usepackage{cancel}

\graphicspath{{img/}}

\begin{document}
\begin{flushleft}\includegraphics{logo}\\
\textbf{UNIVERSIDADE ESTADUAL DE PONTA GROSSA} \\
SISTEMA UNIVERSIDADE ABERTA DO BRASIL - UAB \\
\underline{Licenciatura em Matemática | Polo UAB em Jacarezinho}\end{flushleft} 
\textbf{ALUNO:} Ricardo Medeiros da Costa Junior   \textbf{RA:} 151774301 \\
\textbf{DISCIPLINA:} Matemática Financeira \\
\textbf{ATIVIDADE:} Atividade 5 - Tarefa: Taxas e descontos (Valor 5,0) \\
\textbf{TUTOR(A):} Edilane Cristine Budasz \\
\textbf{PERÍODO:} Quarto \\

\textbf{Agora que já estudou a seção 2 e 3 da unidade 2, resolva os problemas do livro didático das páginas 62 e 63 e anexe aqui as resoluções dos seguintes exercícios:  \\\\
  Problemas 4 e 7 da página 63. \\\\
  Problemas 3 da página 67.}

\begin{itemize}

\item Calcular a taxa de desconto racional composto de um título de R\$ 20.000,00, descontado  4  meses  antes  do  vencimento,  recebendo  líquido  o  valor  de  R\$16.290,13.
  
$$ A = R\$16.290,00 $$
$$ N = R\$20.000,00 $$
$$ i = ? $$
$$ n = 4 $$
\\
$$ N = A(1+i)^n $$
$$ 20000 = 16290(1+i)^{4} \Rightarrow $$
$$ \frac{20000}{16290} = (1+i)^{4}  \Rightarrow $$
$$ 1,22 = (1+i)^4  \Rightarrow $$
$$ \sqrt[4]{1,22} = \sqrt[4]{(1+i)^4}  \Rightarrow $$
$$ 1,052 = 1+i  \Rightarrow $$
$$ 0,052 = i  \Rightarrow $$
$$ \boxed{i = 5,26 \% } $$
\\
Pela calculadora HP-12C\\\\
\emph{f REG}\\
\emph{16290 CHS PV}\\
\emph{20000 FV}\\
\emph{4 n}\\
\emph{i 5,26}

\item Um  título  de  R\$  20.000,00  foi  descontado  num  banco,  pelo  desconto comercial composto, à taxa de 5\%a.m., sendo creditada, na conta do cliente, a importância  de  R\$  16.290,13.  Quanto  tempo  antes  do  vencimento  foi descontado esse título?
  
$$ A = R\$16.290,00 $$
$$ N = R\$20.000,00 $$
$$ i = 0,05 $$
$$ n = ? $$
\\
$$ N = A(1+i)^n $$
$$ 20000 = 16290(1+0,05)^{n} \Rightarrow $$
$$ \frac{20000}{16290} = 1,05^{n}  \Rightarrow $$
$$ 1,22 = 1,05^4  \Rightarrow $$
$$ \ln(1,22) = \ln(1,05^n)  \Rightarrow $$
$$ \ln(1,22) = n\ln(1,05) \Rightarrow $$
$$ n = \frac{\ln(1,22)}{\ln(1,05)} \Rightarrow $$
$$ \boxed{n \approx 4 } $$
\\
Pela calculadora HP-12C\\\\
\emph{f REG}\\
\emph{16290 CHS PV}\\
\emph{20000 FV}\\
\emph{5 CHS i}\\
\emph{n 4 CHS}  

\item Se o valor nominal de um título é de R\$ 28.800,00 e será descontado antecipadamente no prazo de 4 meses com uma taxa de 2,5\% ao mês, calcule o valor do desconto composto comercial.

$$ N = R\$28.800,00 $$
$$ i = 0,025 $$
$$ n = 4 $$
$$ d_c = ? $$  
\\
$$ d_c = N(1-(1-i)^n) $$
$$ d_c = 28800(1-(1-0,025)^4) \Rightarrow $$
$$ d_c = 28800(1-0,9036) \Rightarrow $$
$$ d_c = 28800 \cdot 0,0963 \Rightarrow $$
$$ \boxed{d_c \approx 2773,79} $$
\\
Pela calculadora HP-12C\\\\
\emph{f REG}\\
\emph{28800 CHS PV}\\
\emph{4 n}\\
\emph{2,5 CHS i}\\
\emph{FV 26026,21}\\
\emph{RCL PV + CHS 2773,79}

\item Resgatei um título por R\$ 1.645,41, com 120 dias antes do vencimento. Qual o valor nominal do título, sendo a taxa de desconto racional de 60\% a.a. com capitalização mensal?
  
$$ A = R\$1.645,41 $$
$$ i = \frac{60}{12} \Rightarrow i = 0,05 $$
$$ n = \frac{120}{30} \Rightarrow n = 4 $$
$$ N = ? $$  
\\
$$ A = \frac{N}{(1+i)^N} $$
$$ 1645,41 = \frac{N}{(1+0,05)^4} $$
$$ 1645,41 = \frac{N}{(1,05)^4} $$
$$ 1645,41 = \frac{N}{1,2155} $$
$$ \boxed{N = R\$2.000,00} $$
\\
Pela calculadora HP-12C\\\\
\emph{f REG}\\
\emph{1645,41 CHS PV}\\
\emph{120 ENTER 30 $ \div $ n}\\
\emph{60 ENTER 12 $ \div $ i}\\
\emph{FV 2000}  

\item Paulo tem um compromisso representado por duas promissórias: uma de R\$ 200.000,00 e outra de R\$ 250.000,00, vencíveis em quatro e seis meses, respectivamente. Prevendo que não terá esses valores nas datas estipuladas, solicita ao banco credor a substituição dos dois títulos por um único a vencer em  10  meses.  Sabendo-se  que  o  banco  adota  em  juros  compostos  de  5\% a.m.,  com  desconto  racional,  calcule  o  valor  da  nota  promissória.

$$ N = R\$200.000,00 $$
$$ N' = R\$250.000,00 $$
$$ n  = 4 $$
$$ n' = 6 $$
$$ n'' = 10 $$  
$$ i = 0,05 $$
$$ N'' = ? $$  
\\
$$ \frac{N''}{(1+i)^{n''}} = \frac{N}{(1+i)^{n}} + \frac{N'}{(1+i)^{n'}} \Rightarrow $$
$$ \frac{N''}{(1+0,05)^{10}} = \frac{200000}{(1+0,05)^{4}} + \frac{250000}{(1+0,05)^{6}} \Rightarrow $$
$$ \frac{N''}{(1,05)^{10}} = \frac{200000}{(1,05)^{4}} + \frac{250000}{(1,05)^{6}} \Rightarrow $$
$$ \frac{N''}{1,628} = 164540,494 + 186553,84 \Rightarrow $$
$$ \boxed{N'' = R\$ 571.895,69} $$
\\
Pela calculadora HP-12C\\\\
\emph{f REG}\\
\emph{200000 ENTER}\\
\emph{1,05 ENTER 4 $ Y^x  \div $}\\
\emph{250000 ENTER}\\
\emph{1,05 ENTER 6 $ Y^x  \div $}\\
\emph{+}\\
\emph{1,05 ENTER 10 $Y^x  \div $}\\
\emph{$ \times $ 571895,69}\\

\end{itemize}

\end{document}

