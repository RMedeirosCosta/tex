\documentclass[a4paper, 12pt]{article}
\usepackage[top=2cm, bottom=2cm, left=2.5cm, right=2.5cm]{geometry}
\usepackage[utf8]{inputenc}
\usepackage{array}
\usepackage{graphicx}
\usepackage{amsmath}
\usepackage{cancel}

\graphicspath{{img/}}

\begin{document}
\begin{flushleft}\includegraphics{logo}\\
\textbf{UNIVERSIDADE ESTADUAL DE PONTA GROSSA} \\
SISTEMA UNIVERSIDADE ABERTA DO BRASIL - UAB \\
\underline{Licenciatura em Matemática | Polo UAB em Jacarezinho}\end{flushleft} 
\textbf{ALUNO:} Ricardo Medeiros da Costa Junior   \textbf{RA:} 151774301 \\
\textbf{DISCIPLINA:} Matemática Financeira \\
\textbf{ATIVIDADE:} Atividade 7 - Tarefa: Rendas postecipadas (Valor 5,0) \\
\textbf{TUTOR(A):} Edilane Cristine Budasz \\
\textbf{PERÍODO:} Quarto \\

\textbf{Resolva os problemas a seguir de forma detalhada utilizando as fórmulas e com os passos da HP-12C.}
\\
\textbf{Digite em um editor de texto.} \\
\textbf{Não esqueça do cabeçalho padrão!} \\
\textbf{Depois anexe o arquivo nesta atividade.} \\
\textbf{Estes exercícios constam no livro didático.} \\
\textbf{Pag 77 - exercício 1,3,5,7 e 9 }\\

\begin{itemize}

\item Um televisor de R\$ 800,00 à vista é financiado em 10 pagamentos iguais, 
mensais e consecutivos, sem entrada, à taxa de 5,5\% a.m. Calcular o valor de 
cada prestação
  
$$ PMT = ? $$
$$ PV = R\$800,00 $$
$$ i = 5,5\% \Rightarrow i = 0,055 $$
$$ n = 10 $$
\\
$$ PMT = PV\frac{i}{1-(1+i)^{-n}} \Rightarrow $$
$$ PMT = 800\frac{0,055}{1-(1+0,055)^{-10}} \Rightarrow $$
$$ PMT = \frac{44}{1-\frac{1}{1,055}^{-10}} \Rightarrow $$
$$ PMT = \frac{44}{1-0,58543} \Rightarrow $$
$$ \boxed{PMT = R\$106,1342} $$

\\

Pela calculadora HP-12C\\\\
\emph{f REG}\\
\emph{g END}\\
\emph{800 CHS PV}\\
\emph{5,5 i}\\
\emph{10 n}\\
\emph{PMT 106,13}

\item  Uma loja vende um televisor por R\$ 850,00 à vista ou em 5 pagamentos postecipados, à taxa de 1,9\% ao mês. Qual o valor de cada prestação?
  
$$ PV = R\$850,00 $$
$$ i = 1,9\% \Rightarrow i = 0,019 $$
$$ n = 5 $$
$$ PMT = ? $$  
\\
$$ PV = PMT\frac{1-(1+i)^{-n}}{i} \Rightarrow $$
$$ 850 = PMT\frac{1-(1+0,019)^{-5}}{0,019} \Rightarrow $$
$$ 850 = PMT \cdot 4,7271 \Rightarrow $$
$$ \boxed{PMT = 179,81} $$
\\
Pela calculadora HP-12C\\\\
\emph{f REG}\\
\emph{g END}\\
\emph{850 CHS PV}\\
\emph{5 n}\\
\emph{1,9 i}\\
\emph{PV 179,81}  

\item Um veículo cujo preço à vista é R\$ 35.000,00 é vendido com 30\% de 
entrada mais 36 prestações postecipadas à taxa de 1,5\% am. Qual o valor de 
cada prestação?

$$ PV = R\$35000,00 \cdot \frac{3\bcancel{0}}{10\bcancel{0}} \Rightarrow PV = 35000 - 10500 \Rightarrow PV = 24500 $$
$$ i = 1,5\% \Rightarrow i = 0,015 $$
$$ n = 36 $$

\\
$$ PMT = PV\frac{i}{1-(1+i)^{-n}} \Rightarrow $$
$$ PMT = 24500\frac{0,015}{1-(1+0,015)^{-36}} \Rightarrow $$
$$ PMT = \frac{367,5}{0,414910} \Rightarrow $$
$$ \boxed{PMT = R\$885,73} $$
\\
Pela calculadora HP-12C\\\\
\emph{f REG}\\
\emph{g END}\\
\emph{35000 ENTER}\\
\emph{35000 ENTER}\\
\emph{30 \times 100 \div - CHS PV} \\
\emph{1,5 i} \\
\emph{36 n}\\
\emph{PMT 885,73}\\

\item Desejo obter em 36 meses o valor de R\$ 15.000,00 aplicando R\$ 400,00 
mensais em uma caderneta de poupança, ao final de cada mês. Que taxa devo 
procurar para que eu possa obter esse montante? \\\\

%Como nem no livro didático é feito o cálculo pela forma algébrica, foi feito apenas pela calculadora HP-12C \\

$$ PV = R\$15000,00 $$
$$ PMT = R\$400,00 $$
$$ n = 36 $$

  \begin{tabular}{ | c | c | c |}
  \hline
    -0,24 & 15.059,26 & \\ \hline
    i & 15.000,00 & 59,26 \\ \hline
    -0,20 & 14.946,57 & 112,69 \\ \hline  
  \end{tabular} \\\\

  \begin{tabular}{ c c }
    -0,24 & 112,69 \\ 
    i & 59,26 \\ 
  \end{tabular} 
  
$$ 2,3704 = 112,69i \Rightarrow $$
$$ i = \frac{2,3704}{112,69} $$
$$ \boxed{i = 2,1034... \cdot 10^{-2}}

Pela calculadora HP-12C\\\\
\emph{f REG}\\
\emph{g END}\\
\emph{15000 CHS PV}\\
\emph{36 n}\\
\emph{400 PMT}\\
\emph{i 0,22}  

\item Uma mercadoria que custa R\$ 5.000,00 à vista, foi comprada com 10\% de 
entrada e o saldo financiado em 10 pagamentos mensais. Se a loja cobra 3\% 
a.m., qual o valor de cada prestação

$$ PV = R\$500\bcancel{0},00 \cdot \frac{1\bcancel{0}}{1\bcancel{00}} \Rightarrow PV = 5000 - 500 \Rightarrow PV = 4500 $$
$$ n = 10 $$
$$ i = 3\% \Rightarrow i = 0,03 $$

$$ PMT = PV\frac{i}{1-(1+i)^{-n}} \Rightarrow $$
$$ PMT = 4500\frac{0,03}{1-(1+0,03)^{-10}} \Rightarrow $$
$$ PMT = \frac{135}{1-0,744093} \Rightarrow $$
$$ \boxed{PMT = 527,537} \Rightarrow $$

\\
Pela calculadora HP-12C\\\\
\emph{f REG}\\
\emph{g END}\\
\emph{5000 ENTER}\\
\emph{5000 ENTER}\\
\emph{10 \times 100 - 4500 CHS PV} \\
\emph{10 n}\\
\emph{3 i}\\
\emph{PMT 527,54}\\

\end{itemize}

\end{document}

