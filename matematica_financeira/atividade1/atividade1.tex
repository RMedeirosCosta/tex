\documentclass[a4paper, 12pt]{article}

\usepackage[top=2cm, bottom=2cm, left=2.5cm, right=2.5cm]{geometry}
\usepackage[utf8]{inputenc}
\usepackage{array}
\usepackage{graphicx}
\usepackage{amsmath}
\usepackage{cancel}

\graphicspath{{img/}}

\begin{document}
\begin{flushleft}\includegraphics{logo}\\
\textbf{UNIVERSIDADE ESTADUAL DE PONTA GROSSA} \\
SISTEMA UNIVERSIDADE ABERTA DO BRASIL - UAB \\
\underline{Licenciatura em Matemática | Polo UAB em Jacarezinho}\end{flushleft} 
\textbf{ALUNO:} Ricardo Medeiros da Costa Junior   \textbf{RA:} 151774301 \\
\textbf{DISCIPLINA:} Matemática Financeira \\
\textbf{ATIVIDADE:} Atividade 1 - Tarefa Razões, proporções e operações sobre mercadorias \\ 
\textbf{TUTOR(A):} Edilane Cristine Budasz \\
\textbf{PERÍODO:} Quarto \\

\begin{itemize}
\item Um negociante comprou uma mercadoria por R\$200,00. Para ter um lucro de 10\%, por quanto deverá revender o produto? \\
  $$ V = C + L \Rightarrow $$
  $$ V = 200 + (20\bcancel{0} \cdot \frac{1\bcancel{0}}{1\bcancel{00}}) \Rightarrow $$
  $$ V = 200 + 20 \Rightarrow $$
  $$ \boxed{V = R\$200,00} $$    
\item Na venda de um imóvel o proprietário obteve um lucro de 20\%. Se o preço pago pelo comprador foi de R\$600.000,00, qual foi o preço pago inicialmente pelo proprietário? \\
  $$ V = C + L \Rightarrow $$
  $$ 600000 = C + \frac{C2\bcancel{0}}{10\bcancel{0}} \Rightarrow $$  
  $$ 600000 = C \left( 1+\frac{2}{10} \right) \Rightarrow $$
  $$ \frac{600000}{\frac{12}{10}} = C \Rightarrow $$
  $$ 600000 \cdot \frac{10}{12} = C \Rightarrow $$
  $$ \frac{6000000}{12} = C \Rightarrow $$
  $$ \boxed{C = R\$500000,00} $$    
  
\item Um eletrodoméstico foi comprado por R\$750,00 e vendido por R\$1.020,00. Calcule o lucro, na forma percentual, sobre o preço de venda. \\
  $$ V = C + L \Rightarrow $$
  $$ 1020 = 750 + (\frac{102\bcancel{0}L}{10\bcancel{0}}) \Rightarrow $$
  $$ 270 = \frac{102L}{10} \Rightarrow $$
  $$ \frac{2700}{102} = L \Rightarrow $$  
  $$ \boxed{L \approx 26,47\%} $$    
  
\item Um carro foi vendido com prejuízo de 12\% sobre o preço de venda. Calcule o preço de venda, sabendo-se que o preço de custo foi de R\$28.350,00. \\
  $$ V = C + L \Rightarrow $$
  $$ V = 28350 + \left(- \frac{12V}{100} \right) \Rightarrow $$
  $$ V = 28350 - \frac{12V}{100} \Rightarrow $$
  $$ V + \frac{12V}{100} = 28350 \Rightarrow $$
  $$ \frac{112V}{100} = 28350 \Rightarrow $$
  $$ 112V = 2835000 \Rightarrow $$
  $$ V = \frac{2835000}{112} \Rightarrow $$
  $$ \boxed{V = 25312,50} \Rightarrow $$            
 
\item O investimento de R\$10.000,00 na melhoria da logística de uma empresa gera uma economia de R\$2.000,00. Qual a economia se investirmos R\$4.000,00? Para termos uma economia de R\$2.500,00, quanto devemos investir? \\
  $$ 10000x = 8000000 \Rightarrow $$
  $$ x = \frac{800\bcancel{0000}}{1\bcancel{0000}} \Rightarrow $$
  $$ \boxed{x = 800}  $$
  \\
  $$ 20000x = 25000000 \Rightarrow $$
  $$ x = \frac{25000\bcancel{000}}{2\bcancel{000}} \Rightarrow $$
  $$ \boxed{x = 12,500}  $$

    
\end{itemize}

\end{document}
