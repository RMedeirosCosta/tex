\documentclass[a4paper, 12pt]{article}

\usepackage[top=2cm, bottom=2cm, left=2.5cm, right=2.5cm]{geometry}
\usepackage[utf8]{inputenc}
\usepackage{array}
\usepackage{graphicx}
\usepackage{amsmath}

\graphicspath{{img/}}

\begin{document}
\begin{flushleft}\includegraphics{logo}\\
\textbf{UNIVERSIDADE ESTADUAL DE PONTA GROSSA} \\
SISTEMA UNIVERSIDADE ABERTA DO BRASIL - UAB \\
\underline{Licenciatura em Matemática | Polo UAB em Jacarezinho}\end{flushleft} 
\textbf{ALUNO:} Ricardo Medeiros da Costa Junior   \textbf{RA:} 151774301 \\
\textbf{DISCIPLINA:} Fundamentos da Matemática III \\
\textbf{ATIVIDADE:} Atividade 4 - Tarefa: Números Complexos\\
\textbf{TUTOR(A)}: Julio Cezar de Souza\\
\textbf{PERÍODO:} Terceiro\\
\begin{enumerate}
\item A raiz quadrada de -9:\\
  $\sqrt{-9} \Rightarrow$\\
  $\sqrt{9}\cdot\sqrt{(-1)} \Rightarrow$\\
  $3i$
\item $(2+2i)\cdot(1-3i)$ é igual a:\\
  $2-6i+2i-6i^2 \Rightarrow$\\
  $2-4i-6i^2 \Rightarrow$\\
  $-6(-1)-4i+2 \Rightarrow$\\
  $6+2-4i \Rightarrow$\\
  $8-4i$
\item Se $z = 2-3i$, seu conjugado é:\\
  $z = a + bi \Rightarrow \overline{z} = a - bi$\\
  $z = 2 - 3i \Rightarrow \overline{z} = 2 - (-3)i$\\
  $z = 2 - 3i \Rightarrow \overline{z} = 2 + 3i$\\
  $\overline{z} = 2 + 3i$
\item Somando $z_1 = 2 + 2i$ com $z_2 = 1 + 2i$ temos:\\
  $z_1 + z_2 = (a + bi) + (c + di) = (a + c) + (b + d)i$\\
  $z_1 + z_2 = (2 + 2i) + (1 + 2i) = (2 + 1) + (2 + 2)i \Righarrow$\\
  $z_1 + z_2 = 2 + 4i$
\item Se $2 + bi = (1 + a) - 2i$, calcule o valor de \textbf{a} e \textbf{b}:\\
  $z_1 = z_2 \Leftrightarrow a + bi = c + di \Leftrightarrow a = c$ e $b = d$\\
  $z_1 = z_2 \Leftrightarrow 2 + bi = (1 + a) - 2i \Leftrightarrow 2 = (1 + a) \Rightarrow 1 = a$ e $b = -2$  
\end{enumerate}
\end{document}
