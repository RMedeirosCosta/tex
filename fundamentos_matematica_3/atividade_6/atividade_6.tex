\documentclass[a4paper, 12pt]{article}

\usepackage[top=2cm, bottom=2cm, left=2.5cm, right=2.5cm]{geometry}
\usepackage[utf8]{inputenc}
\usepackage{array}
\usepackage{cancel}
\usepackage{graphicx}
\usepackage{amsmath}

\graphicspath{{img/}}

\begin{document}
\begin{flushleft}\includegraphics{logo}\\
\textbf{UNIVERSIDADE ESTADUAL DE PONTA GROSSA} \\
SISTEMA UNIVERSIDADE ABERTA DO BRASIL - UAB \\
\underline{Licenciatura em Matemática | Polo UAB em Jacarezinho}\end{flushleft} 
\textbf{ALUNO:} Ricardo Medeiros da Costa Junior   \textbf{RA:} 151774301 \\
\textbf{DISCIPLINA:} Fundamentos da Matemática III \\
\textbf{ATIVIDADE:} Atividade 6 - Tarefa: Polinômios\\
\textbf{TUTOR(A)}: Julio Cezar de Souza\\
\textbf{PERÍODO:} Terceiro\\
\begin{enumerate}
\item As dimensões em centímetros de uma caixa retangular fechada são $(x+1)$, $(2x-3)$ e $(x+2)$.
  \begin{enumerate}
  \item Escreva os polinômios que $A(x)$, que expressa a área da superfície, e $V(x)$, que expressa o volume da caixa. \\\\
    Legenda: \\
    \begin{itemize}
    \item $A_l = $ Área lateral
    \item $A_b = $ Área da base
    \item $A_t = $ Área total
    \item $l = $ Largura
    \item $c = $ Comprimento
    \item $h = $ Altura
    \end{itemize} \\
    Como o exercício não especifica qual polinômio é de cada medida é assumido, por hipótese, que:\\
    \begin{itemize}
    \item $l = (x+1) $
    \item $c = (x+2) $
    \item $h = (2x-3) $
    \item $A_b = 2lc \Rightarrow A_b = [(x+1)(x+2)] $
    \item $A_l = 2lh + 2ch \Rightarrow  2[(x+1)(2x-3)] + 2[(x+2)(2x-3)] $
    \item $A_b =  A_b + A_l $    
    \end{itemize}
    $$ A(x) = A_t \Rightarrow $$
    $$ A(x) = A_l + 2A_b \Rightarrow $$
    $$ A(x) =  2[(x+1)(2x-3)] + 2[(x+2)(2x-3)] + 2 [(x+1)(x+2)] \Rightarrow $$
    $$ A(x) = 2(x^2+\cancel{2x}+\cancel{x}+2) + 2(2x^2-\cancel{3x}+2x-3) + 2(2x^2-3x+4x-6) \Rightarrow $$
    $$ A(x) = 2(5x^2+3x-7) \Rightarrow $$
    $$ \boxed{A(x) = 10x^2+6x-14} $$ \\ \\ \\
    $$ V(x) = clh \Rightarrow $$
    $$ V(x) = (x+1)(2x-3)(x+2) \Rightarrow $$
    $$ V(x) = 2x^2-x-3(x+2) \Rightarrow $$
    $$ V(x) = 2x^3+4x^2-x^2-2x-3x-6 \Rightarrow $$
    $$ \boxed{ V(x) = 2x^3+3x^2-5x-6} $$
  \item  Encontre o volume da caixa, quando a área da superfície é $94 cm²$.
    $$ A(x) = 10x^2+6x-14 \Rightarrow $$
    $$ 94 = 10x^2+6x-14 \Rightarrow $$
    $$ 10x^2+6x-14 = 0 \Rightarrow $$ \\\\
    $$ \Delta = b^2-4ac \Rightarrow $$
    $$ \Delta = (6)^2-4\cdot10\cdot(-108) \Rightarrow $$
    $$ \Delta = 4356 $$ \\\\
    $$ \frac{-b \pm \sqrt{\Delta}}{2a} \Rightarrow $$
    $$ \frac{-6 \pm 66}{2\cdot10} \Rightarrow $$
    $$ - \frac{72}{20} \Rightarrow $$         
    $$ - \frac{18}{5} $$ \\\\
    $$ \frac{-6 \pm 66}{2\cdot10} \Rightarrow $$
    $$ \frac{\cancel{60}}{2\cdot\cancel{10}} \Rightarrow $$
    $$ \frac{6}{2} \Rightarrow $$
    $$ 3 $$ \\\\
    $$ V(3) = 2\cdot(3)^3 + 3\cdot(3)^2 - 5\cdot(3) - 6 \Rightarrow $$
    $$ V(3) = 2\cdot27 + 3\cdot9 - 15 - 6 \Rightarrow $$
    $$ \boxed{ V(3) = 60 cm^3} $$    
  \end{enumerate}
\item  Elabore um problema que envolva divisão de polinômios e explique a resolução pelo método da chave e pelo método de Briot-Ruffini.\\\\
  A área de uma folha é igual a $x^3-1 $ u.a. Quantas figuras de área $x-1$ u.a. pode-se desenhar nessa folha?\\
  $$ x^3-1 | \underline{x-1} $$ 
  $$ -\cancel{x^3}+x^2\ \ x^2+x+1 $$
  $$ x^2 - 1 $$ 
  $$ \cancel{-x^2} + x $$
  $$ x - 1 $$
  $$ \cancel{-x + 1} $$ 
  $$ \boxed{x^2+x+1} $$ \\\\ 
  $$ 1\ |\ 1\ 0\ 0\ -1 $$
  $$ \ \  1\ 1\ 1\ \ \ 0 $$
  $$ \boxed{x^2 + x + 1} $$ \\
  O polinômio referente a quantidade de desenhos é:  $ x^2+x+1 $ u.a.
  
\end{enumerate}
\end{document}
