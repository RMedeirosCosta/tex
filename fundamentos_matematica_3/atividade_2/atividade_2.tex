\documentclass[a4paper, 12pt]{article}

\usepackage[top=2cm, bottom=2cm, left=2.5cm, right=2.5cm]{geometry}
\usepackage[utf8]{inputenc}
\usepackage{array}
\usepackage{graphicx}
\usepackage{amsmath}

\graphicspath{{img/}}

\begin{document}
\begin{flushleft}\includegraphics{logo}\\
\textbf{UNIVERSIDADE ESTADUAL DE PONTA GROSSA} \\
SISTEMA UNIVERSIDADE ABERTA DO BRASIL - UAB \\
\underline{Licenciatura em Matemática | Polo UAB em Jacarezinho}\end{flushleft} 
\textbf{ALUNO:} Ricardo Medeiros da Costa Junior   \textbf{RA:} 151774301 \\
\textbf{DISCIPLINA:} Fundamentos da Matemática III \\
\textbf{ATIVIDADE:} Atividade 2 - Tarefa: Análise Combinatória \\
\textbf{TUTOR(A)}: Julio Cezar de Souza
\textbf{PERÍODO:} Terceiro
\begin{enumerate}
\item Um grupo de amigos é formado por 4 garotas e 4 rapazes. Crie quatro problemas a partir deste grupo. Um deles deve envolver permutação simples; outro, arranjo simples; outro combinação simples e o último deve envolver, pelo menos dois dos tipos destes agrupamentos.\\
  \begin{itemize}
  \item Permutação Simples \\
    Quantas possibilidades de fila podem ser formadas entre os 8 estudantes?\\
    8! = 40320
  \item Arranjo Simples \\
    Foi realizado uma competição e uma garota ficou na primeira posição. Quantos resultados diferentes podem acontecer para o segundo e terceiro lugar?
    $A_{7,2}=\frac{7!}{(7-2)!}\rightarrow
     A_{7,2}=42$
   \item Combinação Simples
     Supondo que para essa competição, um grupo de 5 competidores foram selecionados para um exame \emph{antidoping}. De quantas maneiras o grupo poderá ser formado?\\
     $C_{8,5}=\frac{8!}{5!(8-5)!}\rightarrow
      C_{8,5}=56$
    \item Híbrido \\
      O regulamento do campeonato diz que 5 competidores devem ser selecionados para o exame \emph{antidoping} e estes selecionados devem fazer uma fila para o exame. Quantos competidores podem ser selecionados para o exame e quantas as possibilidades de filas?
      \begin{enumerate}
        \item $C_{8,5}=\frac{8!}{5!(8-5)!}\rightarrow
          C_{8,5}=56$
        \item $P_{5}=5!\rightarrow
              P_{5}=120$        
      \end{enumerate}
  \end{itemize}
\item Elabore um texto que sintetize os principais conceitos do Triângulo de Pascal e o Binômio de Newton, ilustrando suas explicações com exemplos.\\
 O Triângulo de Pascal é um triângulo numérico infinito formado por números binomiais. O triângulo foi descoberto pelo matemático chinês Yang Hui (1238-1298)
e depois suas propriedades foram estudadas pelo francês Blaise Pascal (1623-1662). O Triângulo de Pascal possue diversas propriedades, sendo algumas delas:\\
\begin{itemize}
\item Relação de Stifel\\
Cada número do triângulo de Pascal é igual à soma do número imediatamente acima e do antecessor do número de cima.
\item Soma de uma linha\\
A soma de uma linha do triângulo de pascal é igual a $2ˆn$
\item Soma de uma coluna\\
A soma da coluna, no triângulo de Pascal, pode ser calcula pela relação
\item Simetria\\
O triângulo de Pascal apresenta simetria em relação à altura.
\item Soma de uma diagonal\\
A partir da propriedade da \textbf{Soma de uma coluna} e da \textbf{Simetria} é possível encontrar essa propriedade.
\item Desigualdades
Essa propriedade recente foi descoberta em 2014 e diz: \\
1- Em toda a infinita coluna central do Triângulo, na figura abaixo, o produto de dois de seus elementos é maior do que o produto de dois elementos pertencentes à mesma coluna central, localizados simetricamente entre eles. Por exemplo, na figura abaixo: 1 x 20 > 2 x 6, ou então, 2 x 20 > 6 x 6, ou ainda, 1 x 6 > 2 x 2. Isto vale para toda a coluna central.\\
2 - Dados dois elementos A e B da coluna central, o produto deles é maior do que o produto de dois elementos C e D pertencentes às diagonais que passam por A e por B, que estejam simetricamente localizados em relação a A e a B. Por exemplo, olhando novamente a figura acima: se A = 2 e B = 20, então:\\\\

2 x 20 > 3 x 10 > 4 x 4 > 1 x 5.\\\\

Se A = 1 e B = 20, então:\\\\

1 x 20 > 1 x 10 > 1 x 4 > 1 x 1.\\\\

O Triângulo de Pascal tem diversas aplicações, tais como: Binômio de Newton, Sequência de Fibonacci, para encontrar potências do número onze e o Triângulo de Sierpinski.\\\\

O Binômio de Newton permite facilitar o desenvolvimento de um binômio. Os coeficientes são chamados de coeficientes binomiais. Utilizando o Triângulo de Pascal é possível determinar os coeficientes binomiais. O teorema do binômio de Newton se escreve como segue:
$$(x+y)ˆn=\sum_{k=0}^{n}\binom{n}{k}xˆ(n-k)y^k$$
\end{itemize}
\end{enumerate}
\end{document}
