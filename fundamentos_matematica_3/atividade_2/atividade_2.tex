\documentclass[a4paper, 12pt]{article}

\usepackage[top=2cm, bottom=2cm, left=2.5cm, right=2.5cm]{geometry}
\usepackage[utf8]{inputenc}
\usepackage{array}
\usepackage{graphicx}
\usepackage{amsmath}

\graphicspath{{img/}}

\begin{document}
\begin{flushleft}\includegraphics{logo}\\
\textbf{UNIVERSIDADE ESTADUAL DE PONTA GROSSA} \\
SISTEMA UNIVERSIDADE ABERTA DO BRASIL - UAB \\
\underline{Licenciatura em Matemática | Polo UAB em Jacarezinho}\end{flushleft} 
\textbf{ALUNO:} Ricardo Medeiros da Costa Junior   \textbf{RA:} 151774301 \\
\textbf{DISCIPLINA:} Fundamentos da Matemática III \\
\textbf{ATIVIDADE:} Atividade 2 - Tarefa: Análise Combinatória \\
\begin{enumerate}
\item Um grupo de amigos é formado por 4 garotas e 4 rapazes. Crie quatro problemas a partir deste grupo. Um deles deve envolver permutação simples; outro, arranjo simples; outro combinação simples e o último deve envolver, pelo menos dois dos tipos destes agrupamentos.\\
  \begin{itemize}
  \item Permutação Simples \\
    Quantas possibilidades de fila podem ser formadas entre os 8 estudantes?\\
    8! = 40320
  \item Arranjo Simples \\
    Foi realizado uma competição e uma garota ficou na primeira posição. Quantos resultados diferentes podem acontecer para o segundo e terceiro lugar?
    $A_{7,2}=\frac{7!}{(7-2)!}\rightarrow
     A_{7,2}=42$
   \item Combinação Simples
     Supondo que para essa competição, um grupo de 5 competidores foram selecionados para um exame \emph{antidoping}. De quantas maneiras o grupo poderá ser formado?\\
     $C_{8,5}=\frac{8!}{5!(8-5)!}\rightarrow
      C_{8,5}=56$
    \item Híbrido \\
      O regulamento do campeonato diz que 5 competidores devem ser selecionados para o exame \emph{antidoping} e estes selecionados devem fazer uma fila para o exame. Quantos competidores podem ser selecionados para o exame e quantas as possibilidades de filas?
      \begin{enumerate}
        \item $C_{8,5}=\frac{8!}{5!(8-5)!}\rightarrow
          C_{8,5}=56$
        \item $P_{5}=5!\rightarrow
              P_{5}=120$        
      \end{enumerate}
  \end{itemize}
\item Elabore um texto que sintetize os principais conceitos do Triângulo de Pascal e o Binômio de Newton, ilustrando suas explicações com exemplos.\\
\end{enumerate}
\end{document}
