\documentclass[a4paper, 12pt]{article}

\usepackage[top=2cm, bottom=2cm, left=2.5cm, right=2.5cm]{geometry}
\usepackage[utf8]{inputenc}
\usepackage{array}
\usepackage{verbatim}
\usepackage{graphicx}
\usepackage{hyperref}

\graphicspath{{img/}}

\begin{document}
\includegraphics{logo}\\
\textbf{UNIVERSIDADE ESTADUAL DE PONTA GROSSA} \\
SISTEMA UNIVERSIDADE ABERTA DO BRASIL - UAB \\
\underline{Licenciatura em Matemática | Polo UAB em Jacarezinho} \\
\textbf{ALUNO:} Ricardo Medeiros da Costa Junior   \textbf{RA:} 151774301 \\
\textbf{DISCIPLINA:} Instrumentação para o Ensino da Matemática IV \\
\textbf{ATIVIDADE:} Tarefa atividade 2 \\

\begin{enumerate}
\item Consulte o Caderno de Expectativas de Aprendizagem de Matemática para o Ensino Médio do Paraná e resolva as questões abaixo:(Valor 5,0)
  \begin{enumerate}
  \item Quais são os conteúdos estruturantes de Matemática?\\\\
    Números e Álgebra, Grandezas e Medidas, Funções, Geometrias, Tratamento de Informação.    
  \item Monte um organograma relacionando os conteúdos estruturantes com os respectivos conteúdos básicos.
    \begin{description}
    \item[Números e Álgebra] Números reais, Números complexos, Sistemas Lineares, Matrizes e determinantes, Polinômios, Equações e Inequações exponenciais, logarítmicas e modulares.
    \item[Grandezas e Medidas] Medidas de área, Medidas de volume, Medidas de grandezas vetoriais, Medidas de informática, Medidas de energia, Trigonometria.
    \item[Funções] Função Afim, Função Quadrática, Função Polinomial, Função Exponencial, Função Logarítmica, Função Trigonométrica, Função Modular, Progressão Aritmética, Progressão Geométrica.
    \item[Geometrias] Geometria Plana, Geometria Espacial, Geometria Analítica, Geometrias Não Euclidianas.
    \item[Tratamento de Informação] Análise Combinatória, Binômio de Newton, Estudo das probabilidades, Estatística, Matemática Financeira.
    \end{description}
  \end{enumerate}
\item Quais são os conteúdos estruturantes ou eixos propostos para o ensino de Matemática nos PCN do Ensino Médio? Compare-os com o Caderno de expectativas do Paraná.(Valor 5,0)\\
  \begin{itemize}
  \item  Identificar variáveis relevantes e selecionar os procedimentos necessários para produção, análise e interpretação de resultados de processos ou experimentos científicos e tecnológicos;
  \item  Compreender o caráter aleatório e não-determinístico dos fenômenos naturais e sociais e utilizar instrumentos adequados para medidas, determinação de amostras e cálculo de probabilidades;
  \item  Identificar, analisar e aplicar conhecimentos sobre valores de variáveis, representados em gráficos, diagramas ou expressões algébricas, realizando previsão de tendências, extrapolações e interpolações, e interpretações;
  \item Analisar qualitativamente dados quantitativos, representados gráfica ou algebricamente, relacionados a contextos sócio-econômicos, científicos ou cotidianos;
  \item  identificar, representar e utilizar o conhecimento geométrico para o aperfeiçoamento da leitura, da compreensão e da ação sobre a realidade;
  \item Compreender conceitos, procedimentos e estratégias matemáticas, e aplicá-las a situações 
diversas no contexto das ciências, da tecnologia e das atividades cotidianas. 
  \end{itemize}  
  Ao meu ver, os conteúdos do Caderno de Expectativas do Paraná são mais específicos e facilitam para o docente, pois parecem ter sido elaborados por profissionais específicos de cada área, visto que são realmente os conteúdos que são (ou que eram para ser) passados aos alunos de cada respectivo ano. Os conteúdos do PCN são mais genéricos e abrangentes e parecem apresentar mais uma visão ``filosófica'' dos conteúdos, uma visão mais geral. Fazendo uma analogia a uma pesquisa científica, os conteúdos do PCN seriam os Objetivos Gerais e os conteúdos do Caderno de Expectativas do Paraná seria os Objetivos Específicos.
  
\end{enumerate}
\end{document}
