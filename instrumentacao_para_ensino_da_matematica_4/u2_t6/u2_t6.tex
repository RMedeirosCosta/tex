\documentclass[a4paper, 12pt]{article}

\usepackage[top=2cm, bottom=2cm, left=2.5cm, right=2.5cm]{geometry}
\usepackage[utf8]{inputenc}
\usepackage{array}
\usepackage{verbatim}
\usepackage{graphicx}
\usepackage{hyperref}

\graphicspath{{img/}}

\begin{document}
\includegraphics{logo}\\
\textbf{UNIVERSIDADE ESTADUAL DE PONTA GROSSA} \\
SISTEMA UNIVERSIDADE ABERTA DO BRASIL - UAB \\
\underline{Licenciatura em Matemática | Polo UAB em Jacarezinho} \\
\textbf{ALUNO:} Ricardo Medeiros da Costa Junior   \textbf{RA:} 151774301 \\
\textbf{DISCIPLINA:} Instrumentação para o Ensino da Matemática IV \\
\textbf{ATIVIDADE:} Atividade 6 - Tarefa Descritiva (Valor 5,0) \\

\begin{enumerate}
\item \textbf{Base Nacional Comum Curricular} \\
Atualmente no Brasil está em fase de discussão a construção da Base Nacional Comum Curricular - BNCC. Acesse no material obrigatório ou no site do MEC: \emph{http:\/\/basenacionalcomum.mec.gov.br\/\#/site\/conhecaDisciplina?disciplina=AC\_MAT\&tipoEnsino=TE\_EM} o documento preliminar para área de matemática (pág. 116 à 148) e responda as questões:

  \begin{enumerate}
  \item Quais são os objetivos gerais da área da matemática no Ensino Médio?
    \begin{itemize}
    \item Aplicar conhecimentos matemáticos em situações diversas, na compreensão das demais ciências, de modo a consolidar uma formação científica geral.
    \item Expressar-se oral, escrita e graficamente, valorizando a precisão da linguagem na comunicação de ideias e na argumentação matemática.
    \item Compreender a Matemática como ciência, com sua linguagem própria e estrutura lógico-dedutiva.
    \item Estabelecer relações entre conceitos matemáticos de um mesmo campo entre os diferentes eixos (Geometria, Grandezas e Medidas, Estatística e Probabilidade, Números e Operações, Álgebra e Funções), bem como entre a Matemática e outras áreas do conhecimento.
    \item Desenvolver a autoestima e perseverança na busca de soluções, trabalhando coletivamente, respeitando o modo de pensar dos/as colegas e aprendendo com eles/as.
    \item Analisar criticamente os usos da Matemática em diferentes práticas sociais e fenômenos naturais, para atuar e intervir na sociedade.
    \item Recorrer às tecnologias digitais para descrever e representar matematicamente situações e fenômenos da realidade, em especial aqueles relacionados ao mundo do trabalho.
    \end{itemize}
  \item Cite os eixos em que foram organizados os objetivos de aprendizagem de matemática para Educação Básica.
    \begin{itemize}
    \item Geometria
    \item Grandezas e Medidas
    \item Estatística e Probabilidade
    \item Números e Operações
    \item Álgebra e Funções
    \end{itemize}
  \item Porque é importante considerar a contextualização na elaboração do currículo de matemática para o Ensino Médio?\\\\
    Porque o estudante necessita desenvolver competência relativa à abstração, tendo em vista que ele/a deverá estabelecer ou apreender relações que são válidas em diferentes contextos. Portanto, para o ensino de um contexto matemático, é interessante considerar a importância do ciclo: contextualizar, descontextualizar e novamente contextualizar e, depois, reiniciar esse movimento
  \item Consulte os objetivos de aprendizagem do componente curricular matemática no Ensino Médio e elabore um quadro que contenha todos os objetivos de cada um dos 5 eixos propostos em cada ano, conforme modelo abaixo:\\
    \begin{description}
    \item[Primeiro Ano EM]\ \\
      \begin{description}
      \item[Geometria]\ \\
        \begin{itemize}
        \item Compreender o conceito de vetor, tanto do ponto de vista geométrico (coleção de segmentos orientados de mesmo comprimento, direção e sentido) quanto do ponto de vista algébrico, caracterizado por suas coordenadas.
        \item Operar com vetores (soma e multiplicação por um escalar), interpretanto essas operações geometricamente e representar transformações no plano por meio de vetores.
        \item Compreender e aplicar o teorema de Tales na resolução de problemas, incluindo a divisão de segmentos em partes proporcionais.
        \item Utilizar a semelhança de triângulos e o teorema de Pitágoras (exemplo: diagonais de prismas e da altura de pirâmides) para resolver e elaborar problemas.
        \item Compreender e aplicar as razões trigonométricas no triângulo retângulo e as relações trigonométricas em triângulos quaisquer.
        \item Construir vistar ortogonais de uma figurar espacial, representando-a em perspectiva a partir de suas vistas ortogonais.
        \end{itemize}
      \item[Grandezas e Medidas]\ \\
        \begin{itemize}
        \item Compreender a noção de grandezas formada por relações entre outras grandezas (exemplo: densidade, aceleração)
        \item Resolver e elaborar problemas envolvendo medida da área e do perímetro de figurar planas, incluindo o círculo, a circunferência e suas partes (exemplo: arcos, setores, coroas).
        \item Resolver e elaborar problemas de cálculo da medida do volume de cilindros e prismas retos.
        \end{itemize}
      \item[Estatística e Probabilidade]\ \\
        \begin{itemize}
        \item Descrever o espaço amostral de experimentos aleatórios, com e sem reposição, usando diagrama de árvore para contagem de possibilidades e o princípio multiplicativo para determinar a probabilidade de eventos.
        \item Construtir tabelas e gráficos adequados (barras, colunas, setores, linha e histogramas) para representar um conjunto de dados, preferencialmente utilizando tecnologias digitais.
        \item Realizar pesquisas, considerando todas as suas etapas (planejamento, incluindo discussão se será censitária ou por amostra e seleção de amostras, elaboração e aplicação de instrumentos de coleta, organização e representação dos dados, incluindo a construção de gráficos apropriados, interpretação, análise crítica e divulgação dos resultados).
        \item Utilizar a média, a mediana e a amplitude para descrever, comparar e interpretar dois conjuntos de dados numéricos em termos de localização (centro) e dispersão (amplitude).
        \end{itemize}
      \item[Números e Operações]\ \\
        \begin{itemize}
        \item Reconhecer as características dos diferentes conjuntos numéricos (naturais, inteiros, racionais, irracionais, reais), suas operações e propriedades e ao necessidade de ampliá-los.
        \item Reconhecer as relações entre as diferentes representações de um número real (decimal, fracionária, potência e radical), o módulo e o simétrico.
        \item Comprar e ordenar números reais e compreender intervalos numéricos, localizando-os na reta numérica.
        \item Resolver e elaborar problemas envolvendo porcentagem e juros compostos (vinculado ao crescimento exponecial), com ou sem o uso de tecnologias digitais.
        \end{itemize}
      \item[Álgebra e Funções]\ \\
        \begin{itemize}
         \item Resolver e elaborar problemas, envolvendo proporcionalidade entre duas ou mais grandezas, inclusive problemas envolvendo escalas, divisão em partes proporcionais e taxa de variação.
         \item Compreender função como um tipo de relação de dependência entre duas variáveis, ideias de domínio e de imagem, associando-as a representações gráfica e/ou algébrica.
         \item Reconhecer função afim em suas representações algébrica e gráfica, identificando variação (taxa, crescimento e decrescimento), pontos de intersecção de seu gráfico com os eixos coordenados e o sentido geométrico dos coeficientes da equação de uma reta.
         \item Descrever função linear como um tipo especial de função afim e associá-la a relações de proporcionalidade direta entre duas grandezas.
         \item Associar sequências numéricas de variação linear (PA) a funções afins de domínios discretos.
         \item Reconhecer função quadrática em suas representações algébrica e gráfica, considerando domínio, imagem, ponto de máximo e mínimo, intervalos de crescimento e decrescimento, pontos de intersecção com os eixos.           
        \end{itemize} 
      \end{description}
    \item[Segundo Ano EM]\ \\
      \item[Geometria]\ \\
        \begin{itemize}
        \item Utilizar o conceito de vetor para associar duas figuras congruentes à composição de transformações no plano (reflexão, translação e rotação), com ou sem uso de tecnologias digitais.
        \item Compreender o conceito de lugar geométrico (exemplo: mediatriz, bissetriz, circunferência).
        \item Resolver problemas, envolvendo figuras poligonais determinadas pelas coordenadas de seus vértices, incluindo o cálculo da distância entre dois pontos.
        \item Reconhecer características e elementos de poliedros (exemplo: faces, arestas, vértices, diagonais), incluindo poliedros regulares, prismas e pirâmides oblíquos.
        \end{itemize}
      \item[Grandezas e Medidas]\ \\
        \begin{itemize}
        \item Compreender o princípio de Cavalieri e utilizá-lo para estabelecer as fórmulas para o cálculo da medidade do volume de figuras geométricas espaciais.
        \item Resolver e elaborar problemas de cálculo da medida do volume de cilindros, prismas, pirâmides e cones retos.
        \end{itemize}
      \item[Estatística e Probabilidade]\ \\
        \begin{itemize}
        \item Determinar a probabilidade da união de dois eventos, utilizando representações diversas.
        \item Descrever o espaço amostral de experimentos aleatórios sucessivos, com e sem reposição.
        \item Calcular e interpretar medidas de dispersão (amplitude, desvio médio, variância e desvio padrão) para um conjunto de dados numéricos agrupados ou não.
        \item Realizar pesquisas, considerando todas as suas etapas e utilizando as medidas de tendência central e de dispersão para a interpretação dos dados e elaboração de relatórios.
        \end{itemize}        
      \item[Números e Operações]\ \\
        \begin{itemize}
        \item Compreender as ideias de densidade e completude dos números reais.
        \item Resolver e elaborar problemas, envolvendo porcentagem em situações financeiras (cálculos de acréscimos e decréscimos, taxa percentual e juros compostos, parcelamentos, financiamentos, dentre outros).
        \item Resolver e elaborar problemas de combinatória, envolvendo estratégias básicas de contagem.
        \end{itemize}
      \item[Álgebra e Funções]\ \\
        \begin{itemize}
        \item Resolver problemas que envolvam sistemas de três equações de primeiro grau e três incógnitas (por substituição e escalonamento).
        \item Reconhecer função exponencial em suas representações algébrica e gráfica, identificando domínio, imagem e crescimento e pontos de interseção com os eixos coordenados e associas sequências numéricas (PG) a funções exponencias de domínio discreto.
        \item Reconhecer funções definidas por mais de uma sentença (exemplos: função modular, tabela de imposto de renda etc.), em suas representações algébrica e gráfica, identificando domínios de validade, imagem, crescimento e decrescimento.
        \item Reconhecer funções seno e cosseno em suas representações algébricas e gráficas e descrevê-las, considerando domínios de validade, imagem e características especiais como periodicidade, amplitude, máximos e mínimos.
        \item Compreender e descrever transformações que ocorrem na forma gráfica, ao se alterarem os parâmetros da forma algébrica de funções (exemplo: o que ocorre com o gráfico da função y = ax + b ou y = b + a.senx, quando se altera o valor de a e/ou de b?), com o apoio de tecnologias digitais.
        \end{itemize}
    \item[Terceiro Ano EM]\ \\
      \item[Geometria]\ \\
        \begin{itemize}
        \item Organizar logicamente os conhecimentos de geometria plana, construídos ao longo da Educação Básica, compreendendo o método axiomático.
        \item Reconhecer posições relativas entre duas retas, entre dois planos e entre retas e planos.
        \item Associar os coeficientes de retas (paralelas, perpendiculares e oblíquas) às suas representações geométricas.
        \item Associar a equação de uma circunferência à sua representação no plano cartesiano.
        \item Resolver problemas que envolvem equações da reta e da circunferência.
        \end{itemize}
      \item[Grandezas e Medidas]\ \\
        \begin{itemize}
        \item Resolver e elaborar problemas de cálculo da medida de área da superfície e do volume de figuras geométricas espaciais (cilindro, prisma, pirâmide, cone e esfera)
        \end{itemize}
      \item[Estatística e Probabilidade]\ \\
      \begin{itemize} 
      \item Analisar os métodos de amostragem em relatórios de pesquisas divulgadas pela mídia e as afirmativas feitas para toda a população baseadas em uma amostra.
      \item Analisar gráficos de relatórios estatísticos que podem induzir a erro de interpretação do leitor, verificando as escalas utilizadas, a apresentação de frequência relativas na comparação de populações distintas.
      \item Compreender o significado e a importância da curva normal.
      \item Interpretar e calcular medidas de posição (inclusive os quartis) e de dispersão para analisar um conjunto de dados.
      \end{itemize}
      \item[Números e Operações]\ \\
      \begin{itemize}
      \item Resolver e elaborar problemas de combinatória.
      \item Resolver e elaborar problemas envolvendo porcentagem em situações financeiras.
      \end{itemize}
      \item[Álgebra e Funções]\ \\
      \begin{itemize}
      \item Utilizar funções para representar situações reais, com ou sem o uso de tecnologias digitais.
      \item Compreender e descrever transformações que ocorrem na forma gráfica, ao se alterarem os parâmetros da forma algébrica de funções (exemplo: o que ocorre com o gráfico da função y = ax + b ou y = b +a.senx quando se altera o valor de a e/ou de b?), com o apoio de tecnologias digitais.
      \end{itemize}
    \end{description}
  \end{enumerate} 
\end{enumerate}
\end{document}
