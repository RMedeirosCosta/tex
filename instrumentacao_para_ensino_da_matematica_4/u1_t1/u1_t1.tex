\documentclass[a4paper, 12pt]{article}

\usepackage[top=2cm, bottom=2cm, left=2.5cm, right=2.5cm]{geometry}
\usepackage[utf8]{inputenc}
\usepackage{array}
\usepackage{verbatim}
\usepackage{graphicx}
\usepackage{hyperref}

\graphicspath{{img/}}

\begin{document}
\includegraphics{logo}\\
\textbf{UNIVERSIDADE ESTADUAL DE PONTA GROSSA} \\
SISTEMA UNIVERSIDADE ABERTA DO BRASIL - UAB \\
\underline{Licenciatura em Matemática | Polo UAB em Jacarezinho} \\
\textbf{ALUNO:} Ricardo Medeiros da Costa Junior   \textbf{RA:} 151774301 \\
\textbf{DISCIPLINA:} Instrumentação para o Ensino da Matemática IV \\
\textbf{ATIVIDADE:} Tarefa atividade 1 \\

\begin{enumerate}
\item Após a leitura e  estudo dos Parâmetros Curriculares  Nacionais  do Ensino Médio  responda as questões abaixo: (Valor 4)
  \begin{enumerate}
  \item Quais foram as mudanças estruturais na sociedade que desencadearam a reforma curricular no Ensino Médio?\\\\
     Primeiramente, o fator econômico se apresenta e se define pela ruptura tecnológica característica da chamada terceira revolução técnico-industrial, na qual os avanços da micro-eletrônica têm um papel preponderante, e, a partir década de 80, se acentuam no País.  \\
   A denominada ``revolução informática'' promove mudanças radicais na área do conhecimento, que passa a ocupar um lugar central nos processos de desenvolvimento, em geral. É possível afirmar que, nas próximas décadas, a educação vá se transformar mais rapidamente do que em muitas outras, em função de uma nova compreensão teórica sobre o papel da escola, estimulada pela incorporação das novas tecnologias.      
 \item A principal referência legal para a reformulação do Ensino Médio foi a LDBEN 9394/96. Qual foi o parecer do Conselho Nacional de Educação que aprovou a nova proposta curricular do Ensino Médio?\\\\
        O Parecer do Conselho Nacional de Educação foi aprovado em 1/06/98 - Parecer nº 15/98 
da Câmara de Educação Básica (CEB), do Conselho Nacional de Educação (CNE), seguindo-se a elaboração da Resolução que estabelece as Diretrizes Curriculares Nacionais para o Ensino Médio, Resolução CEB/CNE nº 03/98 e à qual o Parecer se integra.  
\item Explique as características de terminalidade do atual Ensino Médio brasileiro.\\\\
   O Ensino Médio passa a ter a característica da terminalidade, o que significa assegurar a todos os cidadãos a oportunidade de consolidar e aprofundar os conhecimentos adquiridos no Ensino Fundamental; aprimorar o educando como pessoa humana; possibilitar o prosseguimento de estudos; garantir a preparação básica para o trabalho e a cidadania; dotar o educando dos instrumentos que o permitam ``continuar aprendendo'', tendo em vista o desenvolvimento da compreensão dos ``fundamentos científicos e tecnológicos dos processos produtivos''(Art.35, incisos I a IV). \\
     O Ensino Médio, portanto, é a etapa final de uma educação de caráter geral, afinada com a contemporaneidade, com a construção de competências básicas, que situem o educando como sujeito produtor de conhecimento e participante do mundo do trabalho, e com o desenvolvimento da pessoa, como ``sujeito em situação'' - cidadão. 
   \item Cite as finalidades do Ensino Médio contidas na LDB 9394/96. \\\\
     \begin{description}
     \item[I - ] Desenvolvimento da capacidade de aprender e continuar aprendendo, da autonomia intelectual e do pensamento crítico, de modo a ser capaz de prosseguir os estudos e de adaptar-se com flexibilidade a novas condições de ocupação ou aperfeiçoamento;
     \item[II - ] constituição de significados socialmente construídos e reconhecidos como verdadeiros sobre o mundo físico e natural, sobre a realidade social e política;
     \item[III - ] Compreensão do significado das ciências, das letras e das artes e do processo de transformação da sociedade e da cultura, em especial as do Brasil, de modo a possuir as competências e habilidades necessárias ao exercício da cidadania e do trabalho;
     \item[IV - ] Domínio dos princípios e fundamentos científico-tecnológicos que presidem a produção moderna de bens, serviços e conhecimentos, tanto em seus produtos como em seus processos, de modo a ser capaz de relacionar a teoria com a prática e o desenvolvimento da flexibilidade para novas condições de ocupação ou aperfeiçoamento posteriores; 
     \item[V - ]  Competência no uso da língua portuguesa, das línguas estrangeiras e outras linguagens contemporâneas como instrumentos de comunicação e como processos de constituição de conhecimento e de exercício de cidadania;         
     \end{description}
  \end{enumerate}
\item Leia atentamente as páginas 11 a 13 dos PCN Ensino Médio e faça uma síntese de 10 a 15 linhas sobre a relação entre ``O Papel da Educação na Sociedade Tecnológica e a Nova Concepção Curricular do Ensino Médio''. (Valor 3) \\\\
  É explicado no começo do texto que devido a revolução tecnológica, rompe-se o paradigma no qual a educação seria um instrumento de ``conformação''. Segundo o PCN, isto ocorre porque as competências cognitivas exigidas para o desenvolvimento humano passa a coincidir do que se espera no ambiente de produção. No entanto, o PCN salienta que essa aproximação não garante uma homogeneização das oportunidades sociais. Também é destacado que a necessidade que todos desenvolvam e ampliem suas capacidades é indispensável para combater as desigualdades sociais.\\
  São constatados algumas falhas no processo de ensino e são expostas necessidades de se investir em macroplanejamento, formação de docentes, tratamento dos conteúdos e incorporação de instrumentos tecnológicos modernos, como a informática. \\
  Por fim, é comparado a velocidade das mudanças tecnológicas ocorridas no século passado, tais como: a máquina a vapor ou o motor a explosão, cuja difusão era lenta e ocorria em um longo período de tempo. Diferente dos avanços do conhecimento que se observam neste século. Portanto é proposto no PCN, alteração dos objetivos de formação no nível do Ensino Médio. Prioriza-se a formação ética e o desenvolvimento da autonomia intelectual e do pensamento crítico. O que se deseja é que os estudantes desenvolvam competências básicas que lhes permitam desenvolver a capacidade de continuar aprendendo.  
  
\item A organização Curricular do Ensino Médio, entende que o currículo enquanto instrumentação da cidadania democrática, deve contemplar conteúdos e estratégias de aprendizagem que capacitem o ser humano para a realização de atividades nos três domínios da ação humana: a vida em sociedade, a atividade produtiva e a experiência subjetiva, visando à integração de homens e mulheres no tríplice universo das relações políticas, do trabalho e da simbolização subjetiva. Por isso incorporam os quatro pilares da educação ou premissas apontadas pela UNESCO como eixos estruturais da educação: aprender a conhecer, aprender a fazer, aprender a viver e aprender a ser.\\

  Explique as características desses princípios e sua relação com os três domínios da atividade humana que devem ser contemplados pelo novo Ensino Médio. (Valor 3)
  %  Como citado no texto ``O Papel da Educação na Sociedade Tecnológica e a Nova Concepção Curricular do Ensino Médio'', devido a revolução tecnológica que ocorreu no final do século XX e início do século XXI, no qual as tecnologias evoluem de maneira muito veloz, faz-se necessário formar cidadãos autônomos, cujo desenvolvam competências básicas que lhes permitam continuar aprendendo. Essa autonomia vai de encontro com os três domínios da ação humana: vida em sociedade, atividade produtiva e experiência subjetiva, pois para participar da vida em sociedade e da atividade produtiva é necessário sempre se atualizar e não apenas memorizar métodos, pois estes podem ficar obsoletos em pouco tempo.  %
  \begin{itemize}
  \item Aprender a conhecer\\\\
    Considera-se a importância de uma educação geral, suficientemente ampla, com possibilidade de aprofundamento em determinada área de conhecimento. Prioriza-se o domínio dos próprios instrumentos do conhecimento, considerado como meio e como fim. Meio, enquanto forma de compreender a complexidade do mundo, condição necessária para viver dignamente, para desenvolver possibilidades pessoais e profissionais, para se comunicar. Fim, porque seu fundamento é o prazer de compreender, de conhecer, de descobrir.\\
    O aumento dos saberes que permitem compreender o mundo favorece o desenvolvimento da curiosidade intelectual, estimula o senso crítico e permite compreender o real, mediante a aquisição da autonomia na capacidade de discernir.\\
    Aprender a conhecer garante o aprender a aprender e constitui o passaporte para a educação permanente, na medida em que fornece as bases para continuar aprendendo ao longo da vida.
  \item Aprender a fazer\\\\
    O desenvolvimento de habilidades e o estímulo ao surgimento de novas aptidões tornam-se processos essenciais, na medida em que criam as condições necessárias para o enfrentamento das novas situações que se colocam.\\
    Privilegiar a aplicação da teoria na prática e enriquecer a vivência da ciência na tecnologia e destas no social passa a ter uma significação especial no desenvolvimento da sociedade contemporânea.
  \item Aprender a viver\\\\
    Trata-se de aprender a viver juntos, desenvolvendo o conhecimento do outro e a percepção das interdependências, de modo a permitir a realização de projetos comuns ou a gestão inteligente dos conflitos inevitáveis.  
  \item Aprender a ser\\\\
    A educação deve estar comprometida com o desenvolvimento total da pessoa. Aprender a ser supõe a preparação do indivíduo para elaborar pensamentos autônomos e críticos e para formular os seus próprios juízos de valor, de modo a poder decidir por si mesmo, frente às diferentes circunstâncias da vida. Supõe ainda exercitar a liberdade de pensamento, discernimento, sentimento e imaginação, para desenvolver os seus talentos e permanecer, tanto quanto possível, dono do seu próprio destino.\\
    Aprender a viver e aprender a ser decorrem, assim, das duas aprendizagens anteriores - aprender a conhecer e aprender a fazer - e devem constituir ações permanentes que visem à formação do educando como pessoa e como cidadão.\\
    A partir desses princípios gerais, o currículo deve ser articulado em torno de eixos básicos orientadores da seleção de conteúdos significativos, tendo em vista as competências e habilidades que se pretende desenvolver no Ensino Médio. \\
    Um eixo histórico-cultural dimensiona o valor histórico e social dos conhecimentos, tendo em vista o contexto da sociedade em constante mudança e submetendo o currículo a uma verdadeira prova de validade e de relevância social. Um eixo epistemológico reconstrói os procedimentos envolvidos nos processos de conhecimento, assegurando a eficácia desses processos e a abertura para novos conhecimentos.
  \end{itemize}
\end{enumerate}
\end{document}
