\documentclass[a4paper, 12pt]{article}

\usepackage[top=2cm, bottom=2cm, left=2.5cm, right=2.5cm]{geometry}
\usepackage[utf8]{inputenc}
\usepackage{array}
\usepackage{verbatim}
\usepackage{graphicx}
\usepackage{hyperref}

\graphicspath{{img/}}

\begin{document}
\includegraphics{logo}\\
\textbf{UNIVERSIDADE ESTADUAL DE PONTA GROSSA} \\
SISTEMA UNIVERSIDADE ABERTA DO BRASIL - UAB \\
\underline{Licenciatura em Matemática | Polo UAB em Jacarezinho} \\
\textbf{ALUNO:} Ricardo Medeiros da Costa Junior   \textbf{RA:} 151774301 \\
\textbf{DISCIPLINA:} Instrumentação para o Ensino da Matemática IV \\
\textbf{ATIVIDADE:} Atividade 4 - Tarefa Descritiva (Valor: 10,0) \\
\begin{enumerate}
\item Os eixos estruturantes segundo PCN(+) para Ensino Médio são:\\
  \begin{description}
  \item[Álgebra: Números e Funções]\ \\   
    \begin{itemize} 
    \item Variação de grandezas
    \item Trigonometria
    \item Funções
    \item Sequências
    \item Equações polinomiais
    \item Sistemas lineares \\
    \end{itemize}
  \item[Geometria e Medidas]\ \\
    \begin{itemize}
    \item Geometria plana
    \item Geometria espacial
    \item Geometria analítica
    \item Métrica \\
    \end{itemize}    
  \item[Análise de dados]\ \\
    \begin{itemize}
    \item Estatística
    \item Contagem
    \item Probabilidade \\
    \end{itemize}        
  \end{description} 
\item Os eixos estruturantes segundo DCE são:
  \begin{description}
  \item[Números e Álgebra]\ \\
    \begin{itemize}
    \item Números reais
    \item Números complexos
    \item Sistemas lineares
    \item Matrizes e determinantes
    \item Polinômios
    \item Equações e inequações exponenciais, logarítmicas e modulares.\\
    \end{itemize}   
  \item[Grandezas e Medidas]\ \\
    \begin{itemize}
    \item Medidas de área
    \item Medidas de volume
    \item Medidas de grandezas vetoriais
    \item Medidas de informática
    \item Medidas de energia
    \item Trigonometria \\
    \end{itemize}   
  \item[Geometrias]\ \\
    \begin{itemize}
    \item Geometria plana
    \item Geometria espacial
    \item Geometria analítica
    \item Geometrias não-euclidianas \\
    \end{itemize}       
  \item[Funções]\ \\
    \begin{itemize}
    \item Função afim
    \item Função quadrática
    \item Função polinomial
    \item Função exponencial
    \item Função logarítmica
    \item Função trigonométrica
    \item Função modular
    \item Progressão aritmética
    \item Progressão geométrica \\
    \end{itemize}       
  \item[Tratamento da Informação]\ \\
    \begin{itemize}
    \item Análise combinatória
    \item Binômio de Newton
    \item Estudo das probabilidades
    \item Estatística
    \item Matemática Financeira \\
    \end{itemize}           
  \end{description}
\item Em seguida faça uma análise sobre as semelhanças e diferenças das propostas de organização curricular de matemática dos dois documentos. \\\\
  Nota-se que os dois documentos possuem a organização curricular bem parecida. Praticamente todos os conteúdos propostos em um documento estão no outro. No entanto, o DCE é mais detalhado e possue alguns conteúdos que aparentemente não estão no PCN+. Um exemplo disso é como o eixo \textbf{Funções} está destacado e seus subtópicos bem especificados no DCE. O que não ocorre no PCN+, cujo só aparece \textbf{Funções} como um conteúdo proposto de \textbf{Álgebra: números e funções}. \\
  Há alguns casos que os conteúdos aparece em ambos, mas com nomes diferentes. Por exemplo, no PCN+ há o conteúdo \textbf{Sequências} em \textbf{Álgebra: números e funções}. Por outro lado, no DCE há \textbf{Progressão aritmética/geométrica}. Sabe-se que sequências são pré-requisitos para aprender PA e PG, portanto conclui-se que no DCE também há o conteúdo sequências. \\
  E por fim, existe conteúdos que estão em um documento e não estão em outro. Um caso são as \textbf{Geometrias não-euclidianas} que estão no DCE e não estão no PCN+.
\end{enumerate}
\end{document}
