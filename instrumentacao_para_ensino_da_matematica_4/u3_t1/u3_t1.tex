\documentclass[a4paper, 12pt]{article}

\usepackage[top=2cm, bottom=2cm, left=2.5cm, right=2.5cm]{geometry}
\usepackage[utf8]{inputenc}
\usepackage{array}
\usepackage{graphicx}

\graphicspath{{img/}}

\begin{document}
\begin{flushleft}\includegraphics{logo}\\
\textbf{UNIVERSIDADE ESTADUAL DE PONTA GROSSA} \\
SISTEMA UNIVERSIDADE ABERTA DO BRASIL - UAB \\
\underline{Licenciatura em Matemática | Polo UAB em Jacarezinho}\end{flushleft} 
\textbf{ALUNO:} Ricardo Medeiros da Costa Junior   \textbf{RA:} 151774301 \\
\textbf{DISCIPLINA:} Instrumentação para o Ensino da Matemática IV \\
\textbf{ATIVIDADE:} Atividade 7 - Tarefa Descritiva (Valor: 8,0) \\ 
\textbf{TUTOR(A):} Giane Correia Silva \\ 
\textbf{PERÍODO:} Quarto \\ \\
Após o estudo do documento ``Guia de Livros Didáticos - PNLD 2015 - Ensino Médio - Matemática'', disponível no material obrigatório da Unidade 3, realize as atividades abaixo:

\begin{enumerate}
\item Como está organizado o documento ``Guia de Livros Didáticos - PNLD 2015 - Ensino Médio - Matemática'' e qual é seu objetivo? \textbf{(valor 1)} \\\\
  O documento está organizado em forma de resenhas, cada obra apresenta uma resenha e alguns gráficos nos quais demonstram como estão organizados os conteúdos matemáticos nessas obras. Essas resenhas têm por objetivo contribuir para que o professor exerça seu papel insubstituível de escolher o texto didático que o apoiará na tarefa de formação de seus alunos no ensino médio. Em cada resenha há uma descrição resumida e uma avaliação das características de cada uma das obras aprovadas. \\
\item  A formação do estudante do Ensino Médio é papel fundamental das escolas. A sala de aula nesse contexto é o cenário no qual se estabelecem as inter-relaçções entre o professor, o livro didático e os saberes disciplinares. Qual é o papel do livro didático nesse espaço educativo? \textbf{(valor 1)} \\\\
  O papel do livro didático é trazer para o processo de ensino e aprendizagem um terceiro personagem, o seu autor, que passa a dialogar com o professor e com o aluno. Nesse diálogo, o livro é portador de escolhas sobre: o saber a ser estudado; os métodos adotados para que o aluno consiga aprendê-lo mais eficazmente; e a organização dos conteúdos ao longo dos anos de escolaridade.
\item Segundo Gérard \& Roegiers (1998) o livro didático tem funções para o professor e para o aluno.  Relacione as funções do livro para: \textbf{(valor 2)}
  \begin{itemize}
  \item o aluno: \\
    \begin{itemize}
    \item favorecer a aquisição de saberes socialmente relevantes;
    \item consolidar, ampliar, aprofundar e integrar os conhecimentos;
    \item propiciar o desenvolvimento de competências e habilidades do aluno, que contribuam para aumentar sua autonomia;
    \item contribuir para a formação social e cultural e desenvolver a capacidade de convivência e de exercício da cidadania.
    \end{itemize}
  \item o professor:
    \begin{itemize}
    \item auxiliar no planejamento didático-pedagógico anual e na gestão das aulas;
    \item favorecer na formação didático-pedagógica;
    \item auxiliar na avaliação da aprendizagem do aluno;
    \item favorecer a aquisição de saberes profissionais pertinentes, assumindo o papel de texto de referência.
    \end{itemize}
  \end{itemize}
\item Quais são os critérios gerais definidos pelo PNLD para a avaliação do componente curricular de Matemática para o Ensino Médio? \textbf{(valor 2)} \\
  \begin{itemize}
  \item incluir todos os campos da Matemática escolar, a saber, números, funções, equações algébricas, geometria analítica, geometria, estatística e probabilidade;
  \item privilegiar  a  exploração  dos  conceitos  matemáticos  e  de  sua  utilidade  
para resolver problemas;
  \item apresentar os conceitos com encadeamento lógico, evitando: recorrer a conceitos ainda não definidos para introduzir outro conceito, utilizar-se de definições circulares, confundir tese com hipótese em demonstrações matemáticas, entre outros;
  \item propiciar o desenvolvimento, pelo aluno, de competências cognitivas básicas, como: observação, compreensão, argumentação, organização, análise, síntese, comunicação de ideias matemáticas, memorização, entre outras.
  \end{itemize}
\item Cite os critérios de avaliação definidos pelo PNLD 2015 para o manual do professor. \textbf{(valor 2)}
  \begin{itemize}
  \item apresente linguagem adequada ao seu leitor - o professor - e atenda ao seu objetivo como manual de orientações didáticas, metodológicas e de apoio ao trabalho em sala de aula;
  \item contribua para a formação do professor, oferecendo discussões atualizadas acerca de temas relevantes para o trabalho docente, tais como currículo,  aprendizagem,  natureza  do  conhecimento  matemático  e  de  sua aplicabilidade, avaliação, políticas educacionais, entre outros;
  \item integre os textos e documentos reproduzidos em um todo coerente com a proposta metodológica adotada e com a visão de Matemática e de seu ensino e aprendizagem preconizadas na obra;
  \item não se limite a considerações gerais ao discutir a avaliação em Matemática, mas ofereça orientações efetivas \textbf{do que, como, quando e para que} avaliar,  relacionando-as  com  os  conteúdos  expostos  nos  vários  capítulos, unidades, seções;
  \item contenha,  além  do  Livro  do  Aluno,  orientações  para  o  docente  exercer suas funções em sala de aula, bem como propostas de atividades individuais e em grupo;
  \item explicite as alternativas e recursos didáticos ao alcance do docente, permitindo-lhe selecionar, caso o deseje, os conteúdos que apresentará em sala de aula e a sequência em que serão apresentados;
  \item contenha  as  soluções  detalhadas  de  todos  os  problemas  e  exercícios,  além de orientações de como abordar e tirar o melhor proveito das atividades propostas;
  \item apresente uma bibliografia atualizada para aperfeiçoamento do professor, agrupando os títulos indicados por área de interesse e comentando-os;
  \item separe,  claramente,  as  leituras  indicadas  para  os  alunos  daquelas  que são recomendadas para o professor.
  \end{itemize}
\end{enumerate}
\end{document}
