\documentclass[a4paper, 12pt]{article}

\usepackage[top=2cm, bottom=2cm, left=2.5cm, right=2.5cm]{geometry}
\usepackage[utf8]{inputenc}
\usepackage{array}
\usepackage{graphicx}

\graphicspath{{img/}}

\begin{document}
\begin{flushleft}\includegraphics{logo}\\
\textbf{UNIVERSIDADE ESTADUAL DE PONTA GROSSA} \\
SISTEMA UNIVERSIDADE ABERTA DO BRASIL - UAB \\
\underline{Licenciatura em Matemática | Polo UAB em Jacarezinho}\end{flushleft} 
\textbf{ALUNO:} Ricardo Medeiros da Costa Junior   \textbf{RA:} 151774301 \\
\textbf{DISCIPLINA:} 101505 - Cálculo Diferencial e Integral II \\
\textbf{ATIVIDADE:} Tarefa - Unidade V (Valor: 6 pontos) \\
\textbf{TUTOR(A):} Adilane de Assis Ferreira \\
\textbf{PERÍODO:} Quarto Período \\
\begin{enumerate}
\item ({\it Valor: 1 ponto}) Expresse a equação $ y = \frac{x^{2}+2xy-y^{2}}{x^{2}+y^{2}} $ em coordenadas polares (escreva a equação em termos de r e $ \theta $ ).
\item ({\it Valor: 1 ponto}) Expresse a equação $ r^{2} = sen^{2}( \theta )-cos^{2}( \theta ) $ em coordenadas cartesianas (escreva a equação em termos de x e de y).
\item ({\it Valor: 2.0 ponto}) Calcular a área limitada pela curva $r = a(1+cos \theta) $.
\item ({\it Valor: 2,0 ponto}) Calcular o comprimento da espiral $ r = e^{ \theta } $, para $ \theta \in [0,2\pi] $. 
  
\end{enumerate}
\end{document}
