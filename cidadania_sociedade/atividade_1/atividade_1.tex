\documentclass[a4paper, 12pt]{article}

\usepackage[top=2cm, bottom=2cm, left=2.5cm, right=2.5cm]{geometry}
\usepackage[utf8]{inputenc}
\usepackage{array}
\usepackage{graphicx}

\graphicspath{{img/}}

\begin{document}
\begin{flushleft}\includegraphics{logo}\\
\textbf{UNIVERSIDADE ESTADUAL DE PONTA GROSSA} \\
SISTEMA UNIVERSIDADE ABERTA DO BRASIL - UAB \\
\underline{Licenciatura em Matemática | Polo UAB em Jacarezinho}\end{flushleft} 
\textbf{ALUNO:} Ricardo Medeiros da Costa Junior   \textbf{RA:} 151774301 \\
\textbf{DISCIPLINA:} Cidadania e Sociedade \\
\textbf{ATIVIDADE:} Atividade 1 \\
\textbf{TUTOR(A):} Edilane Cristine Budasz\\
\textbf{PERÍODO:} Terceiro \\ \\
\begin{itemize}
\item Atualmente, apesar das conquistas efetivadas, observa-se que grandes parcelas da população brasileira não têm acesso a todos os direitos sociais, civis e políticos que lhes são devidos.
\item Verifica-se que em nosso país, embora tenham ocorrido avanços importantes no que diz respeito à efetivação dos direitos dos cidadãos, ainda existem sérios problemas a serem enfrentados, quando se pensa numa sociedade verdadeiramente democrática.
\item Nesse processo de conquista da cidadania para todos evidencia-se a importância da educação como um dos fatores capazes de favorecer a mobilização e a organização autônoma da sociedade, no sentido de lutar pela eliminação dos vícios históricos que têm dificultado essa conquista.
\item A escola, veículo por excelência da educação formal, exerce papel de destaque na formação política dos membros da sociedade, especialmente na atual conjuntura, em que o conhecimento é condição indispensável para o pleno exercício da cidadania.
\item Assim, a educação, além de um direito social básico e elementar, representa também o caminho - ou a condição necessária - que vai permitir o exercício e a conquista do conjunto dos direitos e deveres da cidadania, que se ampliam a cada dia em contrapartida às necessidades do homem e da dignidade humana.
\item As instituições escolares, ao garantirem à sociedade a prestação de uma educação democrática, pluralista e de qualidade, estarão contribuindo para a efetivação de uma importante tarefa face à justiça social.
  Desse modo, é atribuição da educação escolar ajudar os alunos a se tornarem sujeitos pensantes, capazes de construir elementos que os levem à compreensão e apreensão crítica da realidade, rumo a um projeto coletivo de exercício da cidadania consciente e responsável.
\end{itemize}
\end{document}
