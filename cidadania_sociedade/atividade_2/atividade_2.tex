\documentclass[a4paper, 12pt]{article}

\usepackage[top=2cm, bottom=2cm, left=2.5cm, right=2.5cm]{geometry}
\usepackage[utf8]{inputenc}
\usepackage{array}
\usepackage{graphicx}

\graphicspath{{img/}}

\begin{document}
\begin{flushleft}\includegraphics{logo}\\
\textbf{UNIVERSIDADE ESTADUAL DE PONTA GROSSA} \\
SISTEMA UNIVERSIDADE ABERTA DO BRASIL - UAB \\
\underline{Licenciatura em Matemática | Polo UAB em Jacarezinho}\end{flushleft} 
\textbf{ALUNO:} Ricardo Medeiros da Costa Junior   \textbf{RA:} 151774301 \\
\textbf{DISCIPLINA:} Cidadania e Sociedade \\
\textbf{ATIVIDADE:} Atividade 2 \\
\textbf{TUTOR(A):} Edilane Cristine Budasz\\
\textbf{PERÍODO:} Terceiro \\ \\
\begin{itemize}
  De forma sutil a banda Skank faz uma crítica ao povo brasileiro. Povo que por meio do péssimo ensino público, continua ignorante de seus direitos, não exerce sua cidadania. Por consequência disso se tornam cidadãos apolíticos, passíveis de manipulação. Que perdem muito tempo com entretenimento, como no exemplo da TV e do rádio e não analisam criticamente a sociedade em que vivem. Além disso, na música ainda são criticados o sistema do neocoronelismo que ainda existe no Brasil e vai além, trazendo o tema da responsabilidade ambiental. Todas essas implicações na vida dos cidadãos se dão, como supracitado, pelo fato de um sistema de ensino deficiente. \\
  Se o ensino público brasileiro fosse eficaz no objetivo de formar cidadãos, não apenas ensinando os saberes relacionados as ciências, mas também, formando os alunos para a importância do exercício da cidadania e de como fazê-la, pressupõe que haveria muito menos ``pacatos cidadãos''. \\
  Cabe a nós, professores, alunos, futuros professores e pesquisadores a fortalecer a relação escola com a formação do cidadão pensante. Que conhece seus direitos e procura diminuir as diferenças sociais. 
  
\end{itemize}
\end{document}
