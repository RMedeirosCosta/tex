\documentclass[a4paper, 12pt]{article}

\usepackage[top=2cm, bottom=2cm, left=2.5cm, right=2.5cm]{geometry}
\usepackage[utf8]{inputenc}
\usepackage{amsmath, amsfonts, amssymb}

\begin{document}
\begin{enumerate}
\item Dados os conjuntos $ A = \{0,1,2,3\} $ e $ B = \{2,3,4,5\} $
  \begin{enumerate}
  \item Obter a relação $ R = \{(x,y) \in A \times B \mid y = x + 1\} $ \newline
    $ R = \{(1,2),(2,3),(3,4)\} $
  \item Representar $R$ em diagrama
  \item Determinar o domínio e a imagem de R. \newline
    $ D(R) = \{1,2,3\} $ e $ Im(R) = \{2,3,4\} $
  \end{enumerate}
\item Com base no exercício anterior, construir $ R^{-1}$ e determinar seu domínio e imagem. \newline
  $ R^{-1} = \{(2,1),(3,2),(4,3)\} $ \newline
  $ Im(R^{-1}) = \{1,2,3\} $ \newline
  $ D(R^{-1}) = \{2,3,4\} $
\item Dados os conjuntos $ A = \{0,1,2,3,4\} $ e $ B = \{-1,0,1,2,3,4,5,6,7\} $ represente as relações abaixo em diagramas e determine os domínios, as imagens e as relações inversas.
  \begin{enumerate}
  \item $ R = \{(x,y) \in A \times B \mid y = x^2\} $ \newline
    $ R = \{(0,0),(1,1),(2,4)\} $ \newline
    $ D(R) = \{0, 1, 2\} $ \newline
    $ Im(R) = \{(0,1,4)\} $ \newline
    $ R^{-1} = \{(0,0),(1,1),(4,2)\} $ \newline
    $ Im(R^{-1}) = \{0,1,2\} $ \newline
    $ D(R^{-1}) = \{0,1,4\} $

  \item $ R = \{(x,y) \in A \times B \mid y = \frac{x}{2} \} $ \newline
    $ R = \{(0,0),(2,1),(4,2)\} $ \newline
    $ D(R) = \{(0,2,4\} $ \newline
    $ Im(R) = \{(0,1,2\} $ \newline
    $ R^{-1} = \{(0,0),(1,2),(2,4)\} $ \newline
    $ D(R^{-1}) = \{0,1,2\} $ \newline
    $ Im(R^{-1}) = \{0,2,4\} $ \newline

  \item $ R = \{(x,y) \in A \times B \mid y = x - 1 \} $ \newline
    $ R = \{(0,-1),(1,0),(2,1),(3,2),(4,3)\} $ \newline
    $ D(R) = \{0,1,2,3,4\} $ \newline
    $ Im(R) = \{-1,0,1,2,3\} $ \newline
    $ R^{-1} = \{(-1,0),(0,1),(1,2),(2,3),(3,4)\} $ \newline
    $ D(R^{-1}) = \{-1,0,1,2,3\} $ \newline
    $ Im(R^{-1}) = \{0,1,2,3,4\} $ \newline

  \item $ R = \{(x,y) \in A \times B \mid y = x^3 - 1 \} $ \newline
    $ R = \{(0,-1),(1,0),(2,7)\} $ \newline
    $ 
    
    
    \end{enumerate}
\end{enumerate}
\end{document}
