\documentclass[a4paper, 12pt]{article}

\usepackage[top=2cm, bottom=2cm, left=2.5cm, right=2.5cm]{geometry}
\usepackage[utf8]{inputenc}
\usepackage{array}
\usepackage{graphicx}

\graphicspath{{img/}}

\begin{document}
\begin{flushleft}\includegraphics{logo}\\
\textbf{UNIVERSIDADE ESTADUAL DE PONTA GROSSA} \\
SISTEMA UNIVERSIDADE ABERTA DO BRASIL - UAB \\
\underline{Licenciatura em Matemática | Polo UAB em Jacarezinho}\end{flushleft} 
\textbf{ALUNO:} Ricardo Medeiros da Costa Junior   \textbf{RA:} 151774301 \\
\textbf{DISCIPLINA:} Instrumentação para o Ensino da Matemática II \\
\textbf{ATIVIDADE:} Tarefa atividade 3  \\
\begin{enumerate}
\item Quais são os objetivos gerais propostos para o Ensino Fundamental brasileiro. Relacione-os. \\ \\
  RESPOSTA AQUI
\item Cite os ciclos (e suas respectivas série/anos) que estruturam o Ensino Fundamental de 9 anos no Brasil, atualmente. \\ \\
  RESPOSTA AQUI
\item As reformas curriculares no ensino de Matemática no Brasil foram influenciadas na década 1960/70 pelo Movimento da Matemática Moderna (MMM). Explique o surgimento desse movimento na organização do ensino da matemática para Educação Básica do Brasil. \\ \\
  RESPOSTA AQUI
\item Em 1980, \textit{National Council of Teachers of Mathematics - NCTM}, nos Estados Unidos publicou um documento com recomendações para o ensino da matemática enfatizando a importância da resolução de problema como metodologia de ensino. Nele estão relacionados os cinco pontos a serem observados nas propostas curriculares de matemática. Quais foram eles? \\ \\
  RESPOSTA AQUI
\item Nos PCN está apresentado da página 21 à 24 um texto sobre o quadro atual do Ensino da Matemática no Brasil. Faça um resumo desse texto ressaltando as características fundamentais para o ensino da matemática. \\ \\
  RESPOSTA AQUI
\item Quais são as principais características do conhecimento matemática, definidas nos PCN? \\ \\
   RESPOSTA AQUI
\item No processo ensino-aprendizagem de Matemática considera-se a relação entre aluno, professor e conhecimento matemática. Pergunta-se: \\ \\ 
  \begin{enumerate}
  \item Quais são as funções do professor nesse processo? \\ \\
     RESPOSTA AQUI
   \item Quais capacidades devem ser desenvolvidas no aluno na relação ensino aprendizagem de matemática ou seja na relação professor-aluno. \\ \\
      RESPOSTA AQUI
  \end{enumerate}
\item Nos PCN estão propostos 8 (oito) objetivos para o ensino da Matemática (página 47). Faça uma síntese desses oito objetivos num texto até 15 linhas. \\ \\
   RESPOSTA AQUI
 \item Os conteúdos selecionados para o ensino da matemática estão organizados em 4 blocos de conteúdo. Quais são eles e que conteúdos os integram? \\ \\
    RESPOSTA AQUI
  \item Quais são os pontos observados para organização dos conteúdos de matemática no ensino fundamental em ciclos. \\ \\
     RESPOSTA AQUI
\item Relacione as colunas quanto à contribuição dos objetivos de matemática do 3 ciclo para o desenvolvimento do pensamento lógico matemático. \\ \\
  \begin{tabular}{m{5cm} m{5cm}}
    (A) Pensamento algébrico & () Resolver situações problemas que envolva figuras geométricas planas utilizando procedimento de decomposição e composição. \\ \\
    (B) Pensamento numérico & () Resolver problemas que envolvam a variação entre grandezas estabelecendo relações sobre elas. \\ \\
    (C) Competência métrica & () Coletar, organizar e analisar informações apresentadas em gráficos e tabelas. \\ \\
    (D) Pensamento geométrico & () Resolver problemas que envolvam diferentes grandezas selecionando unidades de medidas em instrumentos adequados à precisão requerida. \\ \\
    (E) Raciocínio proporcional & () Reconhecer que representações algébricas permite expressar generalizações sobre propriedades das operações aritméticas. \\ \\
    (F) Pensamento estatístico e probabilístico & () Ampliar e construção novos significados para os números naturais, inteiros e racionais.
  \end{tabular}
\end{enumerate}

\end{document}
