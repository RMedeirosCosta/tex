\documentclass[a4paper, 12pt]{article}

\usepackage[top=2cm, bottom=2cm, left=2.5cm, right=2.5cm]{geometry}
\usepackage[utf8]{inputenc}
\usepackage{array}
\usepackage{graphicx}

\graphicspath{{img/}}

\begin{document}
\begin{flushleft}\includegraphics{logo}\\
\textbf{UNIVERSIDADE ESTADUAL DE PONTA GROSSA} \\
SISTEMA UNIVERSIDADE ABERTA DO BRASIL - UAB \\
\underline{Licenciatura em Matemática | Polo UAB em Jacarezinho}\end{flushleft} 
\textbf{ALUNO:} Ricardo Medeiros da Costa Junior   \textbf{RA:} 151774301 \\
\textbf{DISCIPLINA:} Instrumentação para o Ensino da Matemática II \\
\textbf{ATIVIDADE:} Tarefa atividade 5  \\
\begin{enumerate}
\item Faça uma leitura do texto I ``resolução de problemas no ensino-aprendizagem de Matemática'', de Paiva e Rego, das páginas 1 à 4, a partir da leitura efetivada construa uma linha do tempo para evidenciar as principais fases históricas da resolução de problemas em matemática. Considere os períodos abaixo listados: \\ \\
  \begin{tabular}{| c | c |} \hline
    \textbf{Período} & \textbf{Características de cada período} \\
    ANTIGUIDADE & teste & \hline 
    SÉCULO XIX & a &  \hline 
    SÉCULO XX & a & \hline 
    DÉCADA DE 1940 & a & \hline 
    DÉCADA DE 1950 & a & \hline 
    DÉCADA DE 1960/1970 & a & \hline
    DÉCADA DE 1980 & a & \hline 
    DÉCADA DE 1990 & a & \hline 
    ANOS 2000 (ATUALIDADE) & a & \hline 
  \end{tabular} \\
\item O ensino de matemática por meio de resolução de problemas, segundo ONUCHIC (1999), pode ser abordado de três formas diferentes: \\
  \begin{enumerate}
  \item Ensinar sobre resolução de problemas. \\
  \item Ensinar a resolver problemas. \\
  \item Ensinar por meio de resolução de problemas. \\ \\
  \end{enumerate}
  Explique como se faz o ensino de matemática em cada uma delas. \\ \\    

\item O que significa resolução de problemas como metodologia de ensino da Matemática. \\ \\

\item O que significa um problema e um problema matemático?  \\ \\

\item Explique a classificação de problemas, segundo Dante. Exemplifique. \\ \\

\item Segundo a \emph{``Heurística de Polya''}, há um conjunto de etapas para resolução de problemas. Explique cada uma delas. \\ \\

\item Resolva o problema abaixo, segundo a Heurística de Polya. \\
  O elevador de um \textit{shopping} tem um sensor de segurança que apita caso a carga alcance o limite de 770 kg. Certo dia, quando esse elevador parou num dos andares, já transportando várias pessoas, Angélica entrou segurando uma caixa. Imediatamente o sensor tocou, e Angélica saiu deixando a caixa lá dentro. Mas, mesmo assim, o sensor continou a tocar. Angélica resolveu então tirar a caixa e voltar ao elevador e percebeu que, dessa forma, ele funcionou sem problemas. Supondo que, antes da entrada de Angélica com a caixa, o levador estivesse com 700 kg, qual era a massa mínima da caixa? E a máxima?
 
\end{enumerate}
\end{document}
