\documentclass[a4paper, 12pt]{article}

\usepackage[top=2cm, bottom=2cm, left=2.5cm, right=2.5cm]{geometry}
\usepackage[utf8]{inputenc}
\usepackage{array}
\usepackage{graphicx}
\usepackage{longtable}

\graphicspath{{img/}}

\begin{document}
\begin{flushleft}\includegraphics{logo}\\
\textbf{UNIVERSIDADE ESTADUAL DE PONTA GROSSA} \\
SISTEMA UNIVERSIDADE ABERTA DO BRASIL - UAB \\
\underline{Licenciatura em Matemática | Polo UAB em Jacarezinho}\end{flushleft} 
\textbf{ALUNO:} Ricardo Medeiros da Costa Junior   \textbf{RA:} 151774301 \\
\textbf{DISCIPLINA:} Instrumentação para o Ensino da Matemática II \\
\textbf{ATIVIDADE:} Tarefa atividade 5  \\
\begin{enumerate}
\item Faça uma leitura do texto I ``resolução de problemas no ensino-aprendizagem de Matemática'', de Paiva e Rego, das páginas 1 à 4, a partir da leitura efetivada construa uma linha do tempo para evidenciar as principais fases históricas da resolução de problemas em matemática. Considere os períodos abaixo listados: \\ 
  \begin{center}
  \begin{longtable}{| c | p{10cm} |} \hline
    \textbf{Período} & \textbf{Características de cada período} \\
    ANTIGUIDADE & A resolução de problemas esteve na base da criação 
    dos processos de contagem e do conceito de números, 
    a resolução de problemas práticos levou ao desenvolvimento da 
    Matemática pelos romanos, gregos, chineses, mesopotâmios e egípcios. & \hline 
    SÉCULO XIX & Escrita de livros-texto relativos aos conteúdos Matemáticos era integrada por diversos problemas matemáticos, mas apresentavam uma visão limitada da perspectiva da aprendizagem e da capacidade de resolvê-los.
 &  \hline 
    SÉCULO XX & Com a migração da comunidade rural para as cidades ocorreu a necessidade de uma maior quantidade de pessoas possuírem o conhecimento matemático, valorizando assim o ensino desta disciplina, mas este era baseado na repetição de procedimentos e na memorização de regras, formulas e definições. Ao professor cabia ensinar e ao aluno aprender, através de aula expositiva.
 & \hline 
    DÉCADA DE 1940 & A comunidade de educadores e pesquisadores apresentava interesse pela Resolução de Problemas, considerando seu ensino e aprendizagem, surgindo os primeiros resultados de estudos.
 & \hline 
    DÉCADA DE 1950 & Os trabalhos realizados nesta área enfatizavam a aplicação de uma grande quantidade de questões para os alunos, dando importância à resposta e não ao processo. Mas neste período pesquisadores começaram a defender que o ensino deveria valorizar mais as estratégias do que os resultados. & \hline 
    DÉCADA DE 1960/1970 & O ensino de Matemática começou a ser influenciado pelo Movimento da Matemática Moderna, que valorizava o trabalho com base na linguagem da teoria dos conjuntos, na axiomatização, nas estruturas algébricas e na lógica. Valorizava o ensino dos símbolos, de propriedades e o uso da terminologia completa. & \hline
    DÉCADA DE 1980 & Ocorreu o marco da valorização da resolução de problemas, com a publicação do documento “An Agenda for Action: recommendations for school Mathematics of the 1980’s” pena NTCM que apontava a primeira e mais importante das suas recomendações à resolução de problemas como foco do ensino de Matemática. & \hline 
    DÉCADA DE 1990 & Surgiu diversos estudos sobre a utilização da resolução de problemas no processo de ensino-aprendizagem, os que se destacaram foi os trabalhos de autores como Lourdes de la Rosa Onuchic e Luiz Roberto Dante.
 & \hline 
    ANOS 2000 (ATUALIDADE) & Ocorreu à elaboração dos Parâmetros Curriculares Nacionais que discutia a necessidade de um processo de aprendizagem contextualizado com base na realidade dos alunos, preparando-os para o exercício de uma cidadania plena, e para a adoção dos temas transversais, demonstrava a importância dada à resolução de problemas não apenas para a própria Matemática, mas também para as outras áreas de conhecimento.
 & \hline 
  \end{longtable} \\
  \end{center}
\item O ensino de matemática por meio de resolução de problemas, segundo ONUCHIC (1999), pode ser abordado de três formas diferentes: \\
  \begin{enumerate}
  \item Ensinar sobre resolução de problemas. \\ \\
      Ensinar sobre resolução de problemas, constitui-se em demonstrar o estudo sobre os modelos e procedimentos utilizados para a resolução de problemas, destacando a estrutura desses procedimentos.
    \item Ensinar a resolver problemas. \\ \\
      Ensinar a resolver problemas nesta forma o professor se retêm nas formas como os problemas matemáticos podem ser resolvidos e como chegar à sua resposta, focando no processo, no ato ensinar a resolver corretamente os problemas e como utilizar processos, procedimentos e conhecimentos matemáticos para isto ocorra.
      
    \item Ensinar por meio de resolução de problemas. \\ \\
       O ensinar por meio da resolução de problemas é visto não só como ponto de partida para a aprendizagem da Matemática, mas como o meio principal de realiza-lo.
  \end{enumerate}
  Explique como se faz o ensino de matemática em cada uma delas. \\

\item O que significa resolução de problemas como metodologia de ensino da Matemática. \\ \\
  É um dos caminhos metodológicos mais considerados e incentivados pelos pesquisadores da área, eles defendem que esse modelo ajuda a desenvolver a estrutura cognitiva do aluno, exercitar sua criatividade e torná-lo capaz de aprender significativamente conseguindo aplicar o conhecimento conquistado em diferentes contextos da Matemática e em outras áreas de conhecimento, assim como nas situações da vida cotidiana. O trabalho nesta metodologia enfatiza o processo de resolução, possibilitando ao professor analisar a forma como o aluno pensou, identificar quais as estratégias que ele utilizou e identificar quais as dificuldades que ele encontrou no decorrer do processo.

\item O que significa um problema e um problema matemático?  \\ 
  Um problema matemático é uma situação que requer a realização de uma sequência de ações ou operações para obter um resultado. Isto é, que a solução não está acessível de início do problema, mas que possui os dados para sua possível construção. Muitas vezes os problemas apresentados aos alunos não constituem um verdadeiro problema, porque não existe um real desafio nem a necessidade de verificação para validar o processo de solução. O que é problema para um aluno pode não ser para outro.
O problema deve desafiar o aluno a buscar soluções, ele é uma questão não resolvida e que é objeto de discussão, em qualquer domínio do conhecimento, sendo uma questão a ser solucionada e para que isso aconteça precisa utilizar os conhecimentos que possui.

\item Explique a classificação de problemas, segundo Dante. Exemplifique. \\ \\
  \begin{description}
    \item[Exercícios de reconhecimento:]        faz com que os alunos trabalhem funções como reconhecer, identificar, lembrar-se de algum conceito, definição ou propriedade. 
Ex.: Dados os números -5, 6, 7, -2, 10, 1/2, 4², 7/9 quais são números pertencente ao conjunto dos inteiros? 
    \item[Exercícios de algoritmos:]  os alunos devem aplicar um algoritmo estudado anteriormente, este tipo de exercício é utilizado para treinar a aplicação dos algoritmos das operações elementares. 
Ex.: Resolva a expressão algébrica. 8(x – 6) + 4x + 1. 
    \item[Problemas padrões:]  solicita a aplicação direta de um ou mais algoritmos previamente aprendidos, não possui estratégia. São apresentados no final de capítulo nos livros didáticos, possuindo a solução do problema próprio enunciado e a tarefa é transformar a linguagem usual em linguagem matemática, reconhecendo as operações ou algoritmos necessários para resolvê-lo, não desafiando os alunos a desenvolverem estratégias. 
Ex.: Maria possui 60 balas e quer dar para seus 10 amigos em quantidades iguais. Quantas balas Maria irá dar a cada um deles? 
    \item[Problemas-processo ou heurísticos:]  proporciona ao aluno a chance de pensar e arquitetar um plano de ação, uma estratégia que poderá levá-lo à solução do problema, desafiando os alunos a pensarem em estratégia de resolução, tornando-se mais interessantes do que os problemas padrões, com estes problemas os alunos desenvolvem a criatividade, a iniciativa e o espírito explorador. Estes problemas valorizam mais a estratégia e os procedimentos utilizados que a própria resposta correta.
Ex: Dez amigos se encontram e trocam presentes. Quantos presentes são trocados, se cada um deu um presente ao outro uma única vez? E se ao invés de dez, fossem quinze amigos, quantos presentes seriam dados? 
    \item[Problemas de aplicação ou situações-problema:]  buscam responder alguma situação contextualizada que precisa da Matemática para a sua resolução, matematizam situações do cotidiano. A maioria de problemas desse tipo utilizam tabelas e gráficos e podem envolver outras áreas de ensino.
Ex.: Na loja “Americanas”, um par de chuteiras custava R\$ 50,00. No ano passado, foram vendidos para a Seleção Brasileira 20.736 pares de chuteiras. Qual o total em reais que a loja arrecadou na venda das chuteiras?
    \item[Problemas de quebra-cabeça:] desafiam o aluno a perceber um artifício para a obtenção da resposta do problema, dependendo muitas vezes mais da percepção desse artifício do que da utilização de algoritmos matemáticos. 
      Ex.: O Sr. Brown tem 6 luvas pretas e 12 luvas marrom em seu guarda-roupa. Sem olhar, ele pega algumas luvas do guarda-roupa. Qual o mínimo número de luvas que o Sr. Brown terá que pegar para ter certeza que encontrou um par de luvas da mesma cor?
      
  \end{description}

\item Segundo a \emph{``Heurística de Polya''}, há um conjunto de etapas para resolução de problemas. Explique cada uma delas. \\
  \begin{description}
  \item[1ª etapa: Compreender o Problema -]
    Compreender bem o problema é a fase inicial e indispensável do processo, é necessária a familiarização com os aspectos envolvidos, para que o aluno consiga compreender o problema. O professor pode fazer questionamentos sobre o problema, levando o aluno a enxergar o problema através de vários pontos de vista. O aluno precisa acima de tudo desejar resolvê-lo, porque só compreender não é suficiente, sendo importante a escolha do problema de acordo com aspectos cognitivos, culturais e sociais dos alunos. O professor deve auxiliar o processo referente à capacidade de fazer perguntas relativas ao problema até que todos os alunos sejam capazes de, por si só, conduzirem o processo.
  \item[2ª etapa: Elaborar um Plano de Ação –]
    Primeiramente deve ser estabelecida uma conexão entre os dados presentes no problema e o que ele pede, partindo da linguagem usual e chegando a uma sentença matemática. O caminho percorrido para a compreensão do problema até chegar à ideia de um plano nem sempre é fácil. Esta ideia de plano pode surgir de forma gradual, através de tentativas fracassadas, ou por meio de ideias repentinas e imediatas. Mas tudo depende do problema, dos conhecimentos prévios dos alunos e de suas experiências em relação à resolução de problemas. Quanto maior a variedade de problemas eles resolvem, maior a quantidade de referências que possuem.
  \item[3ª etapa: Executar o Plano -]
    é necessário colocar em prática as estratégias elaboradas na etapa anterior, observando, analisando e registrando cada passo a ser dado, efetuando os cálculos e procedimentos necessários. Quanto mais detalhada for à etapa de elaboração do plano, mais fácil será sua execução. O aluno obtém o que é convencionalmente chamado de resposta do problema. Nesta etapa pode ser identificada as dificuldades dos alunos com a utilização de alguns algoritmos ou na compreensão de alguns conceitos matemáticos.
  \item[4ª etapa: Verificar –]
    Nesta etapa é realizada uma análise retrospectiva do processo, é um excelente exercício de aprendizagem, serve para detectar e corrigir os possíveis enganos. Ao analisarmos a solução obtida pelo aluno, estamos repassando todo o percurso de resolução do problema, fazendo com que o aluno reveja como pensou inicialmente, como encaminhou a estratégia de solução, como efetuou os cálculos e a pertinência da resposta encontrada. Sendo levantado se necessário alguns questionamentos sobre outras possibilidades de resolução, caminhos que levem a uma resposta de forma mais rápida ou utilizando outra forma de pensamento. 
  \end{description}  

\item Resolva o problema abaixo, segundo a Heurística de Polya. \\
  O elevador de um \textit{shopping} tem um sensor de segurança que apita caso a carga alcance o limite de 770 kg. Certo dia, quando esse elevador parou num dos andares, já transportando várias pessoas, Angélica entrou segurando uma caixa. Imediatamente o sensor tocou, e Angélica saiu deixando a caixa lá dentro. Mas, mesmo assim, o sensor continou a tocar. Angélica resolveu então tirar a caixa e voltar ao elevador e percebeu que, dessa forma, ele funcionou sem problemas. Supondo que, antes da entrada de Angélica com a caixa, o levador estivesse com 700 kg, qual era a massa mínima da caixa? E a máxima?
  \begin{description}
  \item[Fase 1 - Compreender o Problema:]
    Neste exemplo procura-se o peso da caixa. Anota-se os dados. Elevador suporta 770kg. Angélica com a caixa fazia o elevador apitar, sem a caixa não. Mesmo apenas a caixa dentro do elevador, ele continuava a apitar. Como é suposto que o elevador já possuia 700 kg antes de Angélica entrar, estima-se que a caixa tenha mais de 71 kg e Angélica tenha 70 kg ou menos. 
  \item[Fase 2 - Elaborar um Plano:] Pode-se usar qualquer um dos planos de ação para resolver esse problema.
  \item[Fase 3 – Execução do Plano:] Nesta fase é colocado todos os planos de ação em prática.
    \begin{description}
    \item[i) Na execução da Dramatização:] através da representação de um quadrado no chão representando o elevador, uma média de cinco alunos dentro desse quadrado representando que o elevador esta quase cheio e um aluno ou aluno representando o personagem Angélica e uma caixa, seria analisado o que foi representado no problema, chegando a conclusão que Angélica tem que pesar 70 kg ou menos e a caixa deve pensar 71 kg ou mais.
    \item[ii) Por tentativa e erro:] nessa opção pode-se escrever os números referente ao peso de Angélica e da caixa, um por um até alcançar os valores que supririam a resolução de todas as questões do enunciado.
    \item[iii) Redução ao que tem menos:] podemos raciocinar da seguinte maneira, se Angélica e a caixa possuíssem o mesmo valor cada uma sozinhas poderia subir no elevador, ou se fosse dividido o peso de 70 kg entre eles os dois poderiam subir, mas como apenas Angélica sem a caixa consegue compreende que ela sozinha pode englobar os 70 kg e a caixa sozinha é maior que os 70 kg.
    \item[iv) De maneira análoga pode-se fazer a redução ao que tem mais:] compreende que a caixa é igual ou maior que 71 kg, porque ela sozinha não pode ficar no elevador e o peso restante para que o elevador esteja lotado é de 70 kg.
    \item[v) Representação Algébriga:] Atribuem-se variáveis aos dados do problema. Assim, \emph{a} seria Angélica, \emph{e} o elevador com 700kg, \emph{el} o elevador lotado (770 kg) e \emph{c} a caixa, sendo assim ficaria: $ a+e < el; a+e+c > el; c+e > el $.
    \item[vi) Representação Gráfica:] Desenha-se figuras que representam o elevador lotado, a caixa e Angélica, desenhando a caixa dentro do elevador ele não sairia do lugar (apitaria), desenhando Angélica e a caixa dentro do elevador ele continuaria sem sair do lugar (apitaria) e desenhando apenas Angélica poderíamos desenhar o elevador subindo ou descendo.
    \end{description}
\item[Fase 4 – Verificação:] nessa etapa analisa-se a solução e verifica-se se o resultado obtido realmente satisfaz as condições dispostas no problema. No retrospecto o aluno refaz os passos dados na execução do plano, verifica como efetuou os cálculos e como foi concretizada a estratégia da resolução, funcionando como um processo de feedback.

  \end{description}
 
\end{enumerate}
\end{document}
