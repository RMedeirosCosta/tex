\documentclass[a4paper, 12pt]{article}

\usepackage[top=2cm, bottom=2cm, left=2.5cm, right=2.5cm]{geometry}
\usepackage[utf8]{inputenc}
\usepackage{array}
\usepackage{graphicx}

\graphicspath{{img/}}

\begin{document}
\begin{flushleft}\includegraphics{logo}\\
\textbf{UNIVERSIDADE ESTADUAL DE PONTA GROSSA} \\
SISTEMA UNIVERSIDADE ABERTA DO BRASIL - UAB \\
\underline{Licenciatura em Matemática | Polo UAB em Jacarezinho}\end{flushleft} 
\textbf{ALUNO:} Ricardo Medeiros da Costa Junior   \textbf{RA:} 151774301 \\
\textbf{DISCIPLINA:} Instrumentação para o Ensino da Matemática II \\
\textbf{ATIVIDADE:} Tarefa atividade 3  \\
\begin{enumerate}
\item Quais são os objetivos gerais propostos para o Ensino Fundamental brasileiro. Relacione-os. \\
  \begin{itemize}
  \item Compreensão da cidadania como uma participação social e política, desenvolvendo o exercício dos direitos e deveres políticos, civis e sociais no dia a dia
  \item Posicionamento crítico, responsável e construtivo nas diferentes situações sociais, buscando a utilização do diálogo como forma de mediar conflitos e de tomar decisões sociais, buscando a utilização do diálogo como forma de mediar conflitos e de tomar decisões coletivamente
  \item Conhecimento das características fundamentais do Brasil no aspecto social, material e cultural como meio de construir a noção de identidade nacional e pessoal
  \item Conhecer e valorizar a pluralidade do patrimônio sociocultural brasileiro, aspectos culturais de outros povos e nações, assumindo opinião contra qualquer discriminação relacionada às diferenças culturais, de classe social, de crenças, de sexo, de etnia ou outras características individuais e sociais.
  \item Demonstrar como todos são agentes transformadores do ambiente, discernindo seus elementos e as interações, contribuindo ativamente para melhoria do ambiente.
  \item Necessidade da ampliação do conhecimento ajustado de si mesmo e o sentimento de confiança com suas capacidades afetiva, física, cognitiva, ética, estética, de inter-relação pessoal em conjunto com a inserção social.
  \item Conhecimento do próprio corpo e como cuidar dele, valorizando e praticando hábitos saudáveis como um dos aspéctos básicos da qualidade de vida e agindo com responsabilidade em relação à sua saúde e à saúde coletiva.
  \item Utilização de diferentes linguagens para produzir, expressar e comunicar ideias, interpretar e usufruir das produções culturais, em contextos públicos e privados, atendendo a diferentes intenções e situações de comunicação.
  \item Utilizar diferentes fontes de informação e recursos tecnológicos para desenvolver o conhecimento.
  \item Questionar a realidade formulando e resolvendo problemas utilizando o pensamento lógico, a criatividade, intuição, a capacidade de análise crítica, selecionando procedimentos e certificando sua adequação.    
  \end{itemize}
\item Cite os ciclos (e suas respectivas série/anos) que estruturam o Ensino Fundamental de 9 anos no Brasil, atualmente.
  \begin{description}
  \item[1$^{\circ}$ Ciclo:] 1$^{\circ}$, 2$^{\circ}$ e 3$^{\circ}$ anos
  \item[2$^{\circ}$ Ciclo:] 4$^{\circ}$ e 5$^{\circ}$ anos
  \item[3$^{\circ}$ Ciclo:] 6$^{\circ}$ e 7$^{\circ}$ anos
  \item[4$^{\circ}$ Ciclo:] 8$^{\circ}$ e 9$^{\circ}$ anos
  \end{description}
\item As reformas curriculares no ensino de Matemática no Brasil foram influenciadas na década 1960/70 pelo Movimento da Matemática Moderna (MMM). Explique o surgimento desse movimento na organização do ensino da matemática para Educação Básica do Brasil. \\ \\
 O MMM foi um movimento que propunha aproximar a Matemática oferecida nas escolas, da Matemática vista pelos estudiosos e pesquisadores. No entando o projeto fracassou no Brasil, pois os alunos careciam de conhecimento para conseguir realizar as demonstrações formais, cujo haviam grande discrepância da realidade prática. Esse movimento impulsionou reformas e discussões sobre o currículo de matemática no Brasil. O movimento era conduzido pelos livros didáticos, mas só veio a retroceder após a constatação dos exageros em alguns de seus princípios básicos.
\item Em 1980, \textit{National Council of Teachers of Mathematics - NCTM}, nos Estados Unidos publicou um documento com recomendações para o ensino da matemática enfatizando a importância da resolução de problema como metodologia de ensino. Nele estão relacionados os cinco pontos a serem observados nas propostas curriculares de matemática. Quais foram eles?
  \begin{enumerate}
  \item Condução do ensino fundamental para a aquisição das competências básicas necessárias ao cidadão de forma global
  \item A relevância do papel ativo do aluno na construção do seu conhecimento
  \item O destaque da resolução de problemas nos conteúdos da Matemática utilizando como base os problemas vividos no cotidiano ou das outras disciplinas
  \item Relevância do trabalho com amplos aspectos de conteúdos
  \item Indispensabilidade de fazer os alunos compreender a importância do uso da tecnologia e a acompanhar sua permanente renovação.
  \end{enumerate}
\item Nos PCN está apresentado da página 21 à 24 um texto sobre o quadro atual do Ensino da Matemática no Brasil. Faça um resumo desse texto ressaltando as características fundamentais para o ensino da matemática. \\ \\
No PCN são elencados alguns motivos para altas taxas de retenção em Matemática para os alunos do ensino fundamental. Carência de formação profissional qualificada e más condições de trabalho, assim como inexistência de políticas educacionais efetivas e interpretações errôneas das concepções pedagógicas são elucidadas logo de início. Outro ponto importante levantado pelo PCN, são que os professores organizam os conteúdos de forma hierárquica, fazendo como se cada conteúdo fosse pré-requisito para o próximo. Embora a priori essa concepção esteja correta, a rigidez imposta por certos professores pode atrapalhar o desenvolvimento do aluno. Além disso, o PCN faz algumas críticas na forma como uma parcela de professores trabalham os conteúdos com seus alunos, ignorando a história da matemática e na forma como é aplicado a resolução de problemas, por exemplo. Em contrapartida, o documento demonstra que há instituições que elaboram projetos que atendem as necessidades da comunidade e professores que buscam novos conhecimentos refletindo suas práticas. Para finalizar, no PCN consta uma crítica a sugestão de utilização de recursos tecnológicos que há nas propostas curriculares, no qual, segundo o documento não há na clareza de como devem ser usados esses recursos por parte dos professores.
\item Quais são as principais características do conhecimento matemática, definidas nos PCN? \\ \\
 Hoje a matemática é vista como uma ciência em constante evolução, no entanto, não era essa a concepção que os estudiosos tinham sobre ela. Essa ciência, que foi outrora considera imutável e estática, hoje é tida como como ferramenta para resoler problemas científicos e tecnológicos. Segundo o PCN, a matemática é movida por duas forças: Aplicação as atividades humanas e a especulação. As duas atuam de forma sinérgica. O modelo matemático contemporâneo é derivado da civilização grega, nos quais seus sistemas são formados inicialmente por axiomas e são construídos por intermédio de raciocínio lógico, isso só foi possível após a criação da Teoria dos Conjuntos e da Lógica Matemática no século XIX. A matematica é utilizada em outras áreas científicas, tais como Física, Ciência da Computação e Química.
\item No processo ensino-aprendizagem de Matemática considera-se a relação entre aluno, professor e conhecimento matemática. Pergunta-se: 
  \begin{enumerate}
  \item Quais são as funções do professor nesse processo? \\
    O professor possui o papel de mediador, deve possuir profundo conhecimento dos conceitos. É sua obrigação transpor o saber matemático em saber pedagógico.
   \item Quais capacidades devem ser desenvolvidas no aluno na relação ensino aprendizagem de matemática ou seja na relação professor-aluno. \\
    O aluno é agente de sua aprendizagem. A interação entre professor-aluno é primordial para aprendizagem, no entanto a relação com os demais colegas também é responsável pelo desenvolvimento da construção do seu conhecimento.
  \end{enumerate}
\item Nos PCN estão propostos 8 (oito) objetivos para o ensino da Matemática (página 47). Faça uma síntese desses oito objetivos num texto até 15 linhas. \\ \\
 De acordo com o PCN, os objetivos do ensino fundamental de matemática são: identificar os conhecimentos matemáticos como meios para compreender e transformar o mundo a sua volta. Fazer observações acerca da realidade, estabelecendo relações entre o conhecimento matemático e o meio. Gerenciar informações relevantes e interpretá-las posteriormente. Resolver situações-problema, desenvolvendo forma de raciocínio e processos. Saber utilizar a linguagem matemática para se comunicar, estabelecer conexões entre termos matemáticos de diferentes áreas curriculares, desenvolver auto-estima e perseverança na busca por soluções e interagir com seus pares de forma cooperativa na busca por soluções para problemas propostos.
 \item Os conteúdos selecionados para o ensino da matemática estão organizados em 4 blocos de conteúdo. Quais são eles e que conteúdos os integram? 
   \begin{description}
   \item[Números e Operações:] O aluno terá contato com diversos tipos ou conjuntos numéricos, seus significados; operações e medidas de grandezas envolvendo-os. Além disso, as operações serão aprofundadas no estudo do cálculo, sendo eles: exato e aproximado, mental e escrito. Nas séries finais as atividades algébricas serão ampliadas, porém a abordagem formal do conceito de generalização só será objeto de estudo do ensino médio.
   \item[Espaço e Forma:] Envolve resolver situações problemas sobre figuras geométricas planas utilizando procedimento de decomposição e composições.
   \item[Grandezas e Medidas:] Envolve problemas sobre variação entre grandezas estabelecendo relações sobre elas.
   \item[Tratamento de Informação:] Trabalho em base de conceitos de estatística, o aluno aprenderá a coletar, organizar, comunicar dados, utilizando gráficos e representações que aparecem frequentemente no cotidiano.
   \end{description}
 \item Quais são os pontos observados para organização dos conteúdos de matemática no ensino fundamental em ciclos.
   \begin{itemize}
   \item Estabelecer conexão entre Matemática com situações cotidianas e outra s áreas do conhecimento
   \item Hierarquizar conteúdos por meio de ligações estabelecidas e dos conhecimentos previamente construído pelos alunos
   \item Organizar os conteúdos devendo chegar a um nível de sistematização para que possam ser aplicados em novas situações
   \item Organizar os níveis de aprofundamento de acordo com a capacidade de compreensão dos alunos
   \item Estabelecer qual item é necessário mais ênfase e qual pode ser moderado.
   \end{itemize}
\item Relacione as colunas quanto à contribuição dos objetivos de matemática do 3 ciclo para o desenvolvimento do pensamento lógico matemático. \\ \\
  \begin{tabular}{m{5cm} m{5cm}}
    (A) Pensamento algébrico & (D) Resolver situações problemas que envolva figuras geométricas planas utilizando procedimento de decomposição e composição. \\ \\
    (B) Pensamento numérico & (E) Resolver problemas que envolvam a variação entre grandezas estabelecendo relações sobre elas. \\ \\
    (C) Competência métrica & (F) Coletar, organizar e analisar informações apresentadas em gráficos e tabelas. \\ \\
    (D) Pensamento geométrico & (C) Resolver problemas que envolvam diferentes grandezas selecionando unidades de medidas em instrumentos adequados à precisão requerida. \\ \\
    (E) Raciocínio proporcional & (A) Reconhecer que representações algébricas permite expressar generalizações sobre propriedades das operações aritméticas. \\ \\
    (F) Pensamento estatístico e probabilístico & (B) Ampliar e construção novos significados para os números naturais, inteiros e racionais.
  \end{tabular}
\end{enumerate}

\end{document}
