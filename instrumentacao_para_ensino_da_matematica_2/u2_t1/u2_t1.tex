\documentclass[a4paper, 12pt]{article}

\usepackage[top=2cm, bottom=2cm, left=2.5cm, right=2.5cm]{geometry}
\usepackage[utf8]{inputenc}
\usepackage{array}
\usepackage{graphicx}

\graphicspath{{img/}}

\begin{document}
\begin{flushleft}\includegraphics{logo}\\
\textbf{UNIVERSIDADE ESTADUAL DE PONTA GROSSA} \\
SISTEMA UNIVERSIDADE ABERTA DO BRASIL - UAB \\
\underline{Licenciatura em Matemática | Polo UAB em Jacarezinho}\end{flushleft} 
\textbf{ALUNO:} Ricardo Medeiros da Costa Junior   \textbf{RA:} 151774301 \\
\textbf{DISCIPLINA:} Instrumentação para o Ensino da Matemática II \\
\textbf{ATIVIDADE:} Tarefa atividade 2 \underline{VÍDEO}\\ \\
Os Parâmetros Curriculares Nacionais são documentos que foram criados com base na Constituição Federal de 1988. Eles foram criados por profissionais da educação e voltado para as reais necessidades do ensino. Estes documentos estabelecem conteúdos curriculares mínimos para a educação nacional. Além disso, os PCNs também propõe orientações didáticas aos professores e as instituições de ensino proporcionando flexibilidade para que estes conteúdos possam ser adaptados para determinada realidade.
\end{document}
