\documentclass[a4paper, 12pt]{article}

\usepackage[top=2cm, bottom=2cm, left=2.5cm, right=2.5cm]{geometry}
\usepackage[utf8]{inputenc}
\usepackage{array}

\begin{document}\textbf{UNIVERSIDADE ESTADUAL DE PONTA GROSSA} \\
SISTEMA UNIVERSIDADE ABERTA DO BRASIL - UAB \\
Licenciatura em Matemática | Polo UAB em Jacarezinho \\ \\
\textbf{ALUNO:} Ricardo Medeiros da Costa Junior \textbf{RA:} 151774301 \\
\textbf{DISCIPLINA:} Instrumentação para o Ensino da Matemática II \\
\textbf{ATIVIDADE:} Tarefa atividade 1

\begin{enumerate}
\item O Professor Dario Fiorentini apresenta a evolução histórica das Tendências do ensino da Matemática: Tendência Empírico-ativista, Tendência Formalista Moderna, Tendência Tecnicista, Tendência Construtivista, Tendência Histórico-crítica e Tendência Sócioetnocultura. Monte um quadro com a principal característica de cada uma delas no processo ensino-aprendizagem. \\ \\

  \begin{tabular}{ m{2cm} | m{2cm} | m{2cm} | m{2cm} | m{2cm} | m{2cm} |}
    \textbf{Empírico-ativista} & \textbf{Formalista Moderna} & \textbf{Tecnicista} & \textbf{Constr.} & \textbf{Histórico-crítica} & \textbf{Sócio-etnocultural} \\ \hline
    Utilizam-se atividades experimentais, resolução de problemas e o método científico acreditando-se que o aluno aprende fazendo. &
    Ênfase no uso da linguagem, no rigor e justificativas. &
    Ênfase na apresentação de conteúdos como uma instrução programada. &
    Destaca-se o aprender a aprender e o desenvolvimento do pensamento lógico-formal. &
    Acontece quando o aluno consegue atribuir sentido e significado as ideias matemáticas. &
    Parte-se de problemas da realidade que inseridos em diversos grupos culturais, gerarão temas de trabalho da sala de aula.
  \end{tabular}
\item Explique o que significa uma Tendência em Educação Matemática. \\ \\
  É uma forma de trabalho que surge a partir de busca por soluções para problemas do ensino e aprendizagem da matemática.
\item Cite as atuais Tendências em Educação Matemática. \\ \\
  \textbf{Etnomatemática:} objetiva descrever as práticas matemáticas para grupos culturais a partir da análise das relações entre conhecimento matemático e o contexto cultural. \\ \\  
  \textbf{Educação matemática crítica:} um movimento que discute questões relacionadas aos aspectos políticos da educação matemática e suas relações de poder. \\ \\
  \textbf{História da matemática:} considera a história da matemática como um recurso pedagógico de grande valor. \\ \\
  \textbf{Mídias tecnológicas:} considera a utilização de mídias tecnológicas. \\ \\
  \textbf{Modelagem matemática:} é a arte de expressar, por intermédio da linguagem matemática, situações-problemas reais. \\ \\
  \textbf{Resolução de problemas:} considera um problema como um recurso de aprendizagem para que o aluno construa seus conhecimentos a partir da interação com o professor e outros alunos. \\ \\
  \textbf{Escrita matemática:} considera a escrita e produção de texto, como estratégia geradora de processos reflexivos a respeito da compreensão individual sobre o conteúdo matemático em estudo.
 \end{enumerate}
\end{document}
