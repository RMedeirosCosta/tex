\documentclass[a4paper, 12pt]{article}

\usepackage[top=2cm, bottom=2cm, left=2.5cm, right=2.5cm]{geometry}
\usepackage[utf8]{inputenc}
\usepackage{array}
\usepackage{graphicx}

\graphicspath{{img/}}

\begin{document}
\begin{flushleft}\includegraphics{logo}\\
\textbf{UNIVERSIDADE ESTADUAL DE PONTA GROSSA} \\
SISTEMA UNIVERSIDADE ABERTA DO BRASIL - UAB \\
\underline{Licenciatura em Matemática | Polo UAB em Jacarezinho}\end{flushleft} 
\textbf{ALUNO:} Ricardo Medeiros da Costa Junior   \textbf{RA:} 151774301 \\
\textbf{DISCIPLINA:} Instrumentação para o Ensino da Matemática II \\
\textbf{ATIVIDADE:} Tarefa atividade 4  \\
\begin{enumerate}
\item Leia e estude o texo I, da unidade III, e responda as questões abaixo: \\ \\
  \begin{enumerate}
  \item Qual é o papel do Material Didático de Matemática nas metodologias de ensino voltadas para um processo de aprendizagem significativa? \\ \\
    RESPOSTA AQUI \\ \\
  \item Quais são as principais funções do material didático no ensino da matemática segundo o matemático Manoel Jairo Bezerra? \\ \\
  \item Qual é a importância da utilização de jogos matemáticos como metodologia de ensino na formação dos alunos? \\ \\
  \item Explique o que os PCN de Matemática para o Ensino Fundamental propõe para utilização de jogos como metodologia de ensino nas aulas de Matemática. \\ \\
  \end{enumerate}

\item Assista o vídeo ``Jogos e Matemática'' que está disponível na Videoteca. Faça uma análise sobre a utilização dos jogos pelas professoras da escola pesquisada e os seus comentários sobre as contribuições e desafios do ensino da matemática com a utilização dessa metodologia. Observe como as professoras evidenciam os objetivos dos PCN para o ensino da matemática. \\ \\
\item Após assistir o vídeo ``Fliperauta City'' disponível nos endereços abaixo, escolha um dos jogos realizados no vídeo e analise respondendo as questões: \\ \\
  \begin{enumerate}
  \item Explique qual o jogo escolhido, suas regras, a forma de jogá-lo e como as crianças no desenho determinaram a chance de ganhar. \\ \\
  \item Quais são os conteúdos matemáticos e os eixos de conteúdo dos PCN de Matemática nos quais se encaixam os jogos apresentados no desenho. \\ \\
  \item Esses jogos podem ser adaptados e utilizados na sala de aula? Explique como a metodologia de jogos pode ser utilizada em sala de aula segundo o PCN? \\ \\
  \end{enumerate}
\item Considere as ``situações de ensino'' elaboradas por Fernanda Fetzer, disponíveis em ``material obrigatório'' na Unidade III, no AVA. Elas são exemplos de como o professor pode utilizar os jogos como metodologia de ensino de Matemática. Resolva cada uma delas passo a passo e inclua no envio da tarefa o arquivo com as resoluções. Observe que elas são propostas para o 6$^{\circ}$ ano do Ensino Fundamental e que vocês deverá colocar-se no lugar do aluno dessa faixa etária de escolarização. Portanto, resolva etapa por etapa.
\end{enumerate}
\end{document}
