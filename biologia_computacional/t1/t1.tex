\documentclass[a4paper, 12pt]{article}
\usepackage[top=2cm, bottom=2cm, left=2.5cm, right=2.5cm]{geometry}
\usepackage[utf8]{inputenc}
\usepackage{hyperref}

\begin{document}
\textbf{UNIVERSIDADE TECNOLÓGICA FEDERAL DO PARANÁ}\\
\centerline{\underline{Biologia Computacional e Sistêmica | PPGBIOINFO}}\\\\
\textbf{ALUNO:} Ricardo Medeiros da Costa Junior   \textbf{RA:} a1598996 \\
\textbf{DISCIPLINA:} Biologia Computacional \\
\textbf{ATIVIDADE:} Tarefa 1 \\
\begin{itemize} 
\item Quais as recomendações listadas no artigo que você já está utilizando? \\ \\
  Sobre as recomendações \emph{teóricas} - as que estão em negrito, organizada por tópicos - penso realizar quase todas. Entender os objetivos e escolher os métodos apropriados é uma dica que deve ser realizado em qualquer trabalho científico, assim como escolher a ferramenta certa para o trabalho (que pode ser até considerado uma dica redundante) e procurar primeiro se alguém fez antes. As outras dicas como: \emph{Faça armadilhas para seus scripts}, cujo provê algumas recomendações para criar validações para todos os resultados esperados; \emph{Seja desconfiado} e \emph{Seja um detetive} os quais, de forma resumida, dizem para ser um pouco pessimista, não confiar cegamente nos resultados e procurar interpretar de forma criterioza os dados, são práticas que é preciso ter quando se trabalha com análise de dados. Usar uma ferramenta de versionamento e fazer funcionar primeiro, depois deixar o código limpo e documentado - citado por \emph{You're a scientist, not a programmer} - são abordagens que deveriam ser empregados em qualquer projeto de desenvolvimento de software (apesar de concordar com o autor, com relação que muitos desenvolvedores se preocupam em usar padrões e documentar excessivamente). Utilizar a filosofia do Obama, apesar de ser bem engraçado, faz muito sentido. As dificuldades irão aparecer, é preciso ter isso em mente, portanto é preciso ter a perseverança e não desistir. Ao meu ver, a surpresa acerca das dicas teóricas é quando o autor cita uma suposta \emph{pipelineitis}, o que aparenta ser uma obsessão sobre a construção de um pipeline logo de início, sem se atentar se os passos separados estão funcionado. Parace ser semelhante a \emph{febre de padrões de projeto} que atinge alguns desenvolvedores de software, que pretendem utilizar padrões de projeto sem se atentar se são realmente necessários para aquele problema específico.  \\
  Sobre as dicas práticas, que estão nas tabelas, sou acostumado com a maioria. Uso sistema Unix-like faz muito tempo. Também tenho experiência com o make, que segundo o texto, pode ser empregado para  criar pipelines. Assim como usar ferramentas para distribuição de código fonte (como o Git), usar IDE (como o Emacs), linguagem de script. O meu \href{http://github.com/RMedeirosCosta}{Github} tem alguns projetos.
\item O que você pode fazer para se tornar um(a) Bioinformata melhor? \\\\
  Obter conhecimento sobre o domínio biológico vai ser essencial. O texto sugere alguns cursos online como o Khann Academy e o Udacity. Além de alguns fóruns específicos de Bioinformática. Com um maior conhecimento sobre o domínio fica mais fácil identificar problemas da Bioinformática e pode ser útil na interpretação dos resultados.
\end{itemize}
\end{document}
